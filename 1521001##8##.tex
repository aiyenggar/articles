%1521001##1##.tex
\documentclass[12pt,letterpaper]{article}
\usepackage{mathptmx}
\usepackage[margin=1in]{geometry}

\usepackage{tabularx, longtable}  % for 'tabularx' environment and 'X' column type
\usepackage{ragged2e}  % for '\RaggedRight' macro (allows hyphenation)
\newcolumntype{Y}{>{\RaggedRight\arraybackslash}X} 
\usepackage{booktabs}  % for \toprule, \midrule, and \bottomrule macros 

\usepackage{setspace}
\singlespacing
  
\usepackage{amssymb,latexsym}
\usepackage[round,sort]{natbib}
\usepackage{fancyhdr}
\usepackage{lastpage}
\usepackage{graphicx,multirow}
\graphicspath{ {qe8/} }

% Bold Table and Figure captions
\usepackage{caption}
\captionsetup{figurename=FIGURE}
\captionsetup{tablename=TABLE}
\captionsetup[figure]{labelfont=bf}
\captionsetup[table]{labelfont=bf}
  
% Turns off all section numbering
\setcounter{secnumdepth}{0} 

  % Places all tables at end of document and creates AOM-style table-here placeholders
  \usepackage[nolists]{endfloat} % Places all figures and charts at end of manuscript and adds 'insert table x about here' lines.
  \renewcommand{\figureplace}{
    \begin{center}
    \begin{singlespace}
    ------------------------------------\\
    Insert \figurename \ \thepostfig\ about here.\\
    ------------------------------------
    \end{singlespace}
    \end{center}}
  \renewcommand{\tableplace}{%
    \begin{center}
    \begin{singlespace}
    ------------------------------------\\
    Insert \tablename \ \theposttbl\ about here.\\
    ------------------------------------
    \end{singlespace}
    \end{center}}

  \usepackage{titlesec}
   \titleformat{\title}
       {\filcenter\normalfont\bfseries\uppercase}{\thetitle}{1em}{}
  \titleformat{\section}
    {\filcenter\normalfont\bfseries\uppercase}{\thesection}{1em}{}
  \titleformat{\subsection}
    {\normalfont\bfseries}{\thesubsection}{1em}{}
  \titleformat{\subsubsection}[runin]
   {\normalfont\bfseries\slshape}{\thesubsubsection}{1em}{\hspace*{\parindent}}
       
\usepackage{tabu}
\usepackage{textcomp}
\usepackage{listings}
\usepackage{hyperref}
\usepackage{verbatim}
\usepackage{tabu}
\hypersetup{
    colorlinks=true,
    linkcolor=blue,
    filecolor=cyan,      
    urlcolor=cyan,
    citecolor=blue,
}

\usepackage{etoolbox}

\makeatletter

% Patch case where name and year are separated by aysep
\patchcmd{\NAT@citex}
  {\@citea\NAT@hyper@{%
     \NAT@nmfmt{\NAT@nm}%
     \hyper@natlinkbreak{\NAT@aysep\NAT@spacechar}{\@citeb\@extra@b@citeb}%
     \NAT@date}}
  {\@citea\NAT@nmfmt{\NAT@nm}%
   \NAT@aysep\NAT@spacechar\NAT@hyper@{\NAT@date}}{}{}

% Patch case where name and year are separated by opening bracket
\patchcmd{\NAT@citex}
  {\@citea\NAT@hyper@{%
     \NAT@nmfmt{\NAT@nm}%
     \hyper@natlinkbreak{\NAT@spacechar\NAT@@open\if*#1*\else#1\NAT@spacechar\fi}%
       {\@citeb\@extra@b@citeb}%
     \NAT@date}}
  {\@citea\NAT@nmfmt{\NAT@nm}%
   \NAT@spacechar\NAT@@open\if*#1*\else#1\NAT@spacechar\fi\NAT@hyper@{\NAT@date}}
  {}{}

\lstset{
basicstyle=\ttfamily,
columns=flexible,
breaklines=true
}
\newenvironment{hypothesis}{
  	\itshape
  	\leftskip=\parindent \rightskip=\parindent
  	\noindent\ignorespaces}

\fancypagestyle{plain}{
  \renewcommand{\headrulewidth}{0pt}
  \fancyhf{}
}	


\begin{document}
\title{General Purpose Technologies and the Exploration-Exploitation Tradeoff}
\date{}
\maketitle

\begin{abstract} 
\normalsize 
We apply a f
\end{abstract}


{\textbf{Keywords:} \\\indent Capabilities, Complementarity, Firm level strategy, General Purpose Technology}

\newpage
\pagestyle{fancy}
\fancyhf{}
\lhead{GPTs and the Exploration-Exploitation Tradeoff}
\rhead{\thepage}

\begin{center}
\textbf{GPTs and the Exploration-Exploitation Tradeoff}
\end{center}
``Are there such things as \textquotesingle technological prime movers\textquotesingle ?", ask \cite{Bresnahan1995} in their article that highlighted that certain technologies may foster generalized productivity gains in an economy. Calling these technologies General Purpose Technologies (GPTs), they suggest that these technologies have the quality of being able to be applied to different products or industries are low adaptation costs. The benefit of such innovations therefore propagates across markets and industries. Later literature in the economics tradition has suggested that GPTs are more likely to be developed by diversified firms, or when downstream markets are composed of many specialized users rather than a few large customers.  In this paper, I review the literature on GPTs in economics and strategy, identify interesting questions for firm decisions regarding GPTs and then propose a study to address that question. In the section that follows, I review prior art on General Purpose Technologies as they apply to firms.

\section{Firm level implications of GPTs}
Much of the prior literature on GPTs in the economics tradition has focussed on the industrial organization antecedents and consequences of GPTs. I highlight a few of the terms and insights from this literature in the following sections.

\subsection{Dedicated technology vs. General purpose technology}
GPTs are contrasted with dedicated technologies. While a general-purpose technology can be used for different downstream applications. In contrast, a dedicated technology necessitates high adaptation costs when it is applied in distant contexts or for vastly different purposes than for which it was produced. The primary characteristic distinguishing them has been adaptation costs, and there seem to have been little work linking the capabilities literature in strategy to the topic of GPTs.

\subsection{Complementarity of R\&D investments}
In this section, I discuss two types of externalities that arise out of the complementarity of R\&D investments in an economy.
\subsubsection{Vertical externalities}
The industry structure of GPTs are characterized by vertical externalities, in that the R\&D investments of downstream and upstream sectors are complementary with that of the focal industry. When a technology or market shock improves the efficiency of an
application sector, the higher efficiency induces a greater investment in invention in downstream sectors. This in turn raises the investment in invention of the upstream sector because of the complementarity. Vertical externalities are however not unique to GPTs, as they may also occur with dedicated technologies too.

\subsubsection{Horizontal externalities}
A shock to a focal industry induces investments for improving the quality of GPTs by upstream industries. Additionally, the shock also induces improvements and investments in downstream R\&D, particularly in other industries. This is a unique trait of GPTs that a shock in one application industry may easily propagate to other industry sectors. Horizontal externalities are key in GPTs playing the role of \textquotesingle technological prime movers\textquotesingle in an economy.

\subsection{Stylized facts from economics literature}
In this section, I highlight some of the stylized facts developed about industrial organization related to generation and propagation of GPTs.
\subsubsection{Fragmented downstream markets}
GPTs are more likely to be licensed, when downstream markets are fragmented. This enables upstream technology firms companies to increase their rents by expanding their licensee base rather than relying on bargaining. The implication then is that GPTs are likely to be developed by diversified firms. They are advantaged by internalizing positive externalities, and can extract rents from distributed upstream and downstream firms. 
\subsubsection{Large downstream firms} A policy implication of this is that that large downstream firms discourage the production
of GPTs \citep{Bresnahan1998} Bresnahan and Gambardella (1998).
\subsubsection{Industries with many firms} Industries characterized by many different firms or submarkets encourage the production of GPTs. This is because their scale is not large enough to develop a dedicated technology, and they are willing to buy GPTs even if the GPTs are more standard technologies, not perfectly suited to their needs, which then require adaptation costs. One would therefore expect to observe GPTs rather than dedicated technologies small market niches.
\subsubsection{When do firms licence GPTs} \cite{Gambardella2013}  Gambardella and Giarratana (2013) argue that
firms are more likely to license GPTs when product markets are fragmented, that is, when they are characterized by different submarkets. When a technology is dedicated to a specific application, the seller can only sell it to buyers who want to use it for that application. However, if the seller also operates in that product market, they may not be willing to sell the technology to nurture a competitor \cite{Arora2003}. 

In summary, the GPT literature in the economics tradition has focussed more on industrial organization implications of GPTs and has not particularly addressed how heterogenous firms within an industry or sector may determine to generate rents from GPTs.

\subsection{GPTs in strategy literature}
Generalpurpose
technologies meet four criteria: (1) a
wide scope for improvement and elaboration;
(2) applicability across a broad range of uses;
(3) potential for use in a wide range of products
and processes; and (4) strong complementarities
with existing or potential new technologies
(Lipsey et al. 1998).
A related concept is a platform (Gawer and
Cusumano, 2002; Teece, 2012). Whereas an
enabling technology is directly employed by
firms in various industries, a ?platform innovation
is a set of rules and infrastructure that allows a
network of firms to compete in one or more industries,
with resulting benefits to the platform owner
(s). They are especially prevalent in the technology
sector, such as the Windows operating system,
eBay?s auction site and Amazon?s ?cloud?
infrastructure.
If, instead of being licensed, they are embodied
in a product, enabling technologies can bring
value to multiple activities without significant
adaptation, as in the case of microcontrollers,
electronic ?chips? that can be found in a vast
array of products, from interactive toys to ballistic
missiles. Digital rights technology supported by
the Stefik patents covering technology originally
developed


\subsection{Opportunities for strategy research}
Mostly work in economics. Very little work in strategy - some of it in Enabling Technology and Platforms \cite{Teece2012a}
If
the technology is dedicated, it can only be sold to
an individual buyer. Typically, such buyers are
large manufacturing firms, with considerable
bargaining power, while the technology suppliers
are small technology-specialist companies. Most
often, this means that the technology suppliers can
only hope to enjoy a limited share of the gains
from trade. In contrast, if the technology is GPT,
even when the technology specialist enjoys limited
rental fees per transaction, they can enter in
many transactions by selling the technology to
distinct user firms. As a result, they can switch
the source of rents from something that they can
hardly control (bargaining power) to something
that they can control ? that is, their ability to develop a technology that can be applied to a
larger number of applications.

\cite{Gambardella2010} \par
 \par
\cite{Nelson1959} \par
 \par
\cite{Boudreau2010} \par
\cite{Hosasain2011} \par
\cite{Teece2012a} \par
\cite{Rosenberg2004} \par

A few studies have discussed interesting examples about the development of GPT and its implications for company strategy (e.g., \cite{Maine2006, Thoma2009} ). 

I am going to pitch this as an Appropriability Capabilities vs. Appropriability Environment framework that determines if the appropriate strategy is to explore, exploit, explore and exploit or just trade (or stay away).

Bring in the following:
Absorptive Capacity
Dominant Design
Network Effect
Search (Broad search vs Focussed search)
Complementary Assets
\par



\section{Research Proposal}
The tension that I wish to address in this study is the one between balancing exploration-exploitation and the one between general Human Capital (HC) and Firm Specific Human Capital (FSHC). Conflict - levels of analysis. OL literature assumes all human capital to be homogenous. If one were to accept that HC were heterogenous, and that Organizational Learning is driven by routines, then ...

Plot the 2x2 and suggest that there are costs and sources of profits in each of the quadrants. Discuss it, and suggest that the net effect is unclear. Talk about non-competes, reputations for enforcement etc - if they are in favor of employees then they are more likely to extract rents. Alternatively.

Product market - causal ambiguity of some sort.

\subsection{Research Question}


\section{Theory}
\subsection{On the topic of the general hypotheses}
 Figure ~\ref{fig:3a} lays out the average score charts for four agent-field combinations while enforcing the field to start in Right of Center (this is the same as saying $p_{0,F}^0 = 0.75$). 
\subsubsection{Leading into H1a}
We do so since the scale is symmetric across the Center (C), any initial mapping 

\begin{hypothesis}
{Hypothesis 1a: When the institutional field is open to influence, slow learning adversarial agents will raise overall performance higher than slow learning agents with a neutral orientation\\}
\end{hypothesis}

Reference to Table ~\ref{qe8a}

\begin{table}[h]
\renewcommand\arraystretch{2.5} % provide a bit taller rows
\centering
\caption{Firm choices in adopting to general purpose technologies (GPTs)}
\label{qe8a}
\begin{tabularx}{\textwidth}{@{} c c |c |c| } % use 'Y' for first column
&\multicolumn{1}{c}{}&\multicolumn{2}{c}{\textbf{Appropriability Environment}}\\[-2ex]
&\multicolumn{1}{c}{}
&\multicolumn{1}{c}{\textbf{Restrictive}}&\multicolumn{1}{c}{\textbf{Munificent}}\\\cline{3-4}
\multirow{4}{*}{\rotatebox{90}{\textbf{Appropriability Capabilities}}}
&\multirow{2}{*}{\textbf{Weak}}&\textbf{Forced Exploration}&\textbf{Trade}\\
&&\textbf{Rent is uncertain}&\textbf{Entrepreneurial Rent}\\\cline{3-4}
&\multirow{2}{*}{\textbf{Strong}}&\textbf{Exploit}&\textbf{Explore and Exploit}\\
&&\textbf{Ricardian Rent}&\textbf{Schumpeterian and Ricardian Rent}\\\cline{3-4}
\end{tabularx}
\end{table}

\subsubsection{Leading into H2a}
This trend is confirmed further in Figure ~\ref{fig:3a} where the learning rates of agents are increased even further to \textquotesingle Fast\textquotesingle .

\begin{hypothesis}
{Hypothesis 2a: For the same initial outcome preferences,  the overall performance score varies curvilinearly with difference in the rates of learning of the agent and the institutional field\\}
\end{hypothesis}

\subsection{Data and Method}

What is the sample. Why is it chosen. How is it going to contribute. S
\subsubsection{Dependent Variable}


\subsubsection{Degree of Exploration} \par
0 - Exploitation, 1 - Exploration, 2 - Exploration and Exploitation, 3 - Trade \par
\subsection{Independent Variables}
\subsubsection{Appropriability Capability}
\subsubsection{Independent Variables}

\subsubsection{Moderating Variables}

\section{Contributions}
What do you contribute to each literature. Basically nuance. 

\section{Limitations}
Suggest how this study may help inform the literatures that it is drawing from, and the interesting research avenues it will open up. Discuss level of generalizability.

\section{Summary}
Recap and motivate interest in framework, in theoretical value as well as in the particular empirical setting.

\subsubsection{Appropriability Environment}
\renewcommand{\refname}{REFERENCES}
\bibliography{/Users/aiyenggar/code/bibliography/aiyenggar} 
\bibliographystyle{ai-amjlike}


\end{document}
