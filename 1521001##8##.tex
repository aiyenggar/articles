%1521001##1##.tex
\documentclass[12pt,letterpaper]{article}
\usepackage{mathptmx}
\usepackage[margin=1in]{geometry}

\usepackage{tabularx, longtable}  % for 'tabularx' environment and 'X' column type
\usepackage{ragged2e}  % for '\RaggedRight' macro (allows hyphenation)
\newcolumntype{Y}{>{\RaggedRight\arraybackslash}X} 
\usepackage{booktabs}  % for \toprule, \midrule, and \bottomrule macros 

\usepackage{setspace}
\singlespacing
  
\usepackage{amssymb,latexsym}
\usepackage[round,sort]{natbib}
\usepackage{fancyhdr}
\usepackage{lastpage}
\usepackage{graphicx,multirow}
\graphicspath{ {qe8/} }

% Bold Table and Figure captions
\usepackage{caption}
\captionsetup{figurename=FIGURE}
\captionsetup{tablename=TABLE}
\captionsetup[figure]{labelfont=bf}
\captionsetup[table]{labelfont=bf}
  
% Turns off all section numbering
\setcounter{secnumdepth}{0} 

  % Places all tables at end of document and creates AOM-style table-here placeholders
  \usepackage[nolists]{endfloat} % Places all figures and charts at end of manuscript and adds 'insert table x about here' lines.
  \renewcommand{\figureplace}{
    \begin{center}
    \begin{singlespace}
    ------------------------------------\\
    Insert \figurename \ \thepostfig\ about here.\\
    ------------------------------------
    \end{singlespace}
    \end{center}}
  \renewcommand{\tableplace}{%
    \begin{center}
    \begin{singlespace}
    ------------------------------------\\
    Insert \tablename \ \theposttbl\ about here.\\
    ------------------------------------
    \end{singlespace}
    \end{center}}

  \usepackage{titlesec}
   \titleformat{\title}
       {\filcenter\normalfont\bfseries\uppercase}{\thetitle}{1em}{}
  \titleformat{\section}
    {\filcenter\normalfont\bfseries\uppercase}{\thesection}{1em}{}
  \titleformat{\subsection}
    {\normalfont\bfseries}{\thesubsection}{1em}{}
  \titleformat{\subsubsection}[runin]
   {\normalfont\bfseries\slshape}{\thesubsubsection}{1em}{\hspace*{\parindent}}
       
\usepackage{tabu}
\usepackage{textcomp}
\usepackage{listings}
\usepackage{hyperref}
\usepackage{verbatim}
\usepackage{tabu}
\hypersetup{
    colorlinks=true,
    linkcolor=blue,
    filecolor=cyan,      
    urlcolor=cyan,
    citecolor=blue,
}

\usepackage{etoolbox}

\makeatletter

% Patch case where name and year are separated by aysep
\patchcmd{\NAT@citex}
  {\@citea\NAT@hyper@{%
     \NAT@nmfmt{\NAT@nm}%
     \hyper@natlinkbreak{\NAT@aysep\NAT@spacechar}{\@citeb\@extra@b@citeb}%
     \NAT@date}}
  {\@citea\NAT@nmfmt{\NAT@nm}%
   \NAT@aysep\NAT@spacechar\NAT@hyper@{\NAT@date}}{}{}

% Patch case where name and year are separated by opening bracket
\patchcmd{\NAT@citex}
  {\@citea\NAT@hyper@{%
     \NAT@nmfmt{\NAT@nm}%
     \hyper@natlinkbreak{\NAT@spacechar\NAT@@open\if*#1*\else#1\NAT@spacechar\fi}%
       {\@citeb\@extra@b@citeb}%
     \NAT@date}}
  {\@citea\NAT@nmfmt{\NAT@nm}%
   \NAT@spacechar\NAT@@open\if*#1*\else#1\NAT@spacechar\fi\NAT@hyper@{\NAT@date}}
  {}{}

\lstset{
basicstyle=\ttfamily,
columns=flexible,
breaklines=true
}
\newenvironment{hypothesis}{
  	\itshape
  	\leftskip=\parindent \rightskip=\parindent
  	\noindent\ignorespaces}

\fancypagestyle{plain}{
  \renewcommand{\headrulewidth}{0pt}
  \fancyhf{}
}	


\begin{document}
\title{General Purpose Technologies and the Exploration-Exploitation Tradeoff}
\date{}
\maketitle

\begin{abstract} 
\normalsize 
We apply a formal model to understand the effects of the relative learning rates of embedded agents and the institutional field on organizational outcomes. 
\end{abstract}


{\textbf{Keywords:} \\\indent Embedded Agency}

\newpage
\pagestyle{fancy}
\fancyhf{}
\lhead{GPTs and the Exploration-Exploitation Tradeoff}
\rhead{\thepage}

\begin{center}
\textbf{GPTs and the Exploration-Exploitation Tradeoff}
\end{center}



\cite{Arora2003} \par
\cite{Bresnahan1995} \par
\cite{Gambardella2010} \par
\cite{Maine2006} \par
\cite{Nelson1959} \par
\cite{Thoma2009} \par
\cite{Boudreau2010} \par
\cite{Hosasain2011} \par
\cite{Teece2012a} \par
\cite{Rosenberg2004} \par

Mostly work in economics. Very little work in strategy - some of it in Enabling Technology and Platforms \cite{Teece2012a}

Dedicated technology vs. General purpose technology

Low adaptation costs

complementarity of R\&D investments

Vertical externalities

Horizontal externalities - unique to GPTs not in dedicated technologies

I am going to pitch this as an Appropriability Capabilities vs. Appropriability Environment framework that determines if the appropriate strategy is to explore, exploit, explore and exploit or just trade (or stay away).

Bring in the following:
Absorptive Capacity
Dominant Design
Network Effect
Search (Broad search vs Focussed search)
Complementary Assets
\par



\section{Research Proposal}
The tension that I wish to address in this study is the one between balancing exploration-exploitation and the one between general Human Capital (HC) and Firm Specific Human Capital (FSHC). Conflict - levels of analysis. OL literature assumes all human capital to be homogenous. If one were to accept that HC were heterogenous, and that Organizational Learning is driven by routines, then ...

Plot the 2x2 and suggest that there are costs and sources of profits in each of the quadrants. Discuss it, and suggest that the net effect is unclear. Talk about non-competes, reputations for enforcement etc - if they are in favor of employees then they are more likely to extract rents. Alternatively.

Product market - causal ambiguity of some sort.

\subsection{Research Question}


\section{Theory}
\subsection{On the topic of the general hypotheses}
 Figure ~\ref{fig:3a} lays out the average score charts for four agent-field combinations while enforcing the field to start in Right of Center (this is the same as saying $p_{0,F}^0 = 0.75$). 
\subsubsection{Leading into H1a}
We do so since the scale is symmetric across the Center (C), any initial mapping 

\begin{hypothesis}
{Hypothesis 1a: When the institutional field is open to influence, slow learning adversarial agents will raise overall performance higher than slow learning agents with a neutral orientation\\}
\end{hypothesis}

Reference to Table ~\ref{qe8a}

\begin{table}[h]
\renewcommand\arraystretch{2.5} % provide a bit taller rows
\centering
\caption{Firm choices in adopting to general purpose technologies (GPTs)}
\label{qe8a}
\begin{tabularx}{\textwidth}{@{} c c |c |c| } % use 'Y' for first column
&\multicolumn{1}{c}{}&\multicolumn{2}{c}{\textbf{Appropriability Environment}}\\[-2ex]
&\multicolumn{1}{c}{}
&\multicolumn{1}{c}{\textbf{Restrictive}}&\multicolumn{1}{c}{\textbf{Munificent}}\\\cline{3-4}
\multirow{4}{*}{\rotatebox{90}{\textbf{Appropriability Capabilities}}}
&\multirow{2}{*}{\textbf{Weak}}&\textbf{Forced Exploration}&\textbf{Trade}\\
&&\textbf{Rent is uncertain}&\textbf{Entrepreneurial Rent}\\\cline{3-4}
&\multirow{2}{*}{\textbf{Strong}}&\textbf{Exploit}&\textbf{Explore and Exploit}\\
&&\textbf{Ricardian Rent}&\textbf{Schumpeterian and Ricardian Rent}\\\cline{3-4}
\end{tabularx}
\end{table}

\subsubsection{Leading into H2a}
This trend is confirmed further in Figure ~\ref{fig:3a} where the learning rates of agents are increased even further to \textquotesingle Fast\textquotesingle .

\begin{hypothesis}
{Hypothesis 2a: For the same initial outcome preferences,  the overall performance score varies curvilinearly with difference in the rates of learning of the agent and the institutional field\\}
\end{hypothesis}

\subsection{Data and Method}

What is the sample. Why is it chosen. How is it going to contribute. S
\subsubsection{Dependent Variable}


\subsubsection{Degree of Exploration} \par
0 - Exploitation, 1 - Exploration, 2 - Exploration and Exploitation, 3 - Trade \par
\subsection{Independent Variables}
\subsubsection{Appropriability Capability}
\subsubsection{Independent Variables}

\subsubsection{Moderating Variables}

\section{Contributions}
What do you contribute to each literature. Basically nuance. 

\section{Limitations}
Suggest how this study may help inform the literatures that it is drawing from, and the interesting research avenues it will open up. Discuss level of generalizability.

\section{Summary}
Recap and motivate interest in framework, in theoretical value as well as in the particular empirical setting.

\subsubsection{Appropriability Environment}
\renewcommand{\refname}{REFERENCES}
\bibliography{/Users/aiyenggar/code/bibliography/aiyenggar} 
\bibliographystyle{ai-amjlike}


\end{document}
