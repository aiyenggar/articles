%1521001##1##.tex
\documentclass[12pt,letterpaper]{article}
\usepackage{mathptmx}
\usepackage[margin=1in]{geometry}

\usepackage{tabularx, longtable}  % for 'tabularx' environment and 'X' column type
\usepackage{ragged2e}  % for '\RaggedRight' macro (allows hyphenation)
\newcolumntype{Y}{>{\RaggedRight\arraybackslash}X} 
\usepackage{booktabs}  % for \toprule, \midrule, and \bottomrule macros 

\usepackage{setspace}
\singlespacing
  
\usepackage{amssymb,latexsym}
\usepackage[round,sort]{natbib}
\usepackage{fancyhdr}
\usepackage{lastpage}
\usepackage{graphicx,multirow}
\graphicspath{ {qe8/} }

% Bold Table and Figure captions
\usepackage{caption}
\captionsetup{figurename=FIGURE}
\captionsetup{tablename=TABLE}
\captionsetup[figure]{labelfont=bf}
\captionsetup[table]{labelfont=bf}
  
% Turns off all section numbering
\setcounter{secnumdepth}{0} 

  % Places all tables at end of document and creates AOM-style table-here placeholders
  \usepackage[nolists]{endfloat} % Places all figures and charts at end of manuscript and adds 'insert table x about here' lines.
  \renewcommand{\figureplace}{
    \begin{center}
    \begin{singlespace}
    ------------------------------------\\
    Insert \figurename \ \thepostfig\ about here.\\
    ------------------------------------
    \end{singlespace}
    \end{center}}
  \renewcommand{\tableplace}{%
    \begin{center}
    \begin{singlespace}
    ------------------------------------\\
    Insert \tablename \ \theposttbl\ about here.\\
    ------------------------------------
    \end{singlespace}
    \end{center}}

  \usepackage{titlesec}
   \titleformat{\title}
       {\filcenter\normalfont\bfseries\uppercase}{\thetitle}{1em}{}
  \titleformat{\section}
    {\filcenter\normalfont\bfseries\uppercase}{\thesection}{1em}{}
  \titleformat{\subsection}
    {\normalfont\bfseries}{\thesubsection}{1em}{}
  \titleformat{\subsubsection}[runin]
   {\normalfont\bfseries\slshape}{\thesubsubsection}{1em}{\hspace*{\parindent}}
       
\usepackage{tabu}
\usepackage{textcomp}
\usepackage{listings}
\usepackage{hyperref}
\usepackage{verbatim}
\usepackage{tabu}
\hypersetup{
    colorlinks=true,
    linkcolor=blue,
    filecolor=cyan,      
    urlcolor=cyan,
    citecolor=blue,
}

\usepackage{etoolbox}

\makeatletter

% Patch case where name and year are separated by aysep
\patchcmd{\NAT@citex}
  {\@citea\NAT@hyper@{%
     \NAT@nmfmt{\NAT@nm}%
     \hyper@natlinkbreak{\NAT@aysep\NAT@spacechar}{\@citeb\@extra@b@citeb}%
     \NAT@date}}
  {\@citea\NAT@nmfmt{\NAT@nm}%
   \NAT@aysep\NAT@spacechar\NAT@hyper@{\NAT@date}}{}{}

% Patch case where name and year are separated by opening bracket
\patchcmd{\NAT@citex}
  {\@citea\NAT@hyper@{%
     \NAT@nmfmt{\NAT@nm}%
     \hyper@natlinkbreak{\NAT@spacechar\NAT@@open\if*#1*\else#1\NAT@spacechar\fi}%
       {\@citeb\@extra@b@citeb}%
     \NAT@date}}
  {\@citea\NAT@nmfmt{\NAT@nm}%
   \NAT@spacechar\NAT@@open\if*#1*\else#1\NAT@spacechar\fi\NAT@hyper@{\NAT@date}}
  {}{}

\lstset{
basicstyle=\ttfamily,
columns=flexible,
breaklines=true
}
\newenvironment{hypothesis}{
  	\itshape
  	\leftskip=\parindent \rightskip=\parindent
  	\noindent\ignorespaces}

\fancypagestyle{plain}{
  \renewcommand{\headrulewidth}{0pt}
  \fancyhf{}
}	


\begin{document}
\title{General Purpose Technologies and the Exploration-Exploitation Tradeoff}
\date{}
\maketitle

\begin{abstract} 
\normalsize 
In this paper, I propose a novel study to understand firm level strategy implications from the emergence of General Purpose Technologies (GPTs). Departing from the insight from the Industrial Organization literature on GPTs that technology firms without bargaining power may be able to shift to higher volume economies arising from GPTs\textquotesingle horizontal externalities, I build on the capabilities literature in strategy to nuance the argument as an exploration-exploitation tradeoff between the appropriation capabilities of the firm and the appropriability environment of the industry ecosystem.
\end{abstract}


{\textbf{Keywords:} \\\indent Capabilities, Complementarity, Firm level strategy, General Purpose Technology}

\newpage
\pagestyle{fancy}
\fancyhf{}
\lhead{GPTs and the Exploration-Exploitation Tradeoff}
\rhead{\thepage}

\begin{center}
\textbf{GPTs and the Exploration-Exploitation Tradeoff}
\end{center}
``Are there such things as \textquotesingle technological prime movers\textquotesingle ?", ask \cite{Bresnahan1995} in their article that highlighted that certain technologies may foster generalized productivity gains in an economy. Calling these technologies General Purpose Technologies (GPTs), they suggest that these technologies have the quality of being able to be applied to different products or industries are low adaptation costs \citep{Rosenberg2004}. The benefit of such innovations therefore propagates across markets and industries. Later literature in the economics tradition has suggested that GPTs are more likely to be developed by diversified firms, or when downstream markets are composed of many specialized users rather than a few large customers.  In this paper, I review the literature on GPTs in economics and strategy, identify interesting questions for firm decisions regarding GPTs and then propose a study to address that question. In the section that follows, I review prior art on General Purpose Technologies as they apply to firms.

\section{Firm level implications of General Purpose Technology}
Much of the prior literature on GPTs in the economics tradition has focussed on the industrial organization antecedents and consequences of GPTs. I highlight a few of the terms and insights from this literature in the following sections.

\subsection{Dedicated technology vs. General purpose technology}
GPTs are contrasted with dedicated technologies. While a general-purpose technology can be used for different downstream applications. In contrast, a dedicated technology necessitates high adaptation costs when it is applied in distant contexts or for vastly different purposes than for which it was produced \citep{Rosenberg2004}. The primary characteristic distinguishing them has been adaptation costs, and there seem to have been little work linking the capabilities literature in strategy to the topic of GPTs.

\subsection{Complementarity of R\&D investments}
In this section, I discuss two types of externalities that arise out of the complementarity of R\&D investments in an economy.
\subsubsection{Vertical externalities}
The industry structure of GPTs are characterized by vertical externalities, in that the R\&D investments of downstream and upstream sectors are complementary with that of the focal industry. When a technology or market shock improves the efficiency of an
application sector, the higher efficiency induces a greater investment in invention in downstream sectors. This in turn raises the investment in invention of the upstream sector because of the complementarity. Vertical externalities are however not unique to GPTs, as they may also occur with dedicated technologies too.

\subsubsection{Horizontal externalities}
A shock to a focal industry induces investments for improving the quality of GPTs by upstream industries. Additionally, the shock also induces improvements and investments in downstream R\&D, particularly in other industries. This is a unique trait of GPTs that a shock in one application industry may easily propagate to other industry sectors. Horizontal externalities are key in GPTs playing the role of \textquotesingle technological prime movers\textquotesingle in an economy.

\subsection{Stylized facts from economics literature}
In this section, I highlight some of the stylized facts developed about industrial organization related to generation and propagation of GPTs.
\subsubsection{Fragmented downstream markets}
GPTs are more likely to be licensed, when downstream markets are fragmented. This enables upstream technology firms companies to increase their rents by expanding their licensee base rather than relying on bargaining. The implication then is that GPTs are likely to be developed by diversified firms. They are advantaged by internalizing positive externalities, and can extract rents from distributed upstream and downstream firms. 
\subsubsection{Large downstream firms} A policy implication of this is that that large downstream firms discourage the production
of GPTs \citep{Bresnahan1998}.
\subsubsection{Industries with many firms} Industries characterized by many different firms or submarkets encourage the production of GPTs. This is because their scale is not large enough to develop a dedicated technology, and they are willing to buy GPTs even if the GPTs are more standard technologies, not perfectly suited to their needs, which then require adaptation costs. One would therefore expect to observe GPTs rather than dedicated technologies small market niches.
\subsubsection{When do firms licence GPTs?} \cite{Gambardella2013}  argue that firms are more likely to license GPTs when product markets are fragmented, that is, when they are characterized by different submarkets. When a technology is dedicated to a specific application, the seller can only sell it to buyers who want to use it for that application. However, if the seller also operates in that product market, they may not be willing to sell the technology to nurture a competitor \cite{Arora2003}. 

\subsection{Summary of GPTs in economics literature}
The GPT literature in the economics tradition has focussed more on industrial organization implications of GPTs and has not particularly addressed how heterogenous firms within an industry or sector may determine to generate rents from GPTs.

\subsection{GPTs in strategy literature}
While research on GPTs has been slow to take off among strategy scholars, prominent scholar David Teece has suggested that a related concept to the GPT that has been addressed in the strategy literature is that of a platform \citep{Gawer2002, Teece2012a}. Unlike a
GPT that is directly employed by firms in various industries, a platform innovation is a set of rules and infrastructure that allows a
network of firms to compete in one or more industries, with resulting benefits to the platform owner. Examples include the Windows operating system, eBay?s auction site and Amazon\textquotesingle s AWS cloud infrastructure. Teece also suggests that when such technologies are embodied in a product rather than licensed, it may enable effects similar to GPTs since multiple industries may then be able to use it with minimal adaptation. One stream of work has explored factors that affect the propensity for markets to \textquotesingle tip\textquotesingle to a single platform \citep{Katz1994}. For instance, markets are more likely to tip when there are supply-side scale economies and homogeneous demand-side preferences. \cite{Hosasain2011} show that platforms are more likely to tip when there are significant differences in vertical differentiation (or quality) but are much less likely to tip when they exhibit significant degrees of horizontal differentiation. The implication is that systematic differences in the structure of an ecosystem affect the likelihood that a single or multiple platforms will exist. Despite sharing similarities with GPTs, platforms are different in that they (or a key part of them) are typically both closely held and are not quite as wide ranging in impact as GPTs. A few studies have discussed interesting examples about the development of GPT and its implications for company strategy (e.g., \cite{Maine2006, Thoma2009} ). \cite{Boudreau2010} found that granting greater levels of access to independent hardware developers in the handheld computing systems sector produces up to a five-fold acceleration in the rate of new product development. In the following section, I explore opportunities to apply GPTs to firm level questions


\subsection{Opportunities for strategy research}
Prior sections have clarified that most academic literature on GPTs have been in economics and industrial organization traditions. Very little work on this has happened in strategy literature with the little that has being in Enabling Technology and Platforms \cite{Teece2012a}. The opportunities to explore the question of GPTs for firm level decisions is therefore both vast and overwhelming. Nevertheless, I will attempt here to raise a question of interest to strategy scholars.

My point of departure from the economics literature is the recognition that one small  specialist technology firm among a large number of other technology firms can potentially switch the source of its rents from a poor bargaining position (due to poor market concentration) to one from high volume (a factor that the small firm can much better control, \cite{Gambardella2010}). While the option of competing out of a poor bargaining position is valuable, the economics literature itself unable to explain why any one particular firm may be able to do so, and if so what characteristics allow such a transition. This is the question I address in my proposal in the following section.


\section{Research Proposal}
The resouce and dynamic capabilities perspective in strategy literature maintains that firms possessing, creating, and adapting resources and capabilities can capture and sustain competitive advantage \citep{Barney1991, Penrose1959, Teece1997}. Specifically, the capabilities literature suggests that capabilities ``enable business
enterprises to create, deploy, and protect the intangible assets that support superior long-run business performance" \citep{Teece2007}. Firms should therefore be able to leverage their capabilities in determining exploration/exploitation action. The governance approach, on the other hand maintains that higher economic performance can be achieved by investing in complementary and cospecialized assets \citep{Helfat1997, Teece1986} and by governing them in an economizing way \citep{Oxley1997, Williamson1985}. Firms dealing with GPTs therefore have two potential ways in which to negotiate the environment so as to maximize firm profits.  The research question I ask is, how do individual firms determine their choices to explore or exploit so as to benefit the most from appropriating value from a GPT.


\subsection{Theory}
The decision problem may be framed as a choice depending on firm specific capabilities and environment specific attributes. I envision this decision along a 2x2 matrix that is presented in Table ~\ref{qe8a}. Drawing from the strategy literature on capabilities, firms\textquotesingle decision on how to appropriate value from a GPT will depend on their internal capability. This capability, called Appropriation Capabilities may be explained using absorptive capacity \cite{Cohen1990}, presence of complementary assets \citep{Teece1986}, by ability to perform technological search \cite{Rosenkopf2001} or from benefiting from a valuable network position \cite{Ahuja2000}. Firms with the capability to understand recognize the potential from a GPT may be classified as having strong appropriability capabilities while those without such capabilities may be seen as having weak appropriability capabilities. The two classifications are intended for ease of understanding, but in reality are expect to lay somewhere along a continuum between no capability and strong capability. \par

On the other dimension, I capture the forces of the environment as being either restrictive or munificent. This dimension may involve country level anti-trust laws, intellectual property appropriation laws, technology industry characteristics (such as dominant designs), or other factors such as a benevolent capital cycle or demand cycle. While the appropriability capabilities applies idiosyncratically to each firm, appropriability environment applies equally to all firms in an industry or geography.

\begin{table}[h]
\renewcommand\arraystretch{2.5} % provide a bit taller rows
\centering
\caption{Firm choices in adopting to general purpose technologies (GPTs)}
\label{qe8a}
\begin{tabularx}{\textwidth}{@{} c c |c |c| } % use 'Y' for first column
&\multicolumn{1}{c}{}&\multicolumn{2}{c}{\textbf{Appropriability Environment}}\\[-2ex]
&\multicolumn{1}{c}{}
&\multicolumn{1}{c}{\textbf{Restrictive}}&\multicolumn{1}{c}{\textbf{Munificent}}\\\cline{3-4}
\multirow{4}{*}{\rotatebox{90}{\textbf{Appropriation Capabilities}}}
&\multirow{2}{*}{\textbf{Weak}}&\textbf{Forced Exploration or Exit}&\textbf{Trade}\\
&&\textbf{Rent is uncertain}&\textbf{Entrepreneurial Rent}\\\cline{3-4}
&\multirow{2}{*}{\textbf{Strong}}&\textbf{Exploit}&\textbf{Explore and Exploit}\\
&&\textbf{Ricardian Rent}&\textbf{Schumpeterian and Ricardian Rent}\\\cline{3-4}
\end{tabularx}
\end{table}

\subsubsection{Exploitation choice}
Capabilities theory would suggest that firms should focus more on exploitation when appropriation capabilities are strong. As denoted in Table ~\ref{qe8a}, strong appropriation capabilities allow the firm to capture Ricardian rents out of ownership of superior resources and capabilities. When the environment in restrictive, in that either the demand or property rights environment is weak, governance theory would also suggest exploitation. Therefore when appropriation capabilities are strong and the appropriability environment is weak, firms would maximize rents from an exploitation approach.
\begin{hypothesis}
{Hypothesis 1: Firms with strong appropriation capabilities but faced with restrictive appropriability environments will maximize rents generated through exploitative behvior\\}
\end{hypothesis}

\subsubsection{Ambidexterity in choice}
As discussed above strong appropriation capabilities allow firms to collect Ricardian rents even in restrictive appropriability environments. In munificent appropriability environments however, firms be able to collect Schumpeterian rents temporarily due to the superior environmental conditions. Organizations literature has suggested that certain environments are well suited to ambidexterous organziations, those that can simultaneously engage in explorative and exploitative behavior \cite{OReilly2004, OReilly2008}. This leads me to my second hypothesis.

\begin{hypothesis}
{Hypothesis 2: Firms with strong appropriation capabilities in munificent appropriability environments will maximize rents from GPTs through simultaneous explorative and exploitative behvior\\}
\end{hypothesis}

\subsubsection{Fleeting rents}
As discussed above strong appropriation capabilities allow firms to collect Ricardian rents even in restrictive appropriability environments. In munificent appropriability environments however, firms be able to collect Schumpeterian rents temporarily due to the superior environmental conditions. Organizations literature has suggested that certain environments are well suited to ambidexterous organziations, those that can simultaneously engage in explorative and exploitative behavior \cite{OReilly2004, OReilly2008}. This leads me to my second hypothesis.

\begin{hypothesis}
{Hypothesis 2: Firms with strong appropriation capabilities in munificent appropriability environments will maximize rents from GPTs by simultaneous explorative and exploitative behvior\\}
\end{hypothesis}

\subsection{Data and Method}

\subsubsection{Dependent Variable}


\subsubsection{Degree of Exploration} \par
0 - Exploitation, 1 - Exploration, 2 - Exploration and Exploitation, 3 - Trade \par
\subsection{Independent Variables}

\subsubsection{Appropriation Capability}
\subsubsection{Appropriability Environment}

\subsubsection{Independent Variables}

\subsubsection{Moderating Variables}

\section{Contributions}
What do you contribute to each literature. Basically nuance. 

\section{Limitations}
Suggest how this study may help inform the literatures that it is drawing from, and the interesting research avenues it will open up. Discuss level of generalizability.

\section{Summary}
Recap and motivate interest in framework, in theoretical value as well as in the particular empirical setting.

\renewcommand{\refname}{REFERENCES}
\bibliography{/Users/aiyenggar/code/bibliography/aiyenggar} 
\bibliographystyle{ai-amjlike}


\end{document}
