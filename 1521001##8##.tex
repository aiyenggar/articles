%1521001##1##.tex
\documentclass[12pt,letterpaper]{article}
\usepackage{mathptmx}
\usepackage[margin=1in]{geometry}

\usepackage{setspace}
\singlespacing
  
\usepackage{amssymb,latexsym}
\usepackage[round,sort]{natbib}
\usepackage{fancyhdr}
\usepackage{lastpage}
\usepackage{graphicx,multirow}
\graphicspath{ {qe8/} }

% Bold Table and Figure captions
\usepackage{caption}
\captionsetup{figurename=FIGURE}
\captionsetup{tablename=TABLE}
\captionsetup[figure]{labelfont=bf}
\captionsetup[table]{labelfont=bf}
  
% Turns off all section numbering
\setcounter{secnumdepth}{0} 

  % Places all tables at end of document and creates AOM-style table-here placeholders
  \usepackage[nolists]{endfloat} % Places all figures and charts at end of manuscript and adds 'insert table x about here' lines.
  \renewcommand{\figureplace}{
    \begin{center}
    \begin{singlespace}
    ------------------------------------\\
    Insert \figurename \ \thepostfig\ about here.\\
    ------------------------------------
    \end{singlespace}
    \end{center}}
  \renewcommand{\tableplace}{%
    \begin{center}
    \begin{singlespace}
    ------------------------------------\\
    Insert \tablename \ \theposttbl\ about here.\\
    ------------------------------------
    \end{singlespace}
    \end{center}}

  \usepackage{titlesec}
   \titleformat{\title}
       {\filcenter\normalfont\bfseries\uppercase}{\thetitle}{1em}{}
  \titleformat{\section}
    {\filcenter\normalfont\bfseries\uppercase}{\thesection}{1em}{}
  \titleformat{\subsection}
    {\normalfont\bfseries}{\thesubsection}{1em}{}
  \titleformat{\subsubsection}[runin]
   {\normalfont\bfseries\slshape}{\thesubsubsection}{1em}{\hspace*{\parindent}}
       
\usepackage{tabu}
\usepackage{textcomp}
\usepackage{listings}
\usepackage{hyperref}
\usepackage{verbatim}
\usepackage{tabu}
\hypersetup{
    colorlinks=true,
    linkcolor=blue,
    filecolor=cyan,      
    urlcolor=cyan,
    citecolor=blue,
}

\usepackage{etoolbox}

\makeatletter

% Patch case where name and year are separated by aysep
\patchcmd{\NAT@citex}
  {\@citea\NAT@hyper@{%
     \NAT@nmfmt{\NAT@nm}%
     \hyper@natlinkbreak{\NAT@aysep\NAT@spacechar}{\@citeb\@extra@b@citeb}%
     \NAT@date}}
  {\@citea\NAT@nmfmt{\NAT@nm}%
   \NAT@aysep\NAT@spacechar\NAT@hyper@{\NAT@date}}{}{}

% Patch case where name and year are separated by opening bracket
\patchcmd{\NAT@citex}
  {\@citea\NAT@hyper@{%
     \NAT@nmfmt{\NAT@nm}%
     \hyper@natlinkbreak{\NAT@spacechar\NAT@@open\if*#1*\else#1\NAT@spacechar\fi}%
       {\@citeb\@extra@b@citeb}%
     \NAT@date}}
  {\@citea\NAT@nmfmt{\NAT@nm}%
   \NAT@spacechar\NAT@@open\if*#1*\else#1\NAT@spacechar\fi\NAT@hyper@{\NAT@date}}
  {}{}

\lstset{
basicstyle=\ttfamily,
columns=flexible,
breaklines=true
}
\newenvironment{hypothesis}{
  	\itshape
  	\leftskip=\parindent \rightskip=\parindent
  	\noindent\ignorespaces}

\fancypagestyle{plain}{
  \renewcommand{\headrulewidth}{0pt}
  \fancyhf{}
}	


\begin{document}
\title{General Purpose Technologies and the Exploration-Exploitation Tradeoff}
\date{}
\maketitle

\begin{abstract} 
\normalsize 
We apply a formal model to understand the effects of the relative learning rates of embedded agents and the institutional field on organizational outcomes. 
\end{abstract}


{\textbf{Keywords:} \\\indent Embedded Agency}

\newpage
\pagestyle{fancy}
\fancyhf{}
\lhead{GPTs and the Exploration-Exploitation Tradeoff}
\rhead{\thepage}

\begin{center}
\textbf{GPTs and the Exploration-Exploitation Tradeoff}
\end{center}


\section{Research Proposal}
\subsection{Research Question}

\subsection{Data and Method}
What is the sample. Why is it chosen. How is it going to contribute.

\subsection{Hypotheses}

\section{Limitations}
Suggest how this study may help inform the literatures that it is drawing from, and the interesting research avenues it will open up. Discuss level of generalizability.

\section{Summary}
Recap and motivate interest in framework, in theoretical value as well as in the particular empirical setting.





\cite{Arora2003} \par
\cite{Bresnahan1995} \par
\cite{Gambardella2010} \par
\cite{Maine2006} \par
\cite{Nelson1959} \par
\cite{Thoma2009} \par
\cite{Boudreau2010} \par
\cite{Hosasain2011} \par
\cite{Teece2012a} \par
\cite{Rosenberg2004} \par

Mostly work in economics. Very little work in strategy - some of it in Enabling Technology and Platforms \cite{Teece2012a}

Dedicated technology vs. General purpose technology

Low adaptation costs

complementarity of R\&D investments

Vertical externalities

Horizontal externalities - unique to GPTs not in dedicated technologies

I am going to pitch this as an Appropriability Capabilities vs. Appropriability Environment framework that determines if the appropriate strategy is to explore, exploit, explore and exploit or just trade (or stay away).

\begin{table}[h]
\renewcommand\arraystretch{2.5} % provide a bit taller rows
\centering
\caption{The Exploration-Exploitation Tradeoff with GPTs}
\begin{tabular}{ll | c | c |}
&\multicolumn{1}{c}{}&\multicolumn{2}{c}{\textbf{Name1}}\\[-2ex]
&\multicolumn{1}{c}{}
&\multicolumn{1}{c}{Factor1}&\multicolumn{1}{c}{Factor2}\\\cline{3-4}
\multirow{2}{*}{\rotatebox{90}{\textbf{Name2}}}
&Factor1&5&4\\\cline{3-4}
&Factor2&2&2\\\cline{3-4}
\end{tabular}
\end{table}

Bring in the following:
Absorptive Capacity
Dominant Design
Network Effect
Search (Broad search vs Focussed search)
Complementary Assets
\par

\section{Research Design}
\subsection{Dependent Variable}

\subsubsection{Degree of Exploration} \par
0 - Exploitation, 1 - Exploration, 2 - Exploration and Exploitation, 3 - Trade \par
\subsection{Independent Variables}
\subsubsection{Appropriability Capability}

\subsubsection{Appropriability Environment}
\renewcommand{\refname}{REFERENCES}
\bibliography{/Users/aiyenggar/code/bibliography/aiyenggar} 
\bibliographystyle{ai-amjlike}


\end{document}
