%1521001##1##.tex
\documentclass[12pt,letterpaper]{article}
\usepackage{mathptmx}
\usepackage[margin=1in]{geometry}

\usepackage{tabularx, longtable}  % for 'tabularx' environment and 'X' column type
\usepackage{ragged2e}  % for '\RaggedRight' macro (allows hyphenation)
\newcolumntype{Y}{>{\RaggedRight\arraybackslash}X} 
\usepackage{booktabs}  % for \toprule, \midrule, and \bottomrule macros 

\usepackage{setspace}
\singlespacing
  
\usepackage{amssymb,latexsym}
\usepackage[round,sort]{natbib}
\usepackage{fancyhdr}
\usepackage{lastpage}
\usepackage{graphicx,multirow}
\graphicspath{ {qe2/} }

% Bold Table and Figure captions
\usepackage{caption}
\captionsetup{figurename=FIGURE}
\captionsetup{tablename=TABLE}
\captionsetup[figure]{labelfont=bf}
\captionsetup[table]{labelfont=bf}
  
% Turns off all section numbering
\setcounter{secnumdepth}{0} 

  % Places all tables at end of document and creates AOM-style table-here placeholders
  \usepackage[nolists]{endfloat} % Places all figures and charts at end of manuscript and adds 'insert table x about here' lines.
  \renewcommand{\figureplace}{
    \begin{center}
    \begin{singlespace}
    ------------------------------------\\
    Insert \figurename \ \thepostfig\ about here.\\
    ------------------------------------
    \end{singlespace}
    \end{center}}
  \renewcommand{\tableplace}{%
    \begin{center}
    \begin{singlespace}
    ------------------------------------\\
    Insert \tablename \ \theposttbl\ about here.\\
    ------------------------------------
    \end{singlespace}
    \end{center}}

  \usepackage{titlesec}
   \titleformat{\title}
       {\filcenter\normalfont\bfseries\uppercase}{\thetitle}{1em}{}
  \titleformat{\section}
    {\filcenter\normalfont\bfseries\uppercase}{\thesection}{1em}{}
  \titleformat{\subsection}
    {\normalfont\bfseries}{\thesubsection}{1em}{}
  \titleformat{\subsubsection}[runin]
   {\normalfont\bfseries\slshape}{\thesubsubsection}{1em}{\hspace*{\parindent}}
       
\usepackage{tabu}
\usepackage{textcomp}
\usepackage{listings}
\usepackage{hyperref}
\usepackage{verbatim}
\usepackage{tabu}
\hypersetup{
    colorlinks=true,
    linkcolor=blue,
    filecolor=cyan,      
    urlcolor=cyan,
    citecolor=black,
}

\usepackage{etoolbox}

\makeatletter

% Patch case where name and year are separated by aysep
\patchcmd{\NAT@citex}
  {\@citea\NAT@hyper@{%
     \NAT@nmfmt{\NAT@nm}%
     \hyper@natlinkbreak{\NAT@aysep\NAT@spacechar}{\@citeb\@extra@b@citeb}%
     \NAT@date}}
  {\@citea\NAT@nmfmt{\NAT@nm}%
   \NAT@aysep\NAT@spacechar\NAT@hyper@{\NAT@date}}{}{}

% Patch case where name and year are separated by opening bracket
\patchcmd{\NAT@citex}
  {\@citea\NAT@hyper@{%
     \NAT@nmfmt{\NAT@nm}%
     \hyper@natlinkbreak{\NAT@spacechar\NAT@@open\if*#1*\else#1\NAT@spacechar\fi}%
       {\@citeb\@extra@b@citeb}%
     \NAT@date}}
  {\@citea\NAT@nmfmt{\NAT@nm}%
   \NAT@spacechar\NAT@@open\if*#1*\else#1\NAT@spacechar\fi\NAT@hyper@{\NAT@date}}
  {}{}

\lstset{
basicstyle=\ttfamily,
columns=flexible,
breaklines=true
}
\newenvironment{hypothesis}{
  	\itshape
  	\leftskip=\parindent \rightskip=\parindent
  	\noindent\ignorespaces}

\fancypagestyle{plain}{
  \renewcommand{\headrulewidth}{0pt}
  \fancyhf{}
}	


\begin{document}
\title{Organizational Learning and Human Capital: Similarities, Tensions and a Research Proposal}
\date{}
\maketitle

\begin{abstract} 
\normalsize 
I review the organizational learning and human capital literatures to understand the antecedents, assumptions and mechanisms behind each literature\textquotesingle s explanation of the source of competitive advantages in firms. I focus on the tension between the levels of analysis in the two literatures to propose a study to understand how the nature of human capital of the firm (general vs firm specific human capital) and the nature of organizational learning  in the firm (explorative or exploitative) tradeoff in determining strategic choices and firm profits. I contribute back to both literatures by nuancing each literature\textquotesingle s explanation of the source of competitive advantage in firms.
\end{abstract}


{\textbf{Keywords:} \\\indent Exploitation, Exploration, Firm Specific Human Capital, General Human Capital,  Organizational Learning}

\newpage
\pagestyle{fancy}
\fancyhf{}
\lhead{Organizational Learning and Human Capital}
\rhead{\thepage}

\begin{center}
\textbf{Organizational Learning and Human Capital: Similarities, Tensions \& a Research Proposal}
\end{center}

Building on the Resource Based View of the firm \citep{Barney1991}, scholars of human capital have suggested that human capital is a source of competitive advantage in firms. Researchers in the organizational learning tradition, have built upon evolutionary \citep{Nelson1982} and behavioral theories \citep{Cyert1963, March1958} to suggest that the collective capability of firms are a source of competitive advantage in firms. Both traditions are influential and have provided rich and insightful understanding of the sources of firm heterogeneity. Owing to level differences however, the central tension has been whether explanations of collective outcomes should focus on the individual or organizational and cultural level \citep{Udehn2001}. I discuss each of the traditions briefly in the following sections, and highlight salient similarities and differences in their assumptions and explanations of the antecedents of firm performance. I then propose a study bringing the two literatures and suggest contributions that add to both literatures.

\section{Review}
\subsection{Organizational Learning}
The literature on organizational learning defines organizational learning as the process by which an organization acquires knowledge as a result of its experiences. These experiences may be either direct (through its own activities) or indirect (through observing the actions of others). Organizational learning may also be viewed as a change in the organization\textquotesingle s knowledge that occurs as a function of experience \citep{Levitt1988}.  The knowledge so gained from experience is seen as manifesting in changes in the cognition of organization members, in the organization\textquotesingle s routines and in observable and unobservable performance characteristics such as absorptive capacity \citep{Cohen1990}, speed or accuracy. Scholars have argued that organizational knowledge is embedded in repositories \citep{Walsh1991}, tools \citep{Kogut1992}, routines \cite{Nelson1982}, social networks and transactive memory systems. Such organizational knowledge, once embedded in the organizations is seen to have a life of its own independent on the individuals who helped generate such knowledge. Network scholars have also suggested that the knowledge is embodied in connections between individuals. The implication for firm performance is that once such knowledge is embedded within the organization, it may not be subject to expropriation by individuals. Being a property of the organization, it becomes a factor in determining the competitive advantage and hence profits earned by firms.  

\subsection{Human Capital}
Building upon \cite{Becker1962} seminal work, scholars in the human capital tradition maintain that human capital is heterogeneous and that firms may be endowed with different types and degrees of human capital. In this tradition, human capital captures stocks of education, information and health that have been accumulated both on and off the job \citep{Becker1962}. In this view, firms compete for an optimal match between themselves and the most valuable human capital so as to gain a competitive advantage. 

Within the human capital literature, there has been some debate about whether individuals possess portfolios of both general human capital (GHC) and firm specific human capital (FSHC) \citep{Campbell2012} or if human capital is never specific in the sense that no other firm can use it \citep{Lazear2009}. has suggested that all human capital, is general. Given the important role of recruitment, formal education and training for the acquisition of new knowledge and skills on the one side and the generation of knowledge as a by-product of the work people do in the normal course of their day-to-day activities, I find it hard to take either stand. I view human capital as spanning a continuum between completely general on the one end and being completely firm specific on the other. This view is consistent with that of scholars who have suggested that difficulties with transferring firm-specific human capital to different firms provides the theoretical isolating mechanism that allow a firm to capture quasi-rents \citep{Barney1991, Campbell2012}. In summary, the collective wisdom of strategic human capital scholars is that firm-specific human capital is an important source of sustainable competitive advantage in firms.

\subsection{Similarities}
It is readily apparent that both literatures deal with a few similar concepts. Learning, knowledge, absorptive capacity can all be traced to individual-level constructs that have been applied to the individual level (in the human capital literature) and to the organization as if it were like a person (in the organizational learning literature). For firm-level absorptive capacity \citep{Cohen1990} for example, the founding authors  did not explicitly theorize about how the concept might need to change and evolve when applied to the organization acting as a single person. Similar analogies have been drawn for cognition and learning as evidenced in  \cite{Gavetti2012b, Gavetti2005b}. It therefore seems common that constructs in the two traditions draw from identical sources but then take a life of their own within their respective traditions. This leads to potential tensions while attempting a synthesis of theories.

\subsection{Tensions}
These tensions are apparent even in some of the most highly cited and influential works in strategic management. Several scholars have  argued that individual-level considerations simply are not relevant for strategy and firm-level outcomes \citep{Henderson1994, Kogut1992, Kogut1996, Nahapiet1998, Spender1996}. There has been a tendency in strategy literature to place a higher emphasis on macro factors such as firm-level knowledge and competencies, social capital, networks, and other collective constructs. Strong paradigms lead to contradictory assumptions. For example, \citep{Henderson1994} assume that individuals are homogeneous when they model organizational competencies. This assumption is at direct odds with the human capital tradition that believes that human capital is fundamentally heterogenous and variedly endowed in people. This leads to conclusions that stars may not be a source of competitive advantage since if they were they would appropriate all rents associated with their abilities. \par

Another source of tension, the one that I would like to pursue in a study, is the one caused by conflicting recommendations by the two literatures. Stemming from level issues, managers may be perplexed to be told to invest in FSHC on the one hand (by human capital scholars), and to increase explorative learning on the other (by organization learning scholars) so as to improve rent generation. The practitioner\textquotesingle s perplexion may well turn into asphyxia when the he/she learns from the organizational learning scholar that investing in FSHC is not likely to improve firm performance because the stars will appropriate all rents associated with their superior abilities. In the study I propose in this paper, I hope to elucidate a nuanced argument to this seeming paradox. I have attempted to capture some of these salient similarities and differences in the two literatures Table ~\ref{qe2a}

\begin{table}
\begin{centering}
\caption {Comparison of Organizational Learning and Human Capital Literatures}
\label{qe2a}
{\tabulinesep=1.4mm
\begin{tabu}{|p{0.25\textwidth}|p{0.35\textwidth}|p{0.35\textwidth}|}
\hline
& \textbf{Organizational Learning} & \textbf{Human Capital}  \\
\hline   
\textbf{Theoretical Tradition}& Dynamic Capabilities, Evolutionary Perspectives & Resource Based View  \\\hline
\textbf{Unit of Focus}& Teams and Organizations & Individuals  \\\hline
\textbf{Source of Competitive Advantage}& Exploration Capability & Firm Specific Human Capital  \\\hline
\textbf{Primary Tensions}& Collective assumed as a person & Individual assumed to operate independent of the organization  \\\hline
\end{tabu}}
\end{centering}
\end{table} 

Scholars have suggested two potential reasons for such contradictory truth claims across traditions. First, the performance relationship with explained causes is subject to endogeneity concerns. This endogeneity may stem from various sources such as omitted variables, simultaneity or non-random measurement error. Second, level issues are not addressed in either tradition and the mechanisms at work between levels have not been explained. In my research proposal, I endeavor to push the needle on the latter issue.

\subsection{A Synthesis}
One way to look at the extreme positions between the two schools is to see it as an opportunity to craft research at the intersection of the two traditions. \cite{Coff2011} explored human capital-based competitive advantage and, and noted that when causal ambiguity derived from tacit knowledge creates problems of imitation for not just competitors, but also for firm insiders. Some work \citep{Coff2011, Ployhart2011} has begun to address some of the linkages between levels, and goes by the term ``micro foundations movement". As \cite{Burgelman1991} observed so astutely, the genius of surviving organizations lies in their ability to benefit from both winning and losing individual strategic initiatives through their capacity for learning. I build on this wisdom in suggesting that well performing firms may well be ambidextrous \citep{OReilly2004, OReilly2008}. In the next section, I proceed to propose my research study.

\section{Research Proposal}
As noted in the previous section, the human capital and organizational learning literatures suggest contradictory effects for firm rents. Specifically, while the organizational learning suggests greater exploration so as to increase rents (termed Schumpeterian rents since they arise from the novelty of the innovations produces), while the human capital literature suggests increasing the level of FSHC. However, the organizational learning literature view higher FSHC as reducing or eliminating potential rents arising from innovation because the valuable human capital is expected to appropriate those gains away.  The question that I therefore ask in this research proposal is: How does FSHC affect the relationship between higher exploration and higher firm profits?

\subsection{Theory}
From an organizational learning perspective, firms are concerned with two issues. First, if they can effectively absorb and appropriate the spillover knowledge \citep{Cohen1990}, and Second, if there are additional complementary opportunities for learning from the originator \citep{Teece1986}. In either case, the firm can expect to benefit from generating Schumpeterian rents from its innovations in the short run. On the one hand, the absorptive capacity of the firm for external technological knowledge is dependent on the degree of its knowledge in a particular technological field \citep{Schoenmakers2006}. On the other hand, the accumulation of tacit knowledge \citep{Polanyi1966} related to the technology is also on specialization of the knowledge \citep{Enright1991}. Therefore the organization learning literature would suggest that higher FSHC (a manifestation of the specialization of the knowledge discussed above) would increase absorptive capacity of the firm and thereby increase potential value appropriated from the innovations exploiting the spillover knowledge that comes from exploration. Over time however this knowledge gained from explorative activity is expected to get translated into organizational routines, inter-person networks and organizational knowledge repositories. Being routine-based, history-dependent, and target-oriented in the long run, organizations  learning by encoding inferences from history into routines that guide behavior. As the knowledge enters routines and other organizational repositories of knowledge, the appropriability capabilities of any specific people (who possess highly specialized, or FSHC) reduces to zero. At this stage, any rents that the organization is able to extract out of this knowledge turns in Ricardian rents arising out of superior resources and capabilities. Unlike Schumpeterian rents that exist only temporarily, Ricardian rents are generated out of sustainable competitive advantage.

On the other hand, the FSHC paradox is that firms need workers to invest in FSHC, but workers do not always want to make these investments \citep{Wang2006}. Workers have a choice about whether to invest in general or FSHC. If they invest in general human capital they can always take their human capital to another employer and get paid appropriately for their skills. If they invest in FSHC, they can extract the value from those skills only in their current firm. Risk-averse employees are generally more likely to invest in general human capital, because there is less risk of losing the value of the human capital investment \citep{Wang2006}. 

In the following, I attempt to capture the extent and direction of the rents gained to organizational learning and that lost due to appropriation by highly valuable workers. By quantifying the extent of rents generated and rents appropriated away, I hope to provide a framework for managerial decisions vis-a-vis the exploration/exploitation and GHC/FSHC dilemma.

The decision problem for the organization may be framed as a choice depending on the nature of the Human Capital (GHC or FSHC) and the nature of Organizational Learning strategy. I present this decision choice using a 2x2 matrix that is presented in Table ~\ref{qe2b}. The arrows within each of the boxes suggest the direction of the effect on firm rents. An up arrow suggests that the factor helps increase firm rents, while a down arrow suggests that the factor works to decrease firm rents. Firm decisions can be made based on an estimate of the net-effect of the two factors at play. For example, in the bottom left quadrant managers pursue a strategy of exploration using FSHC. The box suggests that firm can generate Schumpeterian rents out of its innovations (arising out of explorative activity), but also that the firm incurs costs due to FSHC appropriating rents from the firm. Firm outcome is determined by the difference in the quantum of the two. Higher rents paid to human capital will offset rents generated through innovation. The same logic may be applied to the other quadrants.

\begin{table}[h]
\renewcommand\arraystretch{2.5} % provide a bit taller rows
\centering
\caption{Firm Profit Effects from Organizational Learning and Human Capital Characteristics}
\label{qe2b}
\begin{tabularx}{\textwidth}{@{} c c |c |c| } % use 'Y' for first column
&\multicolumn{1}{c}{}&\multicolumn{2}{c}{\textbf{Organizational Learning}}\\[-2ex]
&\multicolumn{1}{c}{}
&\multicolumn{1}{c}{\textbf{Explorative}}&\multicolumn{1}{c}{\textbf{Exploitative}}\\\cline{3-4}
\multirow{4}{*}{\rotatebox{90}{\textbf{Human Capital (HC)}}}
&\multirow{2}{*}{\textbf{General HC}}&\textbf{$\uparrow$ Schumpeterian Rent}&\textbf{$\uparrow$ Ricardian Rent}\\
&&\textbf{$\uparrow$ No rents to human capital}&\textbf{$\downarrow$ Rents to human capital}\\\cline{3-4}
&\multirow{2}{*}{\textbf{Firm Specific HC}}&\textbf{$\uparrow$ Schumpeterian Rent}&\textbf{$\uparrow$ Ricardian Rent}\\
&&\textbf{$\downarrow$ Rents to human capital}&\textbf{$\uparrow$ No rents to human capital}\\\cline{3-4}
\end{tabularx}
\end{table}


\subsubsection{Creating sustainable competitive advantages}
Organization learning literature suggests that specialized knowledge is routinized over time. As the dependence on FSHC reduces due to routinization of knowledge and the associated causal ambiguity arising out of tacit knowledge, the appropriability of rents by highly knowledgable workers reduces. Therefore managers intending to increase rents from exploration may chose a strategy where the rents from innovation are appropriated by highly skilled workers in the short run, but rents accrue to the firm as knowledge present in highly knowledgable workers is transformed into organizational routines. This leads me to my first hypothesis.\\

\begin{hypothesis}
{Hypothesis 1: Long term rent generating capacity of firms from exploration is characterized by an increase in costs due to appropriation by highly knowledgable employees initially, and a decrease in such costs later.
\\}
\end{hypothesis}

\subsubsection{Improving operational performance}
Given a certain organizational strategy (exploration or exploitation), the main lever by which managers may improve firm profits is by reducing costs due to appropriation by highly knowledgable workers. Managers intending to improve short term operational performance may do by reducing the extent of FSHC employed in the firm. This leads me to my second hypothesis.\\

\begin{hypothesis}
{Hypothesis 2: Given a certain organizational learning strategy for a firm, the short term profit performance in firms varies inversely with the extent of FSHC employed in the firm\\}
\end{hypothesis}


\subsection{Data and Method}

I propose to set this study in the semiconductor industry. Exploration behavior maybe obtained by the extent of boundary spanning in a firm\textquotesingle s patents in a year. The extent of firm specificity of human capital may be obtained from industry and product information about whether the firm has developed products on an open platform or a closed platform in a given year. Firm outcome measure can be a market measure such as Tobin\textquotesingle s Q. I propose a  30 year longitudinal panel analysis including firm fixed effects, and year dummies along with controls for prior patenting performance and firm size. Additional controls could include those affecting the quality and nature of human capital, such as reputations for toughness in patent enforcement \citep{Agarwal2009a} and non-compete laws \citep{Marx2009}.

\section{Contributions}
My study draws from prior literature of two influential strategy streams: Organizational learning, and Human capital. I intent to contribute to the literature on organizational learning, a reinforcement of well established principles that routines are an important source of firm profits, and that routines are changed only slowly and that there exists significant causal ambiguity due to tacit knowledge. To the literature on human capital, I intend to contribute to reinforce the insight that while FSHC can be a source of firm level advantages, that effect is moderated by the extent to which the knowledge carried by FSHC is routinized or otherwise absorbed within organizational structures.


\section{Summary}
Building on a review of the literatures on organizational learning and human capital, I identified that due to level issues and causal ambiguity arising out of tacit knowledge, the two literatures suggested opposite profit effects for firms adopting an exploration strategy using firm specific human capital. I proposed a study that quantified the extent and direction of these effects, and provided a simple framework to help managers identify the appropriate strategy at the intersection of organizational learning and human capital.


\begin{singlespace}
\renewcommand{\refname}{REFERENCES}
\bibliography{/Users/aiyenggar/code/bibliography/aiyenggar} 
\bibliographystyle{ai-amjlike}
\end{singlespace}

\end{document}
