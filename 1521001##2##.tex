%1521001##1##.tex
\documentclass[12pt,letterpaper]{article}
\usepackage{mathptmx}
\usepackage[margin=1in]{geometry}

\usepackage{tabularx, longtable}  % for 'tabularx' environment and 'X' column type
\usepackage{ragged2e}  % for '\RaggedRight' macro (allows hyphenation)
\newcolumntype{Y}{>{\RaggedRight\arraybackslash}X} 
\usepackage{booktabs}  % for \toprule, \midrule, and \bottomrule macros 

\usepackage{setspace}
\singlespacing
  
\usepackage{amssymb,latexsym}
\usepackage[round,sort]{natbib}
\usepackage{fancyhdr}
\usepackage{lastpage}
\usepackage{graphicx,multirow}
\graphicspath{ {qe2/} }

% Bold Table and Figure captions
\usepackage{caption}
\captionsetup{figurename=FIGURE}
\captionsetup{tablename=TABLE}
\captionsetup[figure]{labelfont=bf}
\captionsetup[table]{labelfont=bf}
  
% Turns off all section numbering
\setcounter{secnumdepth}{0} 

  % Places all tables at end of document and creates AOM-style table-here placeholders
  \usepackage[nolists]{endfloat} % Places all figures and charts at end of manuscript and adds 'insert table x about here' lines.
  \renewcommand{\figureplace}{
    \begin{center}
    \begin{singlespace}
    ------------------------------------\\
    Insert \figurename \ \thepostfig\ about here.\\
    ------------------------------------
    \end{singlespace}
    \end{center}}
  \renewcommand{\tableplace}{%
    \begin{center}
    \begin{singlespace}
    ------------------------------------\\
    Insert \tablename \ \theposttbl\ about here.\\
    ------------------------------------
    \end{singlespace}
    \end{center}}

  \usepackage{titlesec}
   \titleformat{\title}
       {\filcenter\normalfont\bfseries\uppercase}{\thetitle}{1em}{}
  \titleformat{\section}
    {\filcenter\normalfont\bfseries\uppercase}{\thesection}{1em}{}
  \titleformat{\subsection}
    {\normalfont\bfseries}{\thesubsection}{1em}{}
  \titleformat{\subsubsection}[runin]
   {\normalfont\bfseries\slshape}{\thesubsubsection}{1em}{\hspace*{\parindent}}
       
\usepackage{tabu}
\usepackage{textcomp}
\usepackage{listings}
\usepackage{hyperref}
\usepackage{verbatim}
\usepackage{tabu}
\hypersetup{
    colorlinks=true,
    linkcolor=blue,
    filecolor=cyan,      
    urlcolor=cyan,
    citecolor=blue,
}

\usepackage{etoolbox}

\makeatletter

% Patch case where name and year are separated by aysep
\patchcmd{\NAT@citex}
  {\@citea\NAT@hyper@{%
     \NAT@nmfmt{\NAT@nm}%
     \hyper@natlinkbreak{\NAT@aysep\NAT@spacechar}{\@citeb\@extra@b@citeb}%
     \NAT@date}}
  {\@citea\NAT@nmfmt{\NAT@nm}%
   \NAT@aysep\NAT@spacechar\NAT@hyper@{\NAT@date}}{}{}

% Patch case where name and year are separated by opening bracket
\patchcmd{\NAT@citex}
  {\@citea\NAT@hyper@{%
     \NAT@nmfmt{\NAT@nm}%
     \hyper@natlinkbreak{\NAT@spacechar\NAT@@open\if*#1*\else#1\NAT@spacechar\fi}%
       {\@citeb\@extra@b@citeb}%
     \NAT@date}}
  {\@citea\NAT@nmfmt{\NAT@nm}%
   \NAT@spacechar\NAT@@open\if*#1*\else#1\NAT@spacechar\fi\NAT@hyper@{\NAT@date}}
  {}{}

\lstset{
basicstyle=\ttfamily,
columns=flexible,
breaklines=true
}
\newenvironment{hypothesis}{
  	\itshape
  	\leftskip=\parindent \rightskip=\parindent
  	\noindent\ignorespaces}

\fancypagestyle{plain}{
  \renewcommand{\headrulewidth}{0pt}
  \fancyhf{}
}	


\begin{document}
\title{Organizational Learning and Human Capital: Similarities, Tensions and a Research Proposal}
\date{}
\maketitle

\begin{abstract} 
\normalsize 
I review the organizational learning and human capital literatures to understand the antecedents, assumptions and mechanisms behind each literature\textquotesingle s explanation of the source of competitive advantages in firms. I focus on the tension between the levels of analysis in the two literatures to propose a study to understand how the nature of human capital of the firm (general vs firm specific human capital) and the nature of organizational learning  in the firm (explorative or exploitative) tradeoff in determining strategic choices and firm profits. I contribute back to both literatures by nuancing each literature\textquotesingle s explanation of the source of competitive advantage in firms.
\end{abstract}


{\textbf{Keywords:} \\\indent Exploitation, Exploration, Firm Specific Human Capital, General Human Capital,  Organizational Learning}

\newpage
\pagestyle{fancy}
\fancyhf{}
\lhead{Organizational Learning and Human Capital}
\rhead{\thepage}

\begin{center}
\textbf{Organizational Learning and Human Capital: Similarities, Tensions \& a Research Proposal}
\end{center}

Building on the Resource Based View of the firm \citep{Barney1991}, scholars of human capital have suggested that human capital is a source of competitive advantage in firms. Researchers in the organizational learning tradition, have built upon evolutionary \citep{Nelson1982} and behavioral theories \citep{Cyert1963, March1958} to suggest that the collective capability of firms are a source of competitive advantage in firms. Both traditions are influential and have provided rich and insightful understanding of the sources of firm heterogeneity. Owing to level differences however, the central tension has been whether explanations of collective outcomes should focus on the individual or organizational and cultural level \citep{Udehn2001}. I discuss each of the traditions briefly in the following sections, and highlight salient similarities and differences in their assumptions and explanations of the antecedents of firm performance. I then propose a study bringing the two literatures and suggest contributions that add to both literatures.

\section{Review}
\subsection{Organizational Learning}
The literature on organizational learning defines organizational learning as the process by which an organization acquires knowledge as a result of its experiences. These experiences may be either direct (through its own activities) or indirect (through observing the actions of others). Organizational learning may also be viewed as a change in the organization\textquotesingle s knowledge that occurs as a function of experience \citep{Levitt1988}.  The knowledge so gained from experience is seen as manifesting in changes in the cognition of organization members, in the organization\textquotesingle s routines and in observable and unobservable performance characteristics (e.g., absorptive capacity \citep{Cohen1990}, speed or accuracy). Scholars have argued that organizational knowledge is embedded in repositories \citep{Walsh1991}, tools \citep{Kogut1992}, routines \cite{Nelson1982}, social networks and transactive memory systems. Such organizational knowlege, once embedded in the organizations is seen to have a life of its own indepedent on the individuals who helped generate such knowledge. Network scholars have also suggested that the knowledge is embodied in connections between individuals. The implication for firm performance is that once such knowledge is embedded within the organization, it may not be subject to expropriation by individuals. Being a property of the organization, it becomes a factor in determining the competitive advantage and hence profits earned by firms.  

\subsection{Human Capital}
Building upon \cite{Becker1962} seminal work, scholars in the human capital tradition maintain that human capital is heterogeneous and that firms may be endowed with different types and degrees of human capital. In this tradition, human capital captures stocks of education, information and health that have been accumulated both on and off the job \citep{Becker1962}. In this view, firms compete for an optimal match between themselves and the most valuable human capital so as to gain a competitive advantage. 

Within the human capital literature, there has been some debate about whether individuals possess portfolios of both general human capital (GHC) and firm specific human capital (FSHC) \citep{Campbell2012} or if human capital is never specific in the sense that no other firm can use it \citep{Lazear2009}. has suggested that all human capital, is general. Given the important role of recruitment, formal education and training for the acquisition of new knowledge and skills on the one side and the generation of knowledge as a by-product of the work people do in the normal course of their day-to-day activities, I find it hard to take either stand. I view human capital as spanning a continuum between completely general on the one end and being completely firm specific on the other. This view is consistent with that of scholars who have suggested that difficulties with transferring firm-specific human capital to different firms provides the theoretical isolating mechanism that allow a firm to capture quasi-rents \citep{Barney1991, Campbell2012}. In summary, the collective wisdom of strategic human capital scholars is that firm-specific human capital is an important source of sustainable competitive advantage in firms.

\subsection{Similarities}
It is readily apparent that both literatures deal with a few similar concepts. Learning, knowledge, absorptive capacity can all be traced to individual-level constructs that have been applied to the individual level (in the human capital literature) and to the organization as if it were like a person (in the organizational learning literature). For firm-level absorptive capacity \citep{Cohen1990} for example, the founding authors  did not explicitly theorize about how the concept might need to change and evolve when applied to the organization acting as a single person. Similar analogies have been drawn for cognition and learning as evidenced in  \cite{Gavetti2012b, Gavetti2005b}. It therefore seems common that constructs in the two traditions draw from identical sources but then take a life of their own within their respective traditions. This leads to potential tensions while attempting a synthesis of theories.

\subsection{Tensions}
These tensions are apparent even in some of the most highly cited and influential works in strategic management. Several scholars have  argued that individual-level considerations simply are not relevant for strategy and firm-level outcomes \citep{Henderson1994, Kogut1992, Kogut1996, Nahapiet1998, Spender1996}. There has been a tendency in strategy literature to place a higher emphasis on macro factors such as firm-level knowledge and competencies, social capital, networks, and other collective constructs. Strong paradigms lead to contradictory assumptions. For example, \citep{Henderson1994} assume that individuals are homogeneous when they model organizational competencies. This assumption is at direct odds with the human capital tradition that believes that human capital is fundamentally heterogenous and variedly endowed in people. This leads to conclusions that stars may not be a source of competitive advantage since if they were they would appropriate all rents associated with their abilities. \par

Another source of tension, the one that I would like to pursue in a study, is the one caused by conflicting recommendations by the two literatures. Stemming from level issues, managers may be perplexed to be told to invest in FSHC on the one hand (by human capital scholars), and to increase explorative learning on the other (by organization learning scholars) so as to improve rent generation. The practitioner\textquotesingle s preplexion may well turn into asphyxia when the he/she learns from the organizational learning scholar that investing in FSHC is not likely to improve firm performance because the stars will appropriate all rents associated with their superior abilities. In the study I propose in this paper, I hope to elucidate a nuanced argument to this seeming paradox. I have attempted to capture some of these salient similarities and differences in the two literatures Table ~\ref{qe2a}

\begin{table}
\begin{centering}
\caption {Comparison of Organizational Learning and Human Capital Literatures}
\label{qe2a}
{\tabulinesep=1.4mm
\begin{tabu}{|p{0.25\textwidth}|p{0.35\textwidth}|p{0.35\textwidth}|}
\hline
& \textbf{Organizational Learning} & \textbf{Human Capital}  \\
\hline   
\textbf{Theoretical Tradition}& Dynamic Capabilities, Evolutionary Perspectives & Resource Based View  \\\hline
\textbf{Unit of Focus}& Teams and Organizations & Individuals  \\\hline
\textbf{Recommendations for Competitive Advantage}& Theta & Beta  \\\hline
\textbf{Primary Tensions}& Theta & Beta  \\\hline
\textbf{Tradition}& Theta & Beta  \\\hline
\textbf{Tradition}& Theta & Beta  \\\hline
\textbf{Tradition}& Theta & Beta  \\\hline
\textbf{Tradition}& Theta & Beta  \\\hline 
\end{tabu}}
\end{centering}
\end{table} 

Scholars have suggested two potential reasons for such contradictory truth claims across traditions. First, the performance relationship with explained causes is subject to endogeneity concerns. This endogeneity may stem from various sources such as omitted variables, simultaneity or non-random measurement error. Second, level issues are not addressed in either tradition and the mechanisms at work between levels have not been explained. In my research proposal, I endeavor to push the needle on the latter issue.

\subsection{A Synthesis}
One way to look at the extreme positions between the two schools is to see it as an opportunity to craft research at the intersection of the two traditions. \cite{Coff2011} explored human capital-based competitive advantage and, and noted that when causal ambiguity derived from tacit knowledge creates problems of imitation for not just competitiors, but also for firm insiders. Some work \citep{Coff2011, Ployhart2011} has begun to address some of the linkages between levels, and goes by the term ``microfoundations movement". As \cite{Burgelman1991} observed so astutely, the genius of surviving organizations lies in their ability to benefit from both winning and losing individual strategic initiatives through their capacity for learning. I build on this wisdom in suggesting that well performing firms may well be ambidexterous \citep{OReilly2004, OReilly2008}. In the next session, I proceed to propose my research study.

\section{Research Proposal}
The tension that I wish to address in this study is the one between balancing exploration-exploitation and the one between general Human Capital (HC) and Firm Specific Human Capital (FSHC). Conflict - levels of analysis. OL literature assumes all human capital to be homogenous. If one were to accept that HC were heterogenous, and that Organizational Learning is driven by routines, then ...

Plot the 2x2 and suggest that there are costs and sources of profits in each of the quadrants. Discuss it, and suggest that the net effect is unclear. Talk about non-competes, reputations for enforcement etc - if they are in favor of employees then they are more likely to extract rents. Alternatively.

Product market - causal ambiguity of some sort.

\subsection{Research Question}


\subsection{Theory}
From an organizational learning perspective, recipient firms are likely to be concerned with two issues related to ability to learn: first, whether they can effectively understand and apply the spilled over knowledge; and second, the additional complementary opportunities for learning from the originator. If the knowledge is in one of the areas of specialization of the recipient firm, the recipient?s absorptive capacity \citep{Cohen1990} for the spilled over knowledge as well as for complementary knowledge is likely to be high. The absorptive capacity of the firm for external technological knowledge is dependent to a significant extent on the degree of its knowledge in a particular technological field \citep{Schoenmakers2006}. Further, the development and accumulation of tacit knowledge \citep{Polanyi1966} related to the technology is also dependent on specialization \citep{Enright1991}. Thus a recipient with greater specialization will possess well-developed internal mechanisms for understanding and exploiting spillover knowledge.

Extensive attention has focused on understanding the relative advantage of two different modes of organizational learning, exploration and exploitation \cite{March1991a}. Exploration includes the search for new possibilities, experimentation and risk-taking. For instance, an electronic firm positioning itself as an innovator may want to explicitly set up exploratory learning processes to collect novel ideas from consumer focus groups on a regular basis. Conversely, exploitation involves efficiency and refinement. For example, a firm focusing on exploitation might set up processes to identify and correct the causes of production defects.

Organizational learning is viewed as routine-based, history-dependent, and target-oriented. Organizations are seen as learning by encoding inferences from history into routines that guide behavior.

The FSHC paradox is that firms need workers to invest in firm-specific human capital, but workers don?t always want to make these investments \citep{Wang2006}. Workers have a choice about whether to invest in general or firm-specific human capital. If they invest in general human capital they can always take their human capital to another employer and get paid appropriately for their skills. If
they invest in firm-specific human capital, they can extract the value from those skills only in their current firm. If the firm goes out of business, if the employees need to change jobs for personal reasons, or if the firm decides to act opportunistically and not compensate the employees for these skills, the employees cannot achieve any reasonable return on their investments. Generally, risk-averse employees are more likely to invest in general human capital, because there is less risk of losing the value of the human capital investment (see \cite{Wang2006} for a more detailed review of the paradox). The focus on this paradox has led some scholars to claim
that we have a global underinvestment in firmspecific human capital that may be holding back our economic growth and development.

The resources and (dynamic) capabilities perspective?which we will refer to as the capabilities approach?maintains that firms possessing, creating, and adapting resources and capabilities can capture and sustain competitive advantage \citep{Barney1991, 
Penrose1959, Teece1997}. The governance approach maintains that higher economic performance can be achieved by investing in complementary and cospecialized assets \citep{Helfat1997, Teece1986} and by governing them in an economizing way \citep{Oxley1997, Williamson1985}.

Suggest that the tension is that when organizational learning would suggest that firms should explore more, it would seem that the human capital literature would suggest that stars may appropriate the value leaving little for the firm. The propositions should be about how managers incentivize one or the other so as to retain the capacity for profits for the firm in the long run (the mechanism could be the transformation into routines)

One hypothesis about how managers let go of temporary rents so as to capture future value (the routines argument)

Another about how managers are able to make hay while the sun shines by leveraging firm specific human capital during this phase.
In the build up argue why one or the other rent cancels the other.

\begin{hypothesis}
{Hypothesis 1a: When the institutional field is open to influence, slow learning adversarial agents will raise overall performance higher than slow learning agents with a neutral orientation\\}
\end{hypothesis}

Reference to Table ~\ref{qe2b}

\begin{table}[h]
\renewcommand\arraystretch{2.5} % provide a bit taller rows
\centering
\caption{Firm Profit Effects from Organizational Learning and Human Capital Characteristics}
\label{qe2b}
\begin{tabularx}{\textwidth}{@{} c c |c |c| } % use 'Y' for first column
&\multicolumn{1}{c}{}&\multicolumn{2}{c}{\textbf{Organizational Learning}}\\[-2ex]
&\multicolumn{1}{c}{}
&\multicolumn{1}{c}{\textbf{Explorative}}&\multicolumn{1}{c}{\textbf{Exploitative}}\\\cline{3-4}
\multirow{4}{*}{\rotatebox{90}{\textbf{Human Capital (HC)}}}
&\multirow{2}{*}{\textbf{General HC}}&\textbf{$\uparrow$ Schumpeterian Rent}&\textbf{$\uparrow$ Ricardian Rent}\\
&&\textbf{$\downarrow$ Rents to human capital}&\textbf{$\downarrow$ Rents to human capital}\\\cline{3-4}
&\multirow{2}{*}{\textbf{Firm Specific HC}}&\textbf{$\uparrow$ Schumpeterian Rent}&\textbf{$\uparrow$ Ricardian Rent}\\
&&\textbf{$\uparrow$ No rents to human capital}&\textbf{$\uparrow$ No rents to human capital}\\\cline{3-4}
\end{tabularx}
\end{table}

\subsubsection{Leading into H2a}
This trend is confirmed further in Figure ~\ref{fig:3a} where the learning rates of agents are increased even further to \textquotesingle Fast\textquotesingle .

\begin{hypothesis}
{Hypothesis 2a: For the same initial outcome preferences,  the overall performance score varies curvilinearly with difference in the rates of learning of the agent and the institutional field\\}
\end{hypothesis}

\subsection{Data and Method}

What is the sample. Why is it chosen. How is it going to contribute. Semiconductor industry. Classify. Each of the boxes. Technology change and framework of labor laws as moderating variables. Argue that one is more than the other in the appropriate and therefore leading into the hypothesis. For every year 0 1. Exploration Exploitation - patents filed in that year. Have they been broad or narrow.
Legal framework - change when there is a regulatory change. Like when Michigan.

\subsubsection{Dependent Variable}
Firm outcome - Tobin's Q. Some form of market measure.

\subsubsection{Independent Variables}

\subsubsection{Moderating Variables}

\section{Contributions}
What do you contribute to each literature. Basically nuance. Adding framework of labor laws, routines, exploration-exploitation. For Organizational Learning, the role of FSHC and GHC.

\section{Limitations}
Suggest how this study may help inform the literatures that it is drawing from, and the interesting research avenues it will open up. Discuss level of generalizability.

\section{Summary}
Recap and motivate interest in framework, in theoretical value as well as in the particular empirical setting.


\begin{singlespace}
\renewcommand{\refname}{REFERENCES}
\bibliography{/Users/aiyenggar/code/bibliography/aiyenggar} 
\bibliographystyle{ai-amjlike}
\end{singlespace}

\end{document}
