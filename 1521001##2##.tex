%1521001##1##.tex
\documentclass[12pt,letterpaper]{article}
\usepackage{mathptmx}
\usepackage[margin=1in]{geometry}

\usepackage{tabularx, longtable}  % for 'tabularx' environment and 'X' column type
\usepackage{ragged2e}  % for '\RaggedRight' macro (allows hyphenation)
\newcolumntype{Y}{>{\RaggedRight\arraybackslash}X} 
\usepackage{booktabs}  % for \toprule, \midrule, and \bottomrule macros 

\usepackage{setspace}
\singlespacing
  
\usepackage{amssymb,latexsym}
\usepackage[round,sort]{natbib}
\usepackage{fancyhdr}
\usepackage{lastpage}
\usepackage{graphicx,multirow}
\graphicspath{ {qe2/} }

% Bold Table and Figure captions
\usepackage{caption}
\captionsetup{figurename=FIGURE}
\captionsetup{tablename=TABLE}
\captionsetup[figure]{labelfont=bf}
\captionsetup[table]{labelfont=bf}
  
% Turns off all section numbering
\setcounter{secnumdepth}{0} 

  % Places all tables at end of document and creates AOM-style table-here placeholders
  \usepackage[nolists]{endfloat} % Places all figures and charts at end of manuscript and adds 'insert table x about here' lines.
  \renewcommand{\figureplace}{
    \begin{center}
    \begin{singlespace}
    ------------------------------------\\
    Insert \figurename \ \thepostfig\ about here.\\
    ------------------------------------
    \end{singlespace}
    \end{center}}
  \renewcommand{\tableplace}{%
    \begin{center}
    \begin{singlespace}
    ------------------------------------\\
    Insert \tablename \ \theposttbl\ about here.\\
    ------------------------------------
    \end{singlespace}
    \end{center}}

  \usepackage{titlesec}
   \titleformat{\title}
       {\filcenter\normalfont\bfseries\uppercase}{\thetitle}{1em}{}
  \titleformat{\section}
    {\filcenter\normalfont\bfseries\uppercase}{\thesection}{1em}{}
  \titleformat{\subsection}
    {\normalfont\bfseries}{\thesubsection}{1em}{}
  \titleformat{\subsubsection}[runin]
   {\normalfont\bfseries\slshape}{\thesubsubsection}{1em}{\hspace*{\parindent}}
       
\usepackage{tabu}
\usepackage{textcomp}
\usepackage{listings}
\usepackage{hyperref}
\usepackage{verbatim}
\usepackage{tabu}
\hypersetup{
    colorlinks=true,
    linkcolor=blue,
    filecolor=cyan,      
    urlcolor=cyan,
    citecolor=blue,
}

\usepackage{etoolbox}

\makeatletter

% Patch case where name and year are separated by aysep
\patchcmd{\NAT@citex}
  {\@citea\NAT@hyper@{%
     \NAT@nmfmt{\NAT@nm}%
     \hyper@natlinkbreak{\NAT@aysep\NAT@spacechar}{\@citeb\@extra@b@citeb}%
     \NAT@date}}
  {\@citea\NAT@nmfmt{\NAT@nm}%
   \NAT@aysep\NAT@spacechar\NAT@hyper@{\NAT@date}}{}{}

% Patch case where name and year are separated by opening bracket
\patchcmd{\NAT@citex}
  {\@citea\NAT@hyper@{%
     \NAT@nmfmt{\NAT@nm}%
     \hyper@natlinkbreak{\NAT@spacechar\NAT@@open\if*#1*\else#1\NAT@spacechar\fi}%
       {\@citeb\@extra@b@citeb}%
     \NAT@date}}
  {\@citea\NAT@nmfmt{\NAT@nm}%
   \NAT@spacechar\NAT@@open\if*#1*\else#1\NAT@spacechar\fi\NAT@hyper@{\NAT@date}}
  {}{}

\lstset{
basicstyle=\ttfamily,
columns=flexible,
breaklines=true
}
\newenvironment{hypothesis}{
  	\itshape
  	\leftskip=\parindent \rightskip=\parindent
  	\noindent\ignorespaces}

\fancypagestyle{plain}{
  \renewcommand{\headrulewidth}{0pt}
  \fancyhf{}
}	


\begin{document}
\title{Organizational Learning and Human Capital: Similarities, Tensions and a Research Proposal}
\date{}
\maketitle

\begin{abstract} 
\normalsize 
I review the organizational learning and human capital literatures to understand the antecedents, assumptions and mechanisms behind each literature\textquotesingle s explanation of the source of competitive advantages in firms. I focus on the tension between the levels of analysis in the two literatures to propose a study to understand how the nature of human capital of the firm (general vs firm specific human capital) and the nature of organizational learning  in the firm (explorative or exploitative) tradeoff in determining strategic choices and firm profits. I contribute back to both literatures by nuancing each literature\textquotesingle s explanation of the source of competitive advantage in firms.
\end{abstract}


{\textbf{Keywords:} \\\indent Exploitation, Exploration, Firm Specific Human Capital, General Human Capital,  Organizational Learning}

\newpage
\pagestyle{fancy}
\fancyhf{}
\lhead{Organizational Learning and Human Capital}
\rhead{\thepage}

\begin{center}
\textbf{Organizational Learning and Human Capital: Similarities, Tensions \& a Research Proposal}
\end{center}

Building on the Resource Based View \citep{Barney1991} of the firm, scholars of human capital have suggested that human capital is a source of competitive advantage in firms. Researchers in the organizational learning tradition, build upon evolutionary \citep{Nelson1982} and behavioral theories \citep{Cyert1963, March1958} to suggest that the collective capability of firms are a source of competitive advantage in firms. Both traditions are influential and have provided rich and insightful understanding of the sources of firm heterogeneity. The central tension has been whether explanations of individual and collective or societal outcomes should focus on the individual or societal and cultural level \citep{Udehn2001}. I discuss each of the traditions briefly in the following sections.

\section{Review}
\subsection{Organizational Learning}
Organizational learning is the process by which an organization acquires knowledge as a result of its experiences. It is possible for an organization to acquire such knowledge either directly ? through its own activities ? or indirectly ? through observing the actions of
other units. Organizational learning is a change in the organization?s knowledge that occurs as a function of experience  Organizations can learn directly from their own experience or indirectly from the experience of other units \citep{Levitt1988} the knowledge the organization learns from experience can manifest itself in changes in cognitions of organization members, in the organization?s routines or in characteristics of its performance such as speed or accuracy. The knowledge can be embedded in a variety of repositories (Walsh and Ungson 1991) including tools (Kogut and Zander 1992), routines, social networks and transactive memory systems. Once
the knowledge is embedded in a supra-individual repository, the knowledge would evidence some persistence, even if turnover of individuals occurred.
The literature has long held that routines are ?repositories and carriers of knowledge? (Hodgson, 2008: 25). We show that organizational memory is a natural outcome of the process through which routines emerge, and knowledge is thus not only in the minds of (and embodied in the habits of) individuals, but also in the connections between individuals. Consider employee turnover. We  nd that the exit of one individual does not undermine the performance of the routine. Even if the departing individual does not explicitly transfer his or her knowledge to the replacement person, the routinized behavior of the remaining individuals will lead him or her to select a task approach that resembles his or her predecessor. Thus, the knowledge embodied in the connections between individuals has the properties of tacit knowledge. Once formed, such knowledge is not subject to expropriation by individuals ? it is inher- ently the property of the organization.

\subsection{Human Capital}
Building upon \cite{Becker1962} seminal work, strategists assume that human resources are heterogeneous and endowed with different types and degrees of human capital. Human capital captures stocks of education, information and health that have been accumulated both on and off the job \citep{Becker1962}. Given that human resources are not randomly distributed across firms, the optimal matching of firms, workers and jobs is crucial in achieving a competitive advantage.Some authors have stressed that individuals possess portfolios of both general and specific human capital, and that the portfolio and its use by the firm determine its value \citep{Campbell2012}. Others have argued that human capital is never specific in the sense that no other firm can use it. \cite{Lazear2009} has suggested that all human capital, is general. Although one should not discount the important role of recruitment and formal education and training for the acquisition of new knowledge and skills, the most
important way in which a firm develops its human capital is as a by-product of the work people do, in the normal course of their day-to-day activities (Lucas, 1993). The difficulties with transferring firm-specific human capital to different firms provide theoretical isolating mechanisms that allow the firm to capture quasi-rents \citep{Barney1991, Campbell2012}. The collective wisdom of strategic human capital
scholars, then, is that firm-specific human capital is a critically important source of sustainable competitive advantage.

One recent and major break from traditional HC research is the focus on the strategic importance of HC \citep{Wright2014}. This focus has stemmed primarily from the combined movements of HR into strategic HR (e.g., Becker \& Gerhart, 1996)
and strategy into examining the role of people in the organization \citep{Hitt2001b}. This convergence between strategy and strategic HR has further accelerated with the increasing focus on microfoundations in strategy \citep{Nyberg2014}. The common theme surrounding this convergence is the focus on how the human element can benefit strategic outcomes.
This change in focus from thinking about how individuals acquire more capital in the marketplace to thinking about how organizations use that capital necessitates rethinking the HC construct and more specifically labeling the different constructs that make up this growing strategic human capital resource domain \citep{Wright2014}. Specifically, as researchers move away from examining how individuals
develop in the marketplace (e.g., greater education to secure greater work outcomes) toward examining how employees (either at the individual level, such as a CEO, or at the aggregate level, such as a work unit) contribute to unit-level outcomes, they must recognize that this is no longer HC as it was originally conceptualized \citep{Ployhart2014}. Further, this shift is necessary as we think in
terms of the relationship between employees and unit outcomes, or as \cite{Molloy2015}, ?[V]alue is created only when the use of human capital increases a firm?s revenues and/or decreases a firm?s costs.? That is, as \cite{Ployhart2015} notes, the primary
interest of much of the micro research involving HC continues to focus on the individual (i.e., HC) without much regard for the strategic implications of that individual. This contrasts starkly with macro research (even when moving toward microfoundations) that implicitly thinks about HCR even while using HC language. To help reconcile these challenges, \cite{Ployhart2014} declared HC to be dead and introduced the term HCR.

\subsection{Similarities}
foundational and highly cited macro concepts such as firm-level absorptive capacity \citep{Cohen1990} can be directly traced to equivalent, individual-level concepts in psychology.  These individual-level concepts were applied directly to the firm and coined as ?absorptive capacity?. While the concept of absorptive capacity, of course, is important, the paper itself did not explicitly theorize how the concept might need to change and evolve when applied to the organization as a unitary actor (though, subsequent efforts have been made in this direction). Other examples of the one-to-one application of concepts from micro to macro include cognition and learning by association and analogy (e.g. \cite{Gavetti2012b, Gavetti2005b}).

\subsection{Tensions}
Some of the most highly cited and seminal work in strategic management have specifically argued that individual-level considerations simply are not relevant for strategy and firm-level outcomes \citep{Henderson1994, Kogut1992, Kogut1996, Nahapiet1998, Spender1996}. This literature places an emphasis on macro factors such as firm-level knowledge and competencies, social capital, networks, and other collective constructs. \citep{Henderson1994} assume in their highly cited empirical analysis of organizational competencies that individuals are homogeneous, and thus they ascribe performance variance to collective-level routines and practices.  their macro approach, and assumptions about the homogeneity of human capital, by arguing that stars cannot be a source of sustainable advantage as they appropriate their marginal rents. In other words, the reason that some have focused directly on collective capability is that information about the capability of particular individuals (if markets are relatively efficient), such as stars, is likely to be widely available and thus these individuals are perhaps able to appropriate any rents associated with their abilities. HRM?performance relationship is subject to endogeneity concerns. Endogeneity may stem from various sources such as omitted variables, simultaneity or non-random measurement error.

\subsection{A Synthesis}
One way to look at the extreme positions between the two schools is to see it as an opportunity to craft research at the intersection of the two traditions. \cite{Coff2011} explored human capital-based competitive advantage and, while their focus was clearly within the firm, they did note that when causal ambiguity is derived from tacit knowledge it creates problems of imitation for both people within the firm and for competitors.
Recent work \citep{Coff2011, Ployhart2011} has begun to address some of the linkages between micro and macro areas in the domain of human capital and strategy. \cite{Burgelman1991} the genius of surviving organizations lies in their ability to benefit from both winning and losing individual strategic initiatives through their capacity for learning.


At some level, it would seem that Human Capital and Organizational Learning are but part of the same system because it is the Human Capital within firms that . Organizational Learning literature stresses routines and history, which are by themselves anti-thetical to the notion of Human Capital as a driver of competitive advantage. We therefore have this situation



Some of these salient similarities and differences to Table ~\ref{qe2a}

\begin{table}
\begin{centering}
\caption {Comparison of Organizational Learning and Human Capital Literatures}
\label{qe2a}
{\tabulinesep=1.4mm
\begin{tabu}{|p{0.25\textwidth}|p{0.35\textwidth}|p{0.35\textwidth}|}
\hline
& \textbf{Organizational Learning} & \textbf{Human Capital}  \\
\hline   
\textbf{Theoretical Tradition}& Dynamic Capabilities, Evolutionary Perspectives & Resource Based View  \\\hline
\textbf{Unit of Focus}& Teams and Organizations & Individuals  \\\hline
\textbf{Recommendations for Competitive Advantage}& Theta & Beta  \\\hline
\textbf{Primary Tensions}& Theta & Beta  \\\hline
\textbf{Tradition}& Theta & Beta  \\\hline
\textbf{Tradition}& Theta & Beta  \\\hline
\textbf{Tradition}& Theta & Beta  \\\hline
\textbf{Tradition}& Theta & Beta  \\\hline 
\end{tabu}}
\end{centering}
\end{table} 

There is an opportunity to bridge these two literatures and understand with greater nuance the two. 

\section{Research Proposal}
The tension that I wish to address in this study is the one between balancing exploration-exploitation and the one between general Human Capital (HC) and Firm Specific Human Capital (FSHC). Conflict - levels of analysis. OL literature assumes all human capital to be homogenous. If one were to accept that HC were heterogenous, and that Organizational Learning is driven by routines, then ...

Plot the 2x2 and suggest that there are costs and sources of profits in each of the quadrants. Discuss it, and suggest that the net effect is unclear. Talk about non-competes, reputations for enforcement etc - if they are in favor of employees then they are more likely to extract rents. Alternatively.

Product market - causal ambiguity of some sort.

\subsection{Research Question}


\section{Theory}
\subsection{On the topic of the general hypotheses}
From an organizational learning perspective, recipient firms are likely to be concerned with two issues related to ability to learn: first, whether they can effectively understand and apply the spilled over knowledge; and second, the additional complementary opportunities for learning from the originator. If the knowledge is in one of the areas of specialization of the recipient firm, the recipient?s absorptive capacity \citep{Cohen1990} for the spilled over knowledge as well as for complementary knowledge is likely to be high. The absorptive capacity of the firm for external technological knowledge is dependent to a significant extent on the degree of its knowledge in a particular technological field \citep{Schoenmakers2006}. Further, the development and accumulation of tacit knowledge \citep{Polanyi1966} related to the technology is also dependent on specialization \citep{Enright1991}. Thus a recipient with greater specialization will possess well-developed internal mechanisms for understanding and exploiting spillover knowledge.

Extensive attention has focused on understanding the relative advantage of two different modes of organizational learning, exploration and exploitation \cite{March1991a}. Exploration includes the search for new possibilities, experimentation and risk-taking. For instance, an electronic firm positioning itself as an innovator may want to explicitly set up exploratory learning processes to collect novel ideas from consumer focus groups on a regular basis. Conversely, exploitation involves efficiency and refinement. For example, a firm focusing on exploitation might set up processes to identify and correct the causes of production defects.

Organizational learning is viewed as routine-based, history-dependent, and target-oriented. Organizations are seen as learning by encoding inferences from history into routines that guide behavior.

The FSHC paradox is that firms need workers to invest in firm-specific human capital, but workers don?t always want to make these investments \citep{Wang2006}. Workers have a choice about whether to invest in general or firm-specific human capital. If they invest in general human capital they can always take their human capital to another employer and get paid appropriately for their skills. If
they invest in firm-specific human capital, they can extract the value from those skills only in their current firm. If the firm goes out of business, if the employees need to change jobs for personal reasons, or if the firm decides to act opportunistically and not compensate the employees for these skills, the employees cannot achieve any reasonable return on their investments. Generally, risk-averse employees are more likely to invest in general human capital, because there is less risk of losing the value of the human capital investment (see \cite{Wang2006} for a more detailed review of the paradox). The focus on this paradox has led some scholars to claim
that we have a global underinvestment in firmspecific human capital that may be holding back our economic growth and development.

The resources and (dynamic) capabilities perspective?which we will refer to as the capabilities approach?maintains that firms possessing, creating, and adapting resources and capabilities can capture and sustain competitive advantage \citep{Barney1991, 
Penrose1959, Teece1997}. The governance approach maintains that higher economic performance can be achieved by investing in complementary and cospecialized assets \citep{Helfat1997, Teece1986} and by governing them in an economizing way \citep{Oxley1997, Williamson1985}.

 Figure ~\ref{fig:3a} lays out the average score charts for four agent-field combinations while enforcing the field to start in Right of Center (this is the same as saying $p_{0,F}^0 = 0.75$). 
\subsubsection{Leading into H1a}
We do so since the scale is symmetric across the Center (C), any initial mapping 

\begin{hypothesis}
{Hypothesis 1a: When the institutional field is open to influence, slow learning adversarial agents will raise overall performance higher than slow learning agents with a neutral orientation\\}
\end{hypothesis}

Reference to Table ~\ref{qe2b}

\begin{table}[h]
\renewcommand\arraystretch{2.5} % provide a bit taller rows
\centering
\caption{Firm Profit Effects from Organizational Learning and Human Capital Characteristics}
\label{qe2b}
\begin{tabularx}{\textwidth}{@{} c c |c |c| } % use 'Y' for first column
&\multicolumn{1}{c}{}&\multicolumn{2}{c}{\textbf{Organizational Learning}}\\[-2ex]
&\multicolumn{1}{c}{}
&\multicolumn{1}{c}{\textbf{Explorative}}&\multicolumn{1}{c}{\textbf{Exploitative}}\\\cline{3-4}
\multirow{4}{*}{\rotatebox{90}{\textbf{Human Capital (HC)}}}
&\multirow{2}{*}{\textbf{General HC}}&\textbf{$\uparrow$ Schumpeterian Rent}&\textbf{$\uparrow$ Ricardian Rent}\\
&&\textbf{$\downarrow$ Rents to human capital}&\textbf{$\downarrow$ Rents to human capital}\\\cline{3-4}
&\multirow{2}{*}{\textbf{Firm Specific HC}}&\textbf{$\uparrow$ Schumpeterian Rent}&\textbf{$\uparrow$ Ricardian Rent}\\
&&\textbf{$\uparrow$ No rents to human capital}&\textbf{$\uparrow$ No rents to human capital}\\\cline{3-4}
\end{tabularx}
\end{table}

\subsubsection{Leading into H2a}
This trend is confirmed further in Figure ~\ref{fig:3a} where the learning rates of agents are increased even further to \textquotesingle Fast\textquotesingle .

\begin{hypothesis}
{Hypothesis 2a: For the same initial outcome preferences,  the overall performance score varies curvilinearly with difference in the rates of learning of the agent and the institutional field\\}
\end{hypothesis}

\subsection{Data and Method}

What is the sample. Why is it chosen. How is it going to contribute. Semiconductor industry. Classify. Each of the boxes. Technology change and framework of labor laws as moderating variables. Argue that one is more than the other in the appropriate and therefore leading into the hypothesis. For every year 0 1. Exploration Exploitation - patents filed in that year. Have they been broad or narrow.
Legal framework - change when there is a regulatory change. Like when Michigan.

\subsubsection{Dependent Variable}
Firm outcome - Tobin's Q. Some form of market measure.

\subsubsection{Independent Variables}

\subsubsection{Moderating Variables}

\section{Contributions}
What do you contribute to each literature. Basically nuance. Adding framework of labor laws, routines, exploration-exploitation. For Organizational Learning, the role of FSHC and GHC.

\section{Limitations}
Suggest how this study may help inform the literatures that it is drawing from, and the interesting research avenues it will open up. Discuss level of generalizability.

\section{Summary}
Recap and motivate interest in framework, in theoretical value as well as in the particular empirical setting.


\begin{singlespace}
\renewcommand{\refname}{REFERENCES}
\bibliography{/Users/aiyenggar/code/bibliography/aiyenggar} 
\bibliographystyle{ai-amjlike}
\end{singlespace}

\end{document}
