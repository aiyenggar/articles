%1521001##1##.tex
\documentclass[12pt,letterpaper]{article}
\usepackage{mathptmx}
\usepackage[margin=1in]{geometry}

\usepackage{setspace}
\singlespacing
  
\usepackage{amssymb,latexsym}
\usepackage[round,sort]{natbib}
\usepackage{fancyhdr}
\usepackage{lastpage}
\usepackage{graphicx,multirow}
\graphicspath{ {qe2/} }

% Bold Table and Figure captions
\usepackage{caption}
\captionsetup{figurename=FIGURE}
\captionsetup{tablename=TABLE}
\captionsetup[figure]{labelfont=bf}
\captionsetup[table]{labelfont=bf}
  
% Turns off all section numbering
\setcounter{secnumdepth}{0} 

  % Places all tables at end of document and creates AOM-style table-here placeholders
  \usepackage[nolists]{endfloat} % Places all figures and charts at end of manuscript and adds 'insert table x about here' lines.
  \renewcommand{\figureplace}{
    \begin{center}
    \begin{singlespace}
    ------------------------------------\\
    Insert \figurename \ \thepostfig\ about here.\\
    ------------------------------------
    \end{singlespace}
    \end{center}}
  \renewcommand{\tableplace}{%
    \begin{center}
    \begin{singlespace}
    ------------------------------------\\
    Insert \tablename \ \theposttbl\ about here.\\
    ------------------------------------
    \end{singlespace}
    \end{center}}

  \usepackage{titlesec}
   \titleformat{\title}
       {\filcenter\normalfont\bfseries\uppercase}{\thetitle}{1em}{}
  \titleformat{\section}
    {\filcenter\normalfont\bfseries\uppercase}{\thesection}{1em}{}
  \titleformat{\subsection}
    {\normalfont\bfseries}{\thesubsection}{1em}{}
  \titleformat{\subsubsection}[runin]
   {\normalfont\bfseries\slshape}{\thesubsubsection}{1em}{\hspace*{\parindent}}
       
\usepackage{tabu}
\usepackage{textcomp}
\usepackage{listings}
\usepackage{hyperref}
\usepackage{verbatim}
\usepackage{tabu}
\hypersetup{
    colorlinks=true,
    linkcolor=blue,
    filecolor=cyan,      
    urlcolor=cyan,
    citecolor=blue,
}

\usepackage{etoolbox}

\makeatletter

% Patch case where name and year are separated by aysep
\patchcmd{\NAT@citex}
  {\@citea\NAT@hyper@{%
     \NAT@nmfmt{\NAT@nm}%
     \hyper@natlinkbreak{\NAT@aysep\NAT@spacechar}{\@citeb\@extra@b@citeb}%
     \NAT@date}}
  {\@citea\NAT@nmfmt{\NAT@nm}%
   \NAT@aysep\NAT@spacechar\NAT@hyper@{\NAT@date}}{}{}

% Patch case where name and year are separated by opening bracket
\patchcmd{\NAT@citex}
  {\@citea\NAT@hyper@{%
     \NAT@nmfmt{\NAT@nm}%
     \hyper@natlinkbreak{\NAT@spacechar\NAT@@open\if*#1*\else#1\NAT@spacechar\fi}%
       {\@citeb\@extra@b@citeb}%
     \NAT@date}}
  {\@citea\NAT@nmfmt{\NAT@nm}%
   \NAT@spacechar\NAT@@open\if*#1*\else#1\NAT@spacechar\fi\NAT@hyper@{\NAT@date}}
  {}{}

\lstset{
basicstyle=\ttfamily,
columns=flexible,
breaklines=true
}
\newenvironment{hypothesis}{
  	\itshape
  	\leftskip=\parindent \rightskip=\parindent
  	\noindent\ignorespaces}

\fancypagestyle{plain}{
  \renewcommand{\headrulewidth}{0pt}
  \fancyhf{}
}	


\begin{document}
\title{Organizational Learning and Human Capital: Similarities and Tensions}
\date{}
\maketitle

\begin{abstract} 
\normalsize 
We apply a formal model to understand the effects of the relative learning rates of embedded agents and the institutional field on organizational outcomes. 
\end{abstract}


{\textbf{Keywords:} \\\indent Embedded Agency}

\newpage
\pagestyle{fancy}
\fancyhf{}
\lhead{Organizational Learning and Human Capital}
\rhead{\thepage}

\begin{center}
\textbf{Title: Sub-title}
\end{center}


Today?s organizations increasingly rely on intangible assets such as human capital to gain a competitive advantage (e.g., Barney, 1991)
some of the linkages between micro and macro areas are already occurring, for example in the domain of human capital and strategy (e.g. Coff \& Kryscynski, 2011; Ployhart \& Moliterno, 2011. The idea of ?microfoundations? can be traced to historical tensions between micro and macro disciplines in the social sciences. The central tension has been whether explanations of individual and collective or societal outcomes should focus on the individual or societal and cultural level (Udehn, 2001)
Similar intuition has been echoed and further reinforced in some of the most highly cited and seminal work in strategic management over the past two decades, specifically arguing that individual-level considerations simply are not relevant for strategy and firm-level outcomes (see Henderson \& Cock- burn, 1994; Kogut \& Zander, 1992, 1996; Nahapiet \& Ghoshal, 1998; Spender, 1996). This literature places an emphasis on macro factors such as firm-level knowledge and competencies, social capital, networks, and other collective constructs. Henderson and Cockburn (1994) assume in their highly cited empirical analy- sis of organizational competencies that individuals are homogeneous, and thus they ascribe performance variance to collective-level routines and practices. The microfoundations literature and movement, then, can be seen as a reaction to an over-emphasis on collective factors, as well as the seeming disregard for individual-level and social interactional considerations in explaining organiz- ational outcomes.


To provide another specific example, foundational and highly cited macro concepts such as firm-level ?absorptive capacity? (Cohen \& Levinthal, 1990) can be directly traced to equivalent, individual-level concepts in psychology. Specifically, the oldest citations in Cohen and Levinthal?s influential article are to behavioral psychology and learning theory related to individual knowl- edge acquisition based on experience, stimulus, and repetition (e.g. Bower \& Hilgard, 1981; Ellis, 1965; Estes, 1970; Harlow, 1949). These individual-level concepts were applied directly to the firm and coined as ?absorptive capacity?. While the concept of absorptive capacity, of course, is important, the paper itself did not explicitly theorize how the concept might need to change and evolve when applied to the organization as a unitary actor (though, subsequent efforts have been made in this direction). Other examples of the one-to-one application of concepts from micro to macro include cognition and learning by association and analogy (e.g. Gavetti, 2012; Gavetti, Levinthal, \& Rivkin, 2005).



From an organizational learning perspective, recipient firms are likely to be concerned with two issues related to ability to learn: first, whether they can effectively understand and apply the spilled over knowledge; and second, the additional complementary opportunities for learning from the originator. If the knowledge is in one of the areas of specialization of the recipient firm, the recipient?s absorptive capacity \citep{Cohen1990} for the spilled over knowledge as well as for complementary knowledge is likely to be high. The absorptive capacity of the firm for external technological knowledge is dependent to a significant extent on the degree of its knowledge in a particular technological field (Schoenmakers and Duysters, 2006). Further, the develop- ment and accumulation of tacit knowledge (Polanyi, 1966) related to the technology is also dependent on specialization (Enright, 1991). Thus a recipient with greater speciali- zation will possess well-developed internal mechanisms for understanding and exploiting spillover knowledge.
 their macro approach, and assumptions about the homogeneity of human capital, by arguing that stars cannot be a source of sustainable advantage as they appropriate their marginal rents. In other words, the reason that some have focused directly on collective capability is that information about the capability of particu- lar individuals (if markets are relatively efficient), such as stars, is likely to be widely available and thus these individuals are perhaps able to appro- priate any rents associated with their abilities.
Our theory also links work in the knowledge- based view (e.g., \cite{Grant1996b, Kogut1992} with work on strategic human capital (e.g., Campbell, Coff, and Kryscynski, 2012; Coff, 1997). The literature has long held that routines are ?repos- itories and carriers of knowledge? (Hodgson, 2008: 25). We show that organizational memory is a nat- ural outcome of the process through which routines emerge, and knowledge is thus not only in the minds of (and embodied in the habits of) individuals, but also in the connections between individuals. Con- sider employee turnover. We  nd that the exit of one individual does not undermine the performance of the routine. Even if the departing individual does not explicitly transfer his or her knowledge to the replacement person, the routinized behavior of the remaining individuals will lead him or her to select a task approach that resembles his or her predeces- sor. Thus, the knowledge embodied in the connec- tions between individuals has the properties of tacit knowledge. Once formed, such knowledge is not subject to expropriation by individuals ? it is inher- ently the property of the organization.

Coff and Kryscynski (2011) explored human capital-based competitive advan- tage and, while their focus was clearly within the firm, they did note that when causal ambiguity is derived from tacit knowledge it creates problems of imitation for both people within the firm and for competitors.
Characteristics, Base discipline, level of analysis, key assumptions

\section{Organizational Learning}
Organizational learning is the process
by which an organization acquires knowledge as a
result of its experiences. It is possible for an organization
to acquire such knowledge either
directly ? through its own activities ? or
indirectly ? through observing the actions of
other units.
Organizational learning is a change in the organization?s
knowledge that occurs as a function of
experience (Fiol and Lyles 1985; Easterby-Smith
et al. 2000). Organizations can learn directly from
their own experience or indirectly from the experience
of other units (Levitt and March 1988).
he knowledge the organization
learns from experience can manifest itself in
changes in cognitions of organization members,
in the organization?s routines or in characteristics
of its performance such as speed or accuracy. The knowledge
can be embedded in a variety of repositories
(Walsh and Ungson 1991) including tools
(Kogut and Zander 1992), routines, social networks
and transactive memory systems. Once
the knowledge is embedded in a supra-individual
repository, the knowledge would evidence some
persistence, even if turnover of individuals
occurred.

Extensive attention has focused on understanding
the relative advantage of two different modes of
organizational learning, exploration and exploitation
(March 1991). Exploration includes the
search for new possibilities, experimentation and
risk-taking. For instance, an electronic firm positioning
itself as an innovator may want to explicitly
set up exploratory learning processes to
collect novel ideas from consumer focus groups
on a regular basis. Conversely, exploitation
involves efficiency and refinement. For example,
a firm focusing on exploitation might set up processes
to identify and correct the causes of production
defects.

Organizational learning is viewed as routine-based, history-dependent, and target-oriented. Organizations are seen as learning by encoding inferences from history into routines that guide behavior.

\section{Human Capital}
Building upon
Becker?s (1962) seminal work, strategists assume
that human resources are heterogeneous and
endowed with different types and degrees of
human capital. Human capital captures stocks
of education, information and health that have
been accumulated both on and off the job
(Becker 1962). Given that human resources are
not randomly distributed across firms, the optimal
matching of firms, workers and jobs is crucial in
achieving a competitive advantage.

Some authors have stressed that individuals possess
portfolios of both general and specific human
capital, and that the portfolio and its use by the
firm determine its value (Campbell et al. 2012). Others have argued that human capital is never
specific in the sense that no other firm can use
it. Lazear (2009) has suggested that all human
capital, is general.



not much about how
a company can redirect the actions and behaviors of
its critical human capital to deliver on the changing
demands of the external marketplace. This is just
one example of demanding, practical human capital
problems that managers deal with every day but
that our academic literature seems to ignore?yet
more evidence of the commonly discussed divide
between theory and practice (Bartunek & Rynes,
2014). This emerging domain seems to be
bringing together scholars and practitioners from
two different traditions: strategy and human resourcemanagement
(Wright, Coff, &Moliterno, 2014). Based on logic from the resource-based view
(RBV), firm-specific human capital should be a particularly
important strategic resource because it is
uniquely valuable in the focal firm. The difficulties
with transferring firm-specific human capital to
different firms provide theoretical isolating mechanisms
that allow the firm to capture quasi-rents
(Barney, 1991; Campbell et al., 2012; Rumelt, 1984).
The collective wisdom of strategic human capital
scholars, then, is that firm-specific human capital
is a critically important source of sustainable competitive
advantage.

The FSHC paradox is that firms need workers
to invest in firm-specific human capital, but workers
don?t always want to make these investments
(Wang & Barney, 2006).
Workers have a choice about
whether to invest in general or firm-specific human
capital. If they invest in general human capital they
can always take their human capital to another employer
and get paid appropriately for their skills. If
they invest in firm-specific human capital, they can
extract the value from those skills only in their current
firm. If the firm goes out of business, if the employees
need to change jobs for personal reasons, or if
the firm decides to act opportunistically and not
compensate the employees for these skills, the employees
cannot achieve any reasonable return on
their investments. Generally, risk-averse employees
are more likely to invest in general human capital,
because there is less risk of losing the value of the
human capital investment (see Wang and Barney
2006 for a more detailed review of the paradox). The
focus on this paradox has led some scholars to claim
that we have a global underinvestment in firmspecific
human capital that may be holding back our
economic growth and development.
Assumption 1: Firm-specific human capital is
important for a firm?s competitive performance

Lazear (2009) suggested that firm-specific human
capital is not particularly important in practice. He
argued that some of the knowledge that is truly firm
specific, such as finding the bathrooms, is important
for daily functioning but not particularly relevant
for competitive performance. Thus, he proposed
that different combinations of general skills may be
more practically relevant than trying to search for
unique and difficult-to-transfer skills.

Becker, G. (1964). Human capital: A theoretical and empirical
analysis, with special reference to education.
Chicago: University of Chicago Press.

The resources and (dynamic) capabilities
perspective?which we will refer to as the capabilities
approach?maintains that firms possessing, creating,
and adapting resources and capabilities can capture
and sustain competitive advantage (Barney, 1991;
Penrose, 1959; Teece, Pisano, & Shuen, 1997). The governance
approach maintains that higher economic
performance can be achieved by investing in
complementary and cospecialized assets (Helfat,
1997; Teece, 1986) and by governing them in an economizing way (Oxley, 1997; Williamson, 1985).

One recent and major break from traditional HC
research is the focus on the strategic importance of
HC (Wright, Coff, & Moliterno, 2014). This focus has
stemmed primarily from the combined movements
of HR into strategic HR (e.g., Becker & Gerhart, 1996)
and strategy into examining the role of people in the
organization (e.g., Hitt, Bierman, Shimizu, & Kochhar,
2001). This convergence between strategy and strategic
HR has further accelerated with the increasing
focus on microfoundations in strategy (Nyberg et al.,
2014). The common theme surrounding this convergence
is the focus on how the human element can
benefit strategic outcomes.
This change in focus from thinking about how
individuals acquire more capital in the marketplace
to thinking about how organizations use that capital
necessitates rethinking the HC construct and more
specifically labeling the different constructs that
make up this growing strategic human capital resource
domain (Wright et al., 2014). Specifically, as
researchers move away from examining how individuals
develop in the marketplace (e.g., greater
education to secure greater work outcomes) toward
examining how employees (either at the individual
level, such as a CEO, or at the aggregate level, such
as a work unit) contribute to unit-level outcomes,
they must recognize that this is no longer HC as it
was originally conceptualized (Ployhart et al.,
2014). Further, this shift is necessary as we think in
terms of the relationship between employees and
unit outcomes, or as Molloy and Barney state (in this
issue), ?[V]alue is created only when the use of
human capital increases a firm?s revenues and/or
decreases a firm?s costs.?
This fundamental change in the construct has
evolved, albeit without name. For instance, strategy
researchers have long used this higher-order concept
of HC to examine the impact of aggregate levels
of knowledge, skills, abilities, and other characteristics
(KSAOs) on organizational outcomes, while
still referring to it as HC (Nyberg et al., 2014). This
creates challenges when conversing across levels
and disciplines about the role of people in helping
to achieve organizational outcomes because traditional
HC scholars and those trying to apply these
lessons to unit-level outcomes often approach the
issues with different conceptions of the construct.
That is, as Ployhart (this issue) notes, the primary
interest of much of the micro research involving HC
continues to focus on the individual (i.e., HC)
without much regard for the strategic implications
of that individual. This contrasts starkly with macro
research (even when moving toward microfoundations)
that implicitly thinks about HCR even
while using HC language.
To help reconcile these challenges, Ployhart and
colleagues (2014) declared HC to be dead and introduced
the term HCR. Naturally, HC remains
a vital construct and one that continues to deserve
attention, but it should be limited to a focus on the
KSAOs at the individual level of analysis. Hence,
we too advocate for using the term HCR when
thinking about unit-level outcomes. By using HCR,
a construct that is defined as unit relevant, researchers
would begin projects with a common
language and set of assumptions?building blocks
that are necessary for developing strong foundations
that can facilitate scientific advancement (Schwab,
1980).
Social

\subsection{Methodological Issues}
Gerhart (2007) has summarized the empirical
challenges of HRM research. Weller and Gerhart
(2012) provide an overview for the international
context. In both cases, two issues deserve attention:
first, because HRM is multi-level, many empirical problems centre on some sort of nested
data. Examples include individuals nested in firms
nested in industries. The problem is that clustered
data are not independent, and thus assumptions of
the standard regression model are violated. Violations
may result in increased type 1 error rates
because the degrees of freedom differ within the
data and standard errors may be biased downwards
for higher-level variables.
Second, it is difficult to establish the causal link
between HRM and firm performance. Since most
field data are not randomly drawn from the population
(i.e., ?treatments? like HRM practices are
not randomly assigned), the HRM?performance
relationship is subject to endogeneity concerns.
Endogeneity may stem from various sources
such as omitted variables, simultaneity or
non-random measurement error.

\section{Theory}
\subsection{On the topic of the general hypotheses}
 Figure ~\ref{fig:3a} lays out the average score charts for four agent-field combinations while enforcing the field to start in Right of Center (this is the same as saying $p_{0,F}^0 = 0.75$). 
\subsubsection{Leading into H1a}
We do so since the scale is symmetric across the Center (C), any initial mapping 

\begin{hypothesis}
{Hypothesis 1a: When the institutional field is open to influence, slow learning adversarial agents will raise overall performance higher than slow learning agents with a neutral orientation\\}
\end{hypothesis}

\subsubsection{Leading into H2a}
This trend is confirmed further in Figure ~\ref{fig:3a} where the learning rates of agents are increased even further to \textquotesingle Fast\textquotesingle .

\begin{hypothesis}
{Hypothesis 2a: For the same initial outcome preferences,  the overall performance score varies curvilinearly with difference in the rates of learning of the agent and the institutional field\\}
\end{hypothesis}

\section{Limitations and Future Work}



\section{Conclusion}

\begin{singlespace}
\renewcommand{\refname}{REFERENCES}
\bibliography{/Users/aiyenggar/code/bibliography/aiyenggar} 
\bibliographystyle{ai-amjlike}
\end{singlespace}

\end{document}
