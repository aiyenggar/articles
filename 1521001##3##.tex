%1521001##1##.tex
\documentclass[12pt,letterpaper]{article}
\usepackage{mathptmx}
\usepackage[margin=1in]{geometry}

\usepackage{setspace}
\singlespacing
  
\usepackage{amssymb,latexsym}
\usepackage[round,sort]{natbib}
\usepackage{fancyhdr}
\usepackage{lastpage}
\usepackage{graphicx,multirow}
\graphicspath{ {qe3/} }

% Bold Table and Figure captions
\usepackage{caption}
\captionsetup{figurename=FIGURE}
\captionsetup{tablename=TABLE}
\captionsetup[figure]{labelfont=bf}
\captionsetup[table]{labelfont=bf}
  
% Turns off all section numbering
\setcounter{secnumdepth}{0} 

  % Places all tables at end of document and creates AOM-style table-here placeholders
  \usepackage[nolists]{endfloat} % Places all figures and charts at end of manuscript and adds 'insert table x about here' lines.
  \renewcommand{\figureplace}{
    \begin{center}
    \begin{singlespace}
    ------------------------------------\\
    Insert \figurename \ \thepostfig\ about here.\\
    ------------------------------------
    \end{singlespace}
    \end{center}}
  \renewcommand{\tableplace}{%
    \begin{center}
    \begin{singlespace}
    ------------------------------------\\
    Insert \tablename \ \theposttbl\ about here.\\
    ------------------------------------
    \end{singlespace}
    \end{center}}

  \usepackage{titlesec}
   \titleformat{\title}
       {\filcenter\normalfont\bfseries\uppercase}{\thetitle}{1em}{}
  \titleformat{\section}
    {\filcenter\normalfont\bfseries\uppercase}{\thesection}{1em}{}
  \titleformat{\subsection}
    {\normalfont\bfseries}{\thesubsection}{1em}{}
  \titleformat{\subsubsection}[runin]
   {\normalfont\bfseries\slshape}{\thesubsubsection}{1em}{\hspace*{\parindent}}
       
\usepackage{tabu}
\usepackage{textcomp}
\usepackage{listings}
\usepackage{hyperref}
\usepackage{verbatim}
\usepackage{tabu}
\hypersetup{
    colorlinks=true,
    linkcolor=blue,
    filecolor=cyan,      
    urlcolor=cyan,
    citecolor=blue,
}

\usepackage{etoolbox}

\makeatletter

% Patch case where name and year are separated by aysep
\patchcmd{\NAT@citex}
  {\@citea\NAT@hyper@{%
     \NAT@nmfmt{\NAT@nm}%
     \hyper@natlinkbreak{\NAT@aysep\NAT@spacechar}{\@citeb\@extra@b@citeb}%
     \NAT@date}}
  {\@citea\NAT@nmfmt{\NAT@nm}%
   \NAT@aysep\NAT@spacechar\NAT@hyper@{\NAT@date}}{}{}

% Patch case where name and year are separated by opening bracket
\patchcmd{\NAT@citex}
  {\@citea\NAT@hyper@{%
     \NAT@nmfmt{\NAT@nm}%
     \hyper@natlinkbreak{\NAT@spacechar\NAT@@open\if*#1*\else#1\NAT@spacechar\fi}%
       {\@citeb\@extra@b@citeb}%
     \NAT@date}}
  {\@citea\NAT@nmfmt{\NAT@nm}%
   \NAT@spacechar\NAT@@open\if*#1*\else#1\NAT@spacechar\fi\NAT@hyper@{\NAT@date}}
  {}{}

\lstset{
basicstyle=\ttfamily,
columns=flexible,
breaklines=true
}
\newenvironment{hypothesis}{
  	\itshape
  	\leftskip=\parindent \rightskip=\parindent
  	\noindent\ignorespaces}

\fancypagestyle{plain}{
  \renewcommand{\headrulewidth}{0pt}
  \fancyhf{}
}	


\begin{document}
\title{Contradictory Predictions from Applying Capabilities and Transaction Cost Economics Approaches to Firms}
\date{}
\maketitle

\begin{abstract} 
\normalsize 
We apply a formal model to understand the effects of the relative learning rates of embedded agents and the institutional field on organizational outcomes. 
\end{abstract}


{\textbf{Keywords:} \\\indent Dynamic Capabilities, Transaction Cost Economics, Theory of the Firm}

\newpage
\pagestyle{fancy}
\fancyhf{}
\lhead{Capabilities vs. TCE Approaches to Firms}
\rhead{\thepage}

\begin{center}
\textbf{The Capabilities Critique of Transaction Cost Economics}
\end{center}

Assuming bounded rationality and opportunism as attributes of contracting actors in firms, Transaction Cost Economics (TCE) uses an economizing argument to predict the nature of governance of transactions between firms \citep{Williamson1985}.  Transaction attributes of asset specificity, frequency and uncertainty are suggested to determine the nature of governance between three options: hierarchy, market or hybrid organization. Where the Transaction Cost Economics argument glosses over managers\textquotesingle ability to learn and adapt to a changing external environment over time, the Dynamic Capabilities (or simply, Capabilities) approach suggests that capabilities ``enable business enterprises to create, deploy, and protect the intangible assets that support superior long-run business performance" \citep{Teece2007}. Scholars have debated the relative merits and demerits of both theories of the firm behavior over the last two decades. In this article, I attempt to organize some of those arguments into a coherent set of dimensions, and then propose a study to expand the envelope of this conversation further.

\section{TCE\textquotesingle s Contribution}
The most durable contribution of the TCE approach has been to highlight the role of co-specialized investments. TCE suggests that firms cannot easily choose a different trading partner without incurring significant adjustment costs due to asset specificity. This framing raises a potential paradox for firms: Can the value created by specific investments be destroyed by ex-post exchange hazards such as haggling? In order to create value, TCE suggests that firms may invest in co-specialized assets for a transaction so as to reap value either by enhancing quality or by reducing production costs associated with the transaction. As will be highlighted in the following arguments, the ex-ante reorganization of governance mode based on potential ex-post governance costs is a valid but probably not an exclusive solution to the paradox.

\section{Arguments against TCE}
In this section, I organize some of the criticisms of TCE being the sole theory of firm decisions. While they apply to firm decisions in general, the criticisms are most salient in four aspects of firm behavior where TCE has had the strongest impact. First, make-or-buy decisions or decisions of vertical integration; Second, decisions about structuring of internal organizations; Third, about alliance and partnership decisions; and Finally, about diversification decisions. To some the assumptions of opportunism presents human manager in poor light \citep{Ghoshal1996}.

\subsection{Transaction Costs are not Exogenous}
\cite{Jacobides2005a} suggest that transaction costs are intertwined with capabilities when
it comes to determining vertical scope. They suggest that capability differences are a necessary
condition for vertical integration and that transaction cost reductions are only incidental. They
suggest that four evolutionary mechanisms shape vertical scope over time: First, selection processes
themselves shape vertical scope. Selection processes are driven by capability differences,
while transaction costs may be seen as static. Second, transaction costs may be endogenously changed as firms reshape their market and
environment in the quest for greater profit and market share.
Third, capability development processes reshape the capability pool in an industry over time
and therefore changing the set of qualified participants in vertical integration.

\subsection{Learning to Govern and Contract}
\cite{Mayer2004} emphasized the learning ability between two cooperating firms to
design contracts and stressed the role of formal contracts as a repository of knowledge within and
among firms. They also argued that the ability to design formal contracts is essentially a critical
ability to deal with the fastchanging environment and that this contracting ability increases with the
familiarity of interacting parties \cite{Mayer2004}. In sum, when facing difficult business
decisions, managers and strategy researchers can benefit from these approaches to understanding
the interaction between capabilities and transaction cost perspectives to make better judgements of
strategic decisions. 
\cite{Mayer2004} find many changes to the structure of the contracts that cannot be fully explained by changes in the assets at risk in the relationship, and evidence that these changes are largely the result of processes in which the firms were learning how to work together, including learning how to contract with each other
 

\subsection{Firms as Human Organizations}
\cite{Ghoshal1996} Organizations are not mere substitutes for structuring efficient transactions when markets fail; they possess unique advantages for governing certain kinds of economic activities through a logic that is very different from that of a market.

\cite{Moran1996} suggest that markets and industrial organizations feature a vastly different set of dynamics. It is suggested that managers cannot run business enterprises or corporations based on transactional cost economics (TCE) because that particular theory is meant to decipher market situations.

\subsection{Limitations of the Price System of the Market}
Dynamic managerial capabilities \cite{Adner2003} - managerial human capital, managerial social capital and managerial cognition. Dynamic capabilities involve higher-level activities that can enable an enterprise to direct its ordinary activities toward high-payoff endeavors. \cite{Teece2014b} suggests that TCE holds ?production? activity invariant even though production costs may depend endogenously on the governance modes, managerial actions, strategy, and structures chosen. Moreover, production technologies and governance modes are not, in the economic perspective, proprietary, but rather available to all firms. Omniscience, not ignorance, is the norm. Managers are essential to resource allocation and
economic activity (and the theory of the firm). The price system has little relevance to the internal allocation of resources within firms because firms, for very good reasons, generally eschew the use of prices as the exclusive tool to determine the internal distribution of resources. The allocation process is instead orchestrated by managers.markets for high-specificity (idiosyncratic) assets generally don?t exist, and if they do exist they are invariably ?thin.? To overcome this problem, managers collect information, sense opportunities, invest in capabilities, innovate, and transform. They become the instruments that help achieve the shrewd allocation of company resources. Markets and internal resource allocation (organized hierarchically inside the
firm) are not only substitutes, as Coase (1937) implicitly claimed; they are also complements. Williamson seems to agree, noting that ?the relation between competence and governance [is] both rival and complementary?more the latter than the former?
(1999, p. 1106). And even if transaction costs were zero and governance problems evaporated, learning, co-creation (Pitelis \& Teece, 2009), and asset/resource orchestration functions would still need to be carried out. The entrepreneurially managed business firm is where this can be done.

\subsection{Internal Organization}
Asset orchestration involves identify- ing the critical assets and investing in them (search/selection), and then developing a governance system along with a means for their effective use identified. The second part of asset orchestration involves the coordination of co-specialized assets and their use in productive ways (configuration/deployment).
\cite{Jacobides2005b} suggests that gains from intrafirm specialization set off a process of intraorganizational partitioning, which simplifies coordination along parts of the value chain. Similarly, latent gains from trade foster interfirm cospecialization, which leads to information standardization.

\subsection{Environmental Change}
\cite{Mayer2006} argue that strong technological capabilities improve a firm's ability to govern transactions, making outsourcing feasible despite certain contractual hazards.


\subsection{Issues from Empirical Studies}
\cite{Eggers2009} measured managerial cognition in terms of CEO attention. They found that this aspect of dynamic managerial capabilities had a positive impact on the ability of incumbent firms to adapt to radical technological change through faster entry into an evolving new market.

\cite{Sirmon2009} found that asset orchestration through resource investment and deployment worked best when managers made congruent rather than independent resource investment and deployment decisions.

\cite{David2004} to suggest that even empirical results have been somewhat mixed, with the strongest results only for asset specificity. 
By drawing on a detailed data set of the luxury automobile market, \cite{Novak2009} show support for complementarity in product development, suggesting therefore that contracting complementarity may be particularly important when coordination is important to achieve but difficult to monitor.
 
\cite{David2004} find considerable disagreement on how to operationalize some of TCE's other central constructs and propositions, and relatively lower levels of support for uncertainty and performance. \cite{David2004} suggest that two problems need to be addressed. First, some of the key propositions, including that relating to uncertainty have been loosely interpreted, and second that some key variables such as performance have received little scrutiny. \cite{David2004} also suggest that tests about the effects of governance forms on performance are likely to suffer form self-selection bias. They also suggest that tests of the relationship between asset specificity and governance form are tests of the largest surviving firms thereby indicating a survivor bias. \cite{David2004} also suggest that TCE has not been applied across time, in longitudinal studies. 

\subsection{Does it have to be an either/or choice?}
\cite{Parmigiani2007} shows that concurrent sourcing is a distinctly different choice that may co-exist along a make/buy continuum.

\cite{Poppo1998}  suggest that a theory of the firm and a theory of boundary choice is likely to be complex, requiring integration of transaction cost, knowledge-based, and measurement reasoning While some have suggested that these two approaches are complementary (Jacobides \& Winter, 2005; Mahoney, 2001; Poppo \& Zenger, 1998), Nickerson, Yen, and Mahoney (2012) documented
that historically the capabilities and governance approaches have not been joined in away that enables scholars and practitioners to coherently design organizations (Simon, 1996). An intuitive understanding of how firms develop and renew firm-level capabilities
requires research attention to both how much firms invest and how effectively these strategic investments are managed and governed (Argyres, 1996;Kor \& Mahoney, 2005; Mayer \& Salomon, 2006).

My conclusion will be that TCE is useful in a limited context within the larger capabilities framework, particularly when transaction costs maybe safely assumed to be exogenous. Maybe over shorter periods of time.

\section{Research Proposal}
In this section, I build upon the arguments made in the previous section for an integrated Capabilities-TCE approach to propose a study to further our understanding. While each of the four domains of vertical integration (make/buy decisions), internal organization, alliances and diversification may provide an interesting setting for such a study, I chose the context of outsourcing contracts in the Information Technology (IT) sector. I propose a longitudinal study of MNC IT contracting over three decades to capture the simultaneity of the Capability and TCE effects in firm contracting decisions.

\subsection{Research Question}
I seek to understand the factors that influence firms\textquotesingle decisions to simultaneously make and buy.

\subsection{Theory}
\cite{Cohen1990} suggested that firms seek to build absorptive capacity so as to evaluate, assimilate and commercialize knowledge that originates outside the firm. Organizational ambidexterity \citep{Raisch2008}. 
My contradictory case can involve a question of time: TCE has no notion of time and therefore may predict a certain thing. However, capabilities are improved and created over time, and the capabilities angle may suggest different. Say something is not asset specific, TCE will suggest no need to vertically integrate. However, for the development of absorptive capacity, or create a market environment within the firm (intrafirm competition), firms may still acquire and integrate. Think of an example: both ways. When TCE says you should vertically integrate and Capabilities says you shouldn't. And vice versa. Discuss the trends underneath. And the cumulative effect. Another theme to bring in is 
What are the two opposite effects? And how can they be cumulated? What variables would be the moderators?

\subsubsection{Leading into H1}
We do so since the scale is symmetric across the Center (C), any initial mapping 
Some sort of competition hypothesis, in addition to absorptive capacity.
\begin{hypothesis}
{Hypothesis 1: Explain why firms will increase make when they are currently buying\\}
\end{hypothesis}

\subsubsection{Leading into H2}
We do so since the scale is symmetric across the Center (C), any initial mapping 

\begin{hypothesis}
{Hypothesis 2: Explain why firms will increasing buying when they are currently also making\\}
\end{hypothesis}

\subsection{Data and Method}
Look at the IT contracting behavior of Fortune 500 firms over 1985 to 2015. Measure if they solely contract ($\geq$ 80\% of budget), solely build  ($\geq$ 80\% of budget), or do both (if both are atleast 20\%)

\section{Limitations}
Suggest how this study may help inform the literatures that it is drawing from, and the interesting research avenues it will open up. Discuss level of generalizability.

\section{Summary}
Recap and motivate interest in framework, in theoretical value as well as in the particular empirical setting.


\begin{comment}
\section{Notes}

\end{comment}

\begin{singlespace}
\renewcommand{\refname}{REFERENCES}
\bibliography{/Users/aiyenggar/code/bibliography/aiyenggar} 
\bibliographystyle{ai-amjlike}
\end{singlespace}

\end{document}
