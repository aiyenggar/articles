%1521001##1##.tex
\documentclass[12pt,letterpaper]{article}
\usepackage{mathptmx}
\usepackage[margin=1in]{geometry}

\usepackage{setspace}
\singlespacing
  
\usepackage{amssymb,latexsym}
\usepackage[round,sort]{natbib}
\usepackage{fancyhdr}
\usepackage{lastpage}
\usepackage{graphicx,multirow}
\graphicspath{ {qe3/} }

% Bold Table and Figure captions
\usepackage{caption}
\captionsetup{figurename=FIGURE}
\captionsetup{tablename=TABLE}
\captionsetup[figure]{labelfont=bf}
\captionsetup[table]{labelfont=bf}
  
% Turns off all section numbering
\setcounter{secnumdepth}{0} 

  % Places all tables at end of document and creates AOM-style table-here placeholders
  \usepackage[nolists]{endfloat} % Places all figures and charts at end of manuscript and adds 'insert table x about here' lines.
  \renewcommand{\figureplace}{
    \begin{center}
    \begin{singlespace}
    ------------------------------------\\
    Insert \figurename \ \thepostfig\ about here.\\
    ------------------------------------
    \end{singlespace}
    \end{center}}
  \renewcommand{\tableplace}{%
    \begin{center}
    \begin{singlespace}
    ------------------------------------\\
    Insert \tablename \ \theposttbl\ about here.\\
    ------------------------------------
    \end{singlespace}
    \end{center}}

  \usepackage{titlesec}
   \titleformat{\title}
       {\filcenter\normalfont\bfseries\uppercase}{\thetitle}{1em}{}
  \titleformat{\section}
    {\filcenter\normalfont\bfseries\uppercase}{\thesection}{1em}{}
  \titleformat{\subsection}
    {\normalfont\bfseries}{\thesubsection}{1em}{}
  \titleformat{\subsubsection}[runin]
   {\normalfont\bfseries\slshape}{\thesubsubsection}{1em}{\hspace*{\parindent}}
       
\usepackage{tabu}
\usepackage{textcomp}
\usepackage{listings}
\usepackage{hyperref}
\usepackage{verbatim}
\usepackage{tabu}
\hypersetup{
    colorlinks=true,
    linkcolor=blue,
    filecolor=cyan,      
    urlcolor=cyan,
    citecolor=blue,
}

\usepackage{etoolbox}

\makeatletter

% Patch case where name and year are separated by aysep
\patchcmd{\NAT@citex}
  {\@citea\NAT@hyper@{%
     \NAT@nmfmt{\NAT@nm}%
     \hyper@natlinkbreak{\NAT@aysep\NAT@spacechar}{\@citeb\@extra@b@citeb}%
     \NAT@date}}
  {\@citea\NAT@nmfmt{\NAT@nm}%
   \NAT@aysep\NAT@spacechar\NAT@hyper@{\NAT@date}}{}{}

% Patch case where name and year are separated by opening bracket
\patchcmd{\NAT@citex}
  {\@citea\NAT@hyper@{%
     \NAT@nmfmt{\NAT@nm}%
     \hyper@natlinkbreak{\NAT@spacechar\NAT@@open\if*#1*\else#1\NAT@spacechar\fi}%
       {\@citeb\@extra@b@citeb}%
     \NAT@date}}
  {\@citea\NAT@nmfmt{\NAT@nm}%
   \NAT@spacechar\NAT@@open\if*#1*\else#1\NAT@spacechar\fi\NAT@hyper@{\NAT@date}}
  {}{}

\lstset{
basicstyle=\ttfamily,
columns=flexible,
breaklines=true
}
\newenvironment{hypothesis}{
  	\itshape
  	\leftskip=\parindent \rightskip=\parindent
  	\noindent\ignorespaces}

\fancypagestyle{plain}{
  \renewcommand{\headrulewidth}{0pt}
  \fancyhf{}
}	


\begin{document}
\title{Contradictory Predictions from Applying Capabilities and Transaction Cost Economics Approaches to Firms}
\date{}
\maketitle

\begin{abstract} 
\normalsize 
We apply a formal model to understand the effects of the relative learning rates of embedded agents and the institutional field on organizational outcomes. 
\end{abstract}


{\textbf{Keywords:} \\\indent Embedded Agency}

\newpage
\pagestyle{fancy}
\fancyhf{}
\lhead{Capabilities vs. TCE Approaches to Firms}
\rhead{\thepage}

\begin{center}
\textbf{The Capabilities Critique of Transaction Cost Economics}
\end{center}

\cite{Teece2007} Dynamic capabilities enable business enterprises to create, deploy, and protect the intangible assets that support superior long- run business performance.

\cite{Jacobides2005a} suggest that transaction costs are intertwined with capabilities when it comes to determining vertical scope. They suggest that capability differences are a necessary condition for vertical integration and that transaction cost reductions are only incidental. They suggest that four evolutionary mechanisms shape vertical scope over time:
First, selection processes themselves shape vertical scope. Selection processes are driven by capability differences, while transaction costs may be seen as static.

Second, transaction costs may be endogenously changed as firms reshape their market and environment in the quest for greater profit and market share. 

Third, capability development processes reshape the capability pool in an industry over time and therefore changing the set of qualified participants in vertical integration. 

\cite{Ghoshal1996} Organizations are not mere substitutes for structuring efficient transactions when markets fail; they possess unique advantages for governing certain kinds of economic activities through a logic that is very different from that of a market.

\cite{Moran1996} suggest that markets and industrial organizations feature a vastly different set of dynamics. It is suggested that managers cannot run business enterprises or corporations based on transactional cost economics (TCE) because that particular theory is meant to decipher market situations.

Dynamic managerial capabilities \cite{Adner2003} - managerial human capital, managerial social capital and managerial cognition. Dynamic capabilities involve higher-level activities that can enable an enterprise to direct its ordinary activities toward high-payoff endeavors. \cite{Teece2014b} suggests that TCE holds ?production? activity invariant even though production costs may depend endogenously on the governance modes, managerial actions, strategy, and structures chosen. Moreover, production technologies and governance modes are not, in the economic perspective, proprietary, but rather available to all firms. Omniscience, not ignorance, is the norm. Managers are essential to resource allocation and
economic activity (and the theory of the firm). The price system has little relevance to the internal allocation of resources within firms because firms, for very good reasons, generally eschew the use of prices as the exclusive tool to determine the internal distribution of resources. The allocation process is instead orchestrated by managers.markets for high-specificity (idiosyncratic) assets generally don?t exist, and if they do exist they are invariably ?thin.? To overcome this problem, managers collect information, sense opportunities, invest in capabilities, innovate, and transform. They become the instruments that help achieve the shrewd allocation of company resources. Markets and internal resource allocation (organized hierarchically inside the
firm) are not only substitutes, as Coase (1937) implicitly claimed; they are also complements. Williamson seems to agree, noting that ?the relation between competence and governance [is] both rival and complementary?more the latter than the former?
(1999, p. 1106). And even if transaction costs were zero and governance problems evaporated, learning, co-creation (Pitelis \& Teece, 2009), and asset/resource orchestration functions would still need to be carried out. The entrepreneurially managed business firm is where this can be done.

Asset orchestration involves identify- ing the critical assets and investing in them (search/selection), and then developing a governance system along with a means for their effective use identified. The second part of asset orchestration involves the coordination of co-specialized assets and their use in productive ways (configuration/deployment).

\cite{Eggers2009} measured managerial cognition in terms of CEO attention. They found that this aspect of dynamic managerial capabilities had a positive impact on the ability of incumbent firms to adapt to radical technological change through faster entry into an evolving new market.

\cite{Sirmon2009} found that asset orchestration through resource investment and deployment worked best when managers made congruent rather than independent resource investment and deployment decisions.

\cite{David2004} to suggest that even empirical results have been somewhat mixed, with the strongest results only for asset specificity. (Pick from the IO term paper)

\subsection{Effects of Intrafirm Specialization}
\cite{Jacobides2005b} suggests that gains from intrafirm specialization set off a process of intraorganizational partitioning, which simplifies coordination along parts of the value chain. Similarly, latent gains from trade foster interfirm cospecialization, which leads to information standardization.

\cite{Mayer2006} argue that strong technological capabilities improve a firm's ability to govern transactions, making outsourcing feasible despite certain contractual hazards.

\cite{Parmigiani2007} shows that concurrent sourcing is a distinctly different choice that may co-exist along a make/buy continuum.

\cite{Poppo1998}  suggest that a theory of the firm and a theory of boundary choice is likely to be complex, requiring integration of transaction cost, knowledge-based, and measurement reasoning

\cite{Mayer2004} find many changes to the structure of the contracts that cannot be fully explained by changes in the assets at risk in the relationship, and evidence that these changes are largely the result of processes in which the firms were learning how to work together, including learning how to contract with each other
 
By drawing on a detailed data set of the luxury automobile market, \cite{Novak2009} show support for complementarity in product development, suggesting therefore that contracting complementarity may be particularly important when coordination is important to achieve but difficult to monitor.
 
\cite{David2004} find considerable disagreement on how to operationalize some of TCE's other central constructs and propositions, and relatively lower levels of support for uncertainty and performance. \cite{David2004} suggest that two problems need to be addressed. First, some of the key propositions, including that relating to uncertainty have been loosely interpreted, and second that some key variables such as performance have received little scrutiny. \cite{David2004} also suggest that tests about the effects of governance forms on performance are likely to suffer form self-selection bias. They also suggest that tests of the relationship between asset specificity and governance form are tests of the largest surviving firms thereby indicating a survivor bias. \cite{David2004} also suggest that TCE has not been applied across time, in longitudinal studies. 

My conclusion will be that TCE is useful in a limited context within the larger capabilities framework, particularly when transaction costs maybe safely assumed to be exogenous. Maybe over shorter periods of time.

Absorptive capacity is an organizational ability to evaluate, assimilate and commercialize knowledge that originates outside the firm \citep{Cohen1990}.
My contradictory case can involve a question of time: TCE has no notion of time and therefore may predict a certain thing. However, capabilities are improved and created over time, and the capabilities angle may suggest different. Say something is not asset specific, TCE will suggest no need to vertically integrate. However, for the development of absorptive capacity, or create a market environment within the firm (intrafirm competition), firms may still acquire and integrate. Think of an example: both ways. When TCE says you should vertically integrate and Capabilities says you shouldn't. And vice versa. Discuss the trends underneath. And the cumulative effect. Another theme to bring in is Organizational ambidexterity \citep{Raisch2008}

\section{Interpretation of Model Results}
\subsection{On the topic of the general hypotheses}
 Figure ~\ref{fig:3a} lays out the average score charts for four agent-field combinations while enforcing the field to start in Right of Center (this is the same as saying $p_{0,F}^0 = 0.75$). 
\subsubsection{Leading into H1a}
We do so since the scale is symmetric across the Center (C), any initial mapping 

\begin{hypothesis}
{Hypothesis 1a: When the institutional field is open to influence, slow learning adversarial agents will raise overall performance higher than slow learning agents with a neutral orientation\\}
\end{hypothesis}

\section{Limitations and Future Work}
The

\begin{singlespace}
\renewcommand{\refname}{REFERENCES}
\bibliography{/Users/aiyenggar/code/bibliography/aiyenggar} 
\bibliographystyle{ai-amjlike}
\end{singlespace}

\end{document}
