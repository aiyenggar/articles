%1521001##1##.tex
\documentclass[12pt,letterpaper]{article}
\usepackage{mathptmx}
\usepackage[margin=1in]{geometry}

\usepackage{setspace}
\singlespacing
  
\usepackage{amssymb,latexsym}
\usepackage[round,sort]{natbib}
\usepackage{fancyhdr}
\usepackage{lastpage}
\usepackage{graphicx,multirow}
\graphicspath{ {qe3/} }

% Bold Table and Figure captions
\usepackage{caption}
\captionsetup{figurename=FIGURE}
\captionsetup{tablename=TABLE}
\captionsetup[figure]{labelfont=bf}
\captionsetup[table]{labelfont=bf}
  
% Turns off all section numbering
\setcounter{secnumdepth}{0} 

  % Places all tables at end of document and creates AOM-style table-here placeholders
  \usepackage[nolists]{endfloat} % Places all figures and charts at end of manuscript and adds 'insert table x about here' lines.
  \renewcommand{\figureplace}{
    \begin{center}
    \begin{singlespace}
    ------------------------------------\\
    Insert \figurename \ \thepostfig\ about here.\\
    ------------------------------------
    \end{singlespace}
    \end{center}}
  \renewcommand{\tableplace}{%
    \begin{center}
    \begin{singlespace}
    ------------------------------------\\
    Insert \tablename \ \theposttbl\ about here.\\
    ------------------------------------
    \end{singlespace}
    \end{center}}

  \usepackage{titlesec}
   \titleformat{\title}
       {\filcenter\normalfont\bfseries\uppercase}{\thetitle}{1em}{}
  \titleformat{\section}
    {\filcenter\normalfont\bfseries\uppercase}{\thesection}{1em}{}
  \titleformat{\subsection}
    {\normalfont\bfseries}{\thesubsection}{1em}{}
  \titleformat{\subsubsection}[runin]
   {\normalfont\bfseries\slshape}{\thesubsubsection}{1em}{\hspace*{\parindent}}
       
\usepackage{tabu}
\usepackage{textcomp}
\usepackage{listings}
\usepackage{hyperref}
\usepackage{verbatim}
\usepackage{tabu}
\hypersetup{
    colorlinks=true,
    linkcolor=blue,
    filecolor=cyan,      
    urlcolor=cyan,
    citecolor=blue,
}

\usepackage{etoolbox}

\makeatletter

% Patch case where name and year are separated by aysep
\patchcmd{\NAT@citex}
  {\@citea\NAT@hyper@{%
     \NAT@nmfmt{\NAT@nm}%
     \hyper@natlinkbreak{\NAT@aysep\NAT@spacechar}{\@citeb\@extra@b@citeb}%
     \NAT@date}}
  {\@citea\NAT@nmfmt{\NAT@nm}%
   \NAT@aysep\NAT@spacechar\NAT@hyper@{\NAT@date}}{}{}

% Patch case where name and year are separated by opening bracket
\patchcmd{\NAT@citex}
  {\@citea\NAT@hyper@{%
     \NAT@nmfmt{\NAT@nm}%
     \hyper@natlinkbreak{\NAT@spacechar\NAT@@open\if*#1*\else#1\NAT@spacechar\fi}%
       {\@citeb\@extra@b@citeb}%
     \NAT@date}}
  {\@citea\NAT@nmfmt{\NAT@nm}%
   \NAT@spacechar\NAT@@open\if*#1*\else#1\NAT@spacechar\fi\NAT@hyper@{\NAT@date}}
  {}{}

\lstset{
basicstyle=\ttfamily,
columns=flexible,
breaklines=true
}
\newenvironment{hypothesis}{
  	\itshape
  	\leftskip=\parindent \rightskip=\parindent
  	\noindent\ignorespaces}

\fancypagestyle{plain}{
  \renewcommand{\headrulewidth}{0pt}
  \fancyhf{}
}	


\begin{document}
\title{Contradictory Predictions from Applying Capabilities and Transaction Cost Economics Approaches to Firms}
\date{}
\maketitle

\begin{abstract} 
\normalsize 

\end{abstract}
I review the literature examining the tensions and contradictions between Transaction Cost Economics (TCE) theory of firm contracting behavior and the Capabilities literature examining firm behavior. I highlight several themes in the criticism of TCE and conclude that while there may be no easy way to integrate the two schools. I conclude with a research proposal that asks why firms both make and buy.

{\textbf{Keywords:} \\\indent Dynamic Capabilities, Transaction Cost Economics}

\newpage
\pagestyle{fancy}
\fancyhf{}
\lhead{Capabilities vs. TCE Approaches to Firms}
\rhead{\thepage}

\begin{center}
\textbf{The Capabilities Critique of Transaction Cost Economics}
\end{center}

Assuming bounded rationality and opportunism as attributes of contracting actors in firms, Transaction Cost Economics (TCE) uses an economizing argument to predict the nature of governance of transactions between firms \citep{Williamson1985}.  Transaction attributes of asset specificity, frequency and uncertainty are suggested to determine the nature of governance between three options: hierarchy, market or hybrid organization. Where the Transaction Cost Economics argument glosses over managers\textquotesingle \ ability to learn and adapt to a changing external environment over time, the Dynamic Capabilities (or simply, Capabilities) approach suggests that capabilities ``enable business enterprises to create, deploy, and protect the intangible assets that support superior long-run business performance" \citep{Teece2007}. Scholars have debated the relative merits and demerits of both theories of firm behavior over the last two decades. In this article, I attempt to organize some of those arguments into a coherent set of dimensions, and then propose a study to expand the scope of this conversation further.

\section{TCE\textquotesingle s Contribution}
The most durable contribution of the TCE approach has been to highlight the role of co-specialized investments. TCE suggests that firms cannot easily choose a different trading partner without incurring significant adjustment costs due to asset specificity. This framing raises a potential paradox for firms: Can the value created by specific investments be destroyed by ex-post exchange hazards such as haggling? Firms may seek to reap value either by enhancing quality or by reducing production costs associated with the transaction. As will be highlighted in the following arguments, the ex-ante reorganization of governance mode based on potential ex-post governance costs is a valid but probably not an exclusive solution to the paradox.

\section{Arguments against TCE}
In this section, I organize some of the criticisms of TCE being the sole theory explaining firm decisions. While they apply to firm decisions in general, the criticisms are most salient in four aspects of firm behavior where TCE has had the strongest impact. First, make-or-buy decisions or decisions of vertical integration; Second, decisions about structuring of internal organizations; Third, about alliance and partnership decisions; and Finally, about diversification decisions. Some scholars have even argued that the assumptions of opportunism presents human manager in poor light and causes even poorer behavior as the theory is accepted as real \citep{Ghoshal1996}. I lay out the main strands of criticism against TCE in the following sections.

\subsection{Transaction Costs are not Exogenous}
\cite{Jacobides2005a} suggest that transaction costs are intertwined with capabilities when
it comes to determining vertical scope. They suggest that capability differences are a necessary
condition for vertical integration and that transaction cost reductions are only incidental. They
suggest that evolutionary mechanisms shape vertical scope over time: First, selection processes
themselves shape vertical scope. Selection processes are driven by capability differences,
while transaction costs may be static. Second, transaction costs may be endogenously changed as firms reshape their market and
environment in the quest for greater profit and market share.
Finally, capability development processes reshape the capability pool in an industry over time
and therefore changing the set of qualified participants in vertical integration. The implications of \cite{Jacobides2005a} argument is that while TCE may well explain contracting behavior in the short run, TCE would make incorrect predictions over time.

\subsection{Learning to Govern and Contract}
\cite{Mayer2004} emphasized the learning ability between two cooperating firms to design contracts and stressed the role of formal contracts as a repository of knowledge within and among firms. They also argued that the ability to design formal contracts is capability to deal with the fast changing environment and that this contracting ability increases with the familiarity of interacting parties \cite{Mayer2004}. \cite{Mayer2004} also suggest that many changes to the structure of the contracts may not be fully explained by changes in the assets at risk in the relationship. The demonstrate evidence that these changes are largely the result of processes in which the firms were learning how to work together, including learning how to contract with each other. Learning therefore seems to be a much better predictor of contracting behavior over time.
 

\subsection{Firms as Human Organizations}
\cite{Ghoshal1996} suggest that organizations are not mere substitutes for structuring efficient transactions when markets fail. Firms possess unique advantages for governing certain kinds of economic activities that may not be able to be performed by a market. Building the argument further, \cite{Moran1996} suggest that markets and industrial organizations feature a vastly different set of dynamics. They suggest that managers cannot run business enterprises or corporations based on transactional cost economics (TCE) because TCE theory is meant to decipher market situations.

\subsection{Limitations of the Price System of the Market}
Dynamic managerial capabilities \citep{Adner2003} literature highlights the importance of managerial human capital, managerial social capital and managerial cognition in driving firm decisions. Dynamic capabilities involve higher-level activities that can enable an enterprise to direct its ordinary activities toward high-payoff endeavors \citep{Teece2014a}. \cite{Teece2014b} suggests that TCE holds production activity invariant even though production costs may depend endogenously on the governance modes, managerial actions, strategy, and structures chosen. Additionally, Teece argues that modes of production and modes of governance are available to all firms and therefore firm decisions are ultimately a function of managerial cognition and ability. Managers are therefore essential to resource allocation and the price system has but little relevance to the internal allocation of resources within firms. First markets for highly specific and idiosyncratic assets may not exit. Sometimes, even if they do those markets may be infrequent activity. Managers overcome this problem through information, sensemaking, investments, innovations and strategic action. However this does not write off the role for markets. Williamson seems to concede that competence and governance may be both rival and complementary \citep{Williamson1999}.  \cite{Pitelis2009} suggest that even if transaction costs were to go down to zero, no governance issues existed, learning, co-creation and asset/resource orchestration functions would still need to be carried out. This is probably the strongest argument in favor of the  entrepreneurially managed business firm and the most damning of the criticisms against TCE. I next present a couple of arguments about the value of characteristics highlighted by the capabilities literature.

\subsection{Internal Organization}
\cite{Jacobides2005b} suggested that gains from intrafirm specialization set off a process of intraorganizational partitioning. This partitioning, Jacobides suggested simplifies coordination along parts of the value chain. Alternatively, latent gains from trade foster may also increase interfirm cospecialization. This leads information standardization, as evidenced in the development Internet and media protocols. \par
Another key capability suggested by the literature is the importance of asset orchestration as a capability in entrepreneurial firms. Asset orchestration involves identifying the critical assets and investing in them through search or selection. Governance mechanisms are not determined ex-ante based on costs, but much later depending on alternative means available for their effective use. The other aspect of asset orchestration involves the coordination of co-specialized assets so that they may be used in productive ways. This again requires managers to experiment with various configurations and deployment. \cite{Mayer2006}, for example argue that strong technological capabilities improve a firm's ability to govern transactions, making outsourcing feasible despite certain contractual hazards. Clearly, the alternatives from the capabilities literature offer a wider scope and dynamic explanation of contracting methods. However, they do account for governance hazards and therefore any integrated theory of the firm may not be able to dismiss TCE altogether.

\subsection{Issues from Empirical Studies}
In this section, I highlight some of the issues and concerns concerning TCE that have emanated from empirical studies. In a review of the empirical literature on TCE, \cite{David2004} suggest that even empirical results have been somewhat mixed, with the strongest results only for asset specificity. Aditionally, \cite{David2004} find considerable disagreement on how to operationalize some of TCE\textquotesingle s other central constructs and propositions, and relatively lower levels of support for uncertainty and performance. In conclusion they suggest that two problems need to be addressed. First, some of the key propositions, including that relating to uncertainty have been loosely interpreted, and second that some key variables such as performance have received little scrutiny. Additionally, like much other empirical research in strategy many empirical studies attesting to TCE may also suffer form self-selection bias. \cite{David2004} also suggest that TCE has not been applied across time, in longitudinal studies. 
On the other hand, empirical support has been strengthening for the capability approach. \cite{Eggers2009} measured managerial cognition in terms of CEO attention and found that this aspect of dynamic managerial capabilities had a positive impact on the ability of incumbent firms to adapt to radical technological change. By drawing on a detailed data set of the luxury automobile market, \cite{Novak2009} demonstrated support for complementarity in product development, suggesting therefore that contracting complementarity may be particularly important when coordination is important to achieve but difficult to monitor. \cite{Sirmon2009} found that asset orchestration through resource investment and deployment worked best when managers made congruent rather than independent resource investment and deployment decisions. Clearly, we understand a lot about both the strengths and the weaknesses of TCE. The larger question is what do these results imply for explaining firm level economic behavior?


\subsection{Does it have to be an either/or choice?}
Scholars have begun to suggest that it may not necessarily have to be one or the other. \cite{Parmigiani2007} shows that concurrent sourcing is a distinctly different choice that may co-exist along a make/buy continuum. Additionally, \cite{Poppo1998}  suggest that a theory of the firm and a theory of boundary choice is likely to be complex, requiring integration of transaction cost, knowledge-based, and measurement reasoning. Scholars have argued that the two approaches are both both complementary \citep{Jacobides2005a, Poppo1998}, \cite{Nickerson2012} suggest that it may be hard to coherently design organizations by combining both approaches. Scholars therefore suggest that an intuitive understanding of how firms develop and renew firm-level capabilities requires research attention to both how much firms invest and how effectively these strategic investments are managed and governed \citep{Argyres1996, Kor2005, Mayer2006}.

My conclusion from the review of the literature is that while the Capability literature and TCE may always some tension between them, that it may be possible to design limited studies where the perspective of each may enhance the understanding of the phenomenon. Such incremental insights may sit uncomfortably with either theories, but may be valuable from a practitioner perspective. Another possible, but somewhat messy alternative is to have rule based, step like explanation where the appropriate theories are drawn in depending on stylized observations about the phenomenon. I now move on to propose a research study at the intersection of TCE and Capabilities literature.

\section{Research Proposal}
In this section, I build upon the arguments made in the previous section for an integrated Capabilities-TCE approach to propose a study to further our understanding. While each of the four domains of vertical integration (make/buy decisions), internal organization, alliances and diversification may provide an interesting setting for such a study, I chose the context of outsourcing contracts in the Information Technology (IT) sector. I propose a longitudinal study of MNC IT contracting over three decades to capture the simultaneity of the Capability and TCE effects in firm contracting decisions. Anecdotal evidence suggests that many of the MNCs that contract out IT work to services firms operate fairly large IT operations themselves. This suggests, like \cite{Parmigiani2007} that the choice to make or buy may not quite be a make or buy decision, but a question of when to buy more/buy less, and when to make more/make less. In the spirit of this inquiry, I ask the following questions: Why do firms that contract work, ramp up in-house capabilities while continuing to contract work? Why do firms that develop in-house, contract work while continuing to develop-in house? While it may seem that I am asking two separate questions, the reason I lay them out here together because it seems to me that the underlying mechanisms driving the two trends may be related. In case this does not turn out to be so, the questions will need to be addressed in separate studies.

\subsection{Theory}
\cite{Cohen1990} suggested that firms seek to build absorptive capacity so as to evaluate, assimilate and commercialize knowledge that originates outside the firm. Organizational ambidexterity \citep{Raisch2008}. 
My contradictory case can involve a question of time: TCE has no notion of time and therefore may predict a certain thing. However, capabilities are improved and created over time, and the capabilities angle may suggest different. Say something is not asset specific, TCE will suggest no need to vertically integrate. However, for the development of absorptive capacity, or create a market environment within the firm (intrafirm competition), firms may still acquire and integrate. Think of an example: both ways. When TCE says you should vertically integrate and Capabilities says you shouldn't. And vice versa. Discuss the trends underneath. And the cumulative effect. Another theme to bring in is 
What are the two opposite effects? And how can they be cumulated? What variables would be the moderators?

\subsubsection{Leading into H1}
We do so since the scale is symmetric across the Center (C), any initial mapping 
Some sort of competition hypothesis, in addition to absorptive capacity.
\begin{hypothesis}
{Hypothesis 1: Explain why firms will increase make when they are currently buying\\}
\end{hypothesis}

\subsubsection{Leading into H2}
We do so since the scale is symmetric across the Center (C), any initial mapping 

\begin{hypothesis}
{Hypothesis 2: Explain why firms will increasing buying when they are currently also making\\}
\end{hypothesis}

\subsection{Data and Method}
Look at the IT contracting behavior of Fortune 500 firms over 1985 to 2015. Measure if they solely contract ($\geq$ 80\% of budget), solely build  ($\geq$ 80\% of budget), or do both (if both are atleast 20\%)

\section{Limitations}
Suggest how this study may help inform the literatures that it is drawing from, and the interesting research avenues it will open up. Discuss level of generalizability.

\section{Summary}
Recap and motivate interest in framework, in theoretical value as well as in the particular empirical setting.


\begin{comment}
\section{Notes}

\end{comment}

\begin{singlespace}
\renewcommand{\refname}{REFERENCES}
\bibliography{/Users/aiyenggar/code/bibliography/aiyenggar} 
\bibliographystyle{ai-amjlike}
\end{singlespace}

\end{document}
