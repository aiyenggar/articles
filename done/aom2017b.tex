%aom2017b.tex
\documentclass[12pt,letterpaper]{article}
\usepackage{mathptmx}
\usepackage[margin=1in]{geometry}

\usepackage{setspace}
\doublespacing
  
\usepackage{amssymb,latexsym}
\usepackage[round,sort]{natbib}
\usepackage{fancyhdr}
\usepackage{lastpage}
\usepackage{graphicx}
\graphicspath{ {aom2017b/} }

% Bold Table and Figure captions
\usepackage{caption}
\captionsetup{figurename=FIGURE}
\captionsetup{tablename=TABLE}
\captionsetup[figure]{labelfont=bf}
\captionsetup[table]{labelfont=bf}
  
% Turns off all section numbering
\setcounter{secnumdepth}{0} 

  % Places all tables at end of document and creates AOM-style table-here placeholders
  \usepackage[nolists]{endfloat} % Places all figures and charts at end of manuscript and adds 'insert table x about here' lines.
  \renewcommand{\figureplace}{
    \begin{center}
    \begin{singlespace}
    ------------------------------------\\
    Insert \figurename \ \thepostfig\ about here.\\
    ------------------------------------
    \end{singlespace}
    \end{center}}
  \renewcommand{\tableplace}{%
    \begin{center}
    \begin{singlespace}
    ------------------------------------\\
    Insert \tablename \ \theposttbl\ about here.\\
    ------------------------------------
    \end{singlespace}
    \end{center}}

  \usepackage{titlesec}
   \titleformat{\title}
       {\filcenter\normalfont\bfseries\uppercase}{\thetitle}{1em}{}
  \titleformat{\section}
    {\filcenter\normalfont\bfseries\uppercase}{\thesection}{1em}{}
  \titleformat{\subsection}
    {\normalfont\bfseries}{\thesubsection}{1em}{}
  \titleformat{\subsubsection}[runin]
   {\normalfont\bfseries\slshape}{\thesubsubsection}{1em}{\hspace*{\parindent}}
       
\usepackage{tabu}
\usepackage{textcomp}
\usepackage{listings}
\usepackage{hyperref}
\usepackage{verbatim}
\usepackage{tabu}
\hypersetup{
    colorlinks=true,
    linkcolor=blue,
    filecolor=cyan,      
    urlcolor=cyan,
    citecolor=blue,
}
\lstset{
basicstyle=\ttfamily,
columns=flexible,
breaklines=true
}
\newenvironment{hypothesis}{
  	\itshape
  	\leftskip=\parindent \rightskip=\parindent
  	\noindent\ignorespaces}

\fancypagestyle{plain}{
  \renewcommand{\headrulewidth}{0pt}
  \fancyhf{}
  \rhead{15820}
  \cfoot{\thepage}
}	


\begin{document}
\title{Complexity Theory for Strategy Research: Philosophical Arguments}

\maketitle

\begin{abstract}
\normalsize
This paper presents a review of the philosophical foundations of the dominant theories in strategic management and argues that these foundations lead to significant limitations for the theories\textquotesingle  \ applicability to practice.  Complexity theory is presented as an alternative that is better aligned to the phenomena researched by strategy scholars and its philosophical underpinnings and implications are discussed.
\end{abstract}


{\textbf{Keywords:} \\\indent Philosophical Foundations of Strategic Management, Complexity Theory}

\newpage
\pagestyle{fancy}
\fancyhf{}
\lhead{Complexity Theory for Strategy Research}
\cfoot{\thepage}
\rhead{15820}
%\section{Introduction}\label{S:Introduction}
\begin{center}
\textbf{Complexity Theory for Strategy Research: Philosophical Arguments}
\end{center}
\section{Introduction}\label{S:Introduction}
Referring to the state of strategic management research from his vantage point as an observer of large firms, \cite{Bettis1991} begins his editorial essay with the observation that "I am often struck by the sense that most of this research is irrelevant to what is going on in such firms today". While two and a half decades have since passed, the area continues to grapple with the problem of academic research not being able to influence the practice of strategy \citep{Economist2007PracticallyIrrelevant}. The strongly positivistic assumptions originating from the economics roots of the field have been criticized by scholars as constraining the potential contributions that can be made by strategy research. 

In the early 21st century, strategy scholars debated on the usefulness of a constructivist methodology as evidenced in the exchanges between \cite{Mir2000}, \cite{Kwan2001}, and \cite{Mir2001}. \cite{Durand2002} and \cite{Powell2002} even debated the contributions of pragmatism in the explanation of competitive advantage. A particularly scathing was presented by \cite{Rasche2008}   where it was argued that the field of strategy is built on paradoxical foundations and that strategy researchers have been blind to the philosophical inconsistencies that are present in the dominant theories in the field. 

This challenge suggests both an opportunity to contribute to the literature by nudging an improvement on the one hand, and the rather bleak thought that the problem may be intractable. One may easily be forgiven to take the  latter  to be true, given the pedigree and sheer number of scholars who have applied their minds on the issue.  Can there be an alternate paradigm that can take the area forward? This paper is the author \textquotesingle s  quest at the resolution of a philosophical problem that seems to stands central to the pursuit of research in strategic management.

The rest of the paper is organized as follows. We first present the dominant theories in strategic management and then dissect their philosophical moorings and associated implications. Having argued that the positivist-empiricist philosophical stance has been the bane of strategy research in its effort to remain relevant to practice, we then introduce complexity theory and its basic principles. We then provide a detailed assessment of the philosophical assumptions underlying complexity theory, and argue that with a new view of change, this approach may address many of the problems identified with connecting with the practice of strategy making and implementation. 

\section{Dominant Strategic Management Theories}\label{S:DominantTheories}
\cite{Pettigrew1988} had proposed that research in strategic management may be seen as confirming to within a tripartite framework consisting of a) strategy content - with it's focus on the decision itself, b) strategy process - with it's focus on how the decisions are made, and c) strategy context - with the focus on what circumstances influence strategic decisions. This framework has since been widely accepted by scholars in the area, and has also been the framework used by \cite{Rasche2008} in his critique of the philosophical foundations of the dominant theories in strategic management.  

As \cite{Teece1997b} capture, the strategy area was dominated in the 1980s by the industrial organization driven attenuating competitive forces theory \citep{Porter1980}. This was then followed by the game theorists whose focus was on strategic interactions. This school of theorizing did not sustain prominence and was superseded by the resource-based view (RBV), that sought to counter the Porterian view by suggesting that firm heterogeneity was a necessary condition for the creation of competitive advantage. While the Porterian view saw the environment as a given, the Resource-based scholars saw firm resources as out there and given. The Dynamic Capabilities school added to the Resource-based school the notion that firms had capabilities and that resources were able to be created through those capabilities. \cite{Teece1997b} reiterate what strategy scholars have noted in the past - that all the dominant theories in the strategy area are built on a positivist-empiricist philosophical stance where the environment is real, and out there, and where managers are believed to be rational.

\section{Issues in the Phenomena under Study}
Strategy scholars have dealt with both uncertainty and complexity in the phenomena that they have researched for a long time. However, these are indeed the two dimensions that continue to plague many theoretical models because the conception of either does not lend itself elegant mathematics, or when it does, it does not lead to theory that is reflective of the world of business that strategy researchers study. 
\subsection{Complexity}
Complexity arises in firms and industries by virtue of several interacting constituents, including governments, individuals, various stakeholders, and competitors. The typical firm today operates in several countries and by definition is therefore complex. Dominant theories in strategy, and much of the rest of the business school disciplines have tended to make simplifying assumptions, and developed theories assuming a condition of ceterus paribus. This simplification of reality leads to a divergence in the understanding of the context between the researcher and practitioner. Any attempt at bridging this academic-practitioner divide must therefore pay explicit attention to the inherent complexity of the phenomena in strategic management.
\subsection{Uncertainty}
While practitioners and researchers often refer to uncertainty being an inherent aspect of the world of business, the notion of uncertainty has tended to be difficult to clearly define. In some ways, uncertainty is seen to go along with increasing complexity. However, the two are not quite the same. It is possible for physical systems to be complex (in that they consist of many interacting constituent components) without being perceived to demonstrate uncertainty. Many dominant theories in strategic management make the assumption that uncertainty is either ignorable or that uncertainty may be statistically quantified and therefore managed. These simplifications are again at odds with the way practitioners perceive uncertainty and must therefore be explicitly addressed in any alternate theoretical formulation.

\section{Philosophical Constructions}
We noted previously that \cite{Teece1997b} had construed the dominant theories in strategic management as being positivist with a firm notion of a real world that was out there independent of managers, and that managers were assumed to behave rationally. \cite{Ramoglou2016} use the context of entrepreneurship to delineate the three different philosophical schools of empiricism (that is similar to what we have so far characterized as positivist), constructivism, and realism.\cite{Ramoglou2016} provide a summary characterization of the three schools, and how they view the entrepreneurial process. Our notion of the world of business is that  while there does exist a real world out there, much of the world is also either made sense of, is socially constructed or is enacted by individuals. Therefore the realist philosophical stance is seen to come closest to mapping accurate the world of business studied by strategy scholars.


\section{Limitations of Dominant Strategic Management Theories}\label{S:LimitationsDominantTheories}
Having laid out the principal dominant theories in strategic management, and the primary philosophical schools available to construct our understanding of phenomena, we now attempt to understand the implications of the positivist-empiricist philosophical positions of the dominant theories in strategy.

\subsection{Ontological and Epistemological Positions}
The nature of strategic management, being inherently associated with questions of day-to-day and practical importance, makes it necessary for both practitioners and researchers to adopt an attitude of determinism in their understanding of the world. Strategy scholars have for long recognized the role of change, and incorporated the pace of technological change, for example in their models. The world of Schumpeterian creative destruction perceived by the practitioner is however one where not only is there change in the environment and in technology, but also one where notions of competition and the constructs of value are changing. With mathematically backed  econometrics models, however, the dominant theories in strategy are unable to model in an adapting ontology or a flexible epistemology. This leads to a divergence in the world modeled by the researcher and the world experienced by the practicing manager.

\subsection{Perceptions on Empiricism}
The economics inspired foundations of strategic management research leads to the strongly empiricist orientation of much strategy research. So, while the practitioner perceives both hard, out-there aspects influencing strategy decisions, as well as softer, hard-to-measure aspects, the strictly empiricist orientation of strategy researchers forces them to ignore what cannot be measured. This leads to missing variables in strategy models. While we sympathize with strategy researchers in their idealism for hard-science, the question is if this fastidiousness is doing justice to the phenomena they intend to study or if it is blinding them to the underlying phenomena.

\subsection{Causality and the Problem of Reverse Causality}
Inspired by the natural and physical sciences, strategy scholars in the dominant theories seek to establish strong causal links. However, studies have often proved inconclusive and the problem of reverse causality has often been a problem. So, while causal explanations are both sought and provided, the problem seems to be that the practitioner finds  the assumptions unrealistic, and the explanations simplistic or just wrong.

\subsection{Predictive Ability}
The hallmark of a good science has been understood to be in the strength of its predictive ability. Starting with \cite{Porter1980}, the emphasis on empirically driven predictability has however seen to have been either overlooked, or oversimplified. This is demonstrated adequately through the numerous examples of application of popular theory to disastrous effects. The application  of the experience curve approach to  the British Motorcycle Industry is but  one of the many salient such episodes in the history of the area. The question that needs to be asked is if predictive ability is an appropriate objective to seek in strategic management research?

\subsection{Rationality}
Much work in the behavioral tradition has conclusively broken the back of 'economic man', and as \cite{Levinthal2011} suggests, we do not have an option but to model strategy as behavioral and assume rationality as a process. Current dominant theories in strategic management are clearly both at a theoretical disadvantage with their strong rationality assessment, as well as are distanced from much practitioner thinking on the subject. A behaviorally rooted model of rationality that accounts for adaptation is therefore crucial in any potential improvement in theorization of managerial action taking.

\subsection{Methods Used}
Epistemological understanding of nature determines the choice of methods used in research. The heavily quantitative measurement based ideology thus skews strategy researchers into mistaking their measures for the underlying phenomena that they are trying to model. The p-value driven research process exacerbates the problem by confounding the problem of measurement with the high levels of confidence that come with highly significant level of coefficients obtained from regression \citep{Bettis2014}.

\subsection{Assumptions about the Purpose of Research}
One of the primary questions that the philosophy of science raises is that about the purpose of research. When the strategy field came out of the economics and industrial organization departments into a field of scholarship of its own, the objective was clearly to enable the practicing manager better manage strategic decision making. As noted in the introduction of this article, this is an aspect that is not lost on scholars in strategic management. The question of how to remain relevant to practice has been one that has been debated several times in the best strategy journals. Why then, do we still see the glaring gap between research and practice?

\subsection{Relationship with Other Studies}
Philosophers of science have always also asked questions about applicability and interoperability of research with neighboring sub-fields. It is natural to expect that the various disciplines within business school build of each other, and into each other given that their fundamental object of research is the business firm, though the respective focus may vary. Strategy theories though seem to hang somewhat uncomfortably among theories of micro organizational behavior, financial capital allocation, microeconomics and to a lesser extent with organizational theory. Are the philosophical assumptions to blame? What may be done to work toward an integrated and continuously integrating theory of the business firm?

\subsection{Summary of Philosophical Issues with Dominant Theories}
In the discussion above, we identified and reiterated that the strongly positivist-empiricist philosophical underpinnings of much strategy theory was constraining researchers from being able to address the strategic reality of firms in the real world. We also identified the need for a consistent, and integrating approach that is not trivializing of the phenomena. Finally we see the case for plurality in both theorization and interpretation of phenomena.

\section{The Case for Complexity Theory}\label{S:ComplexityTheory}
In this section, we make the case for complexity theory to be considered to overcome the shortcomings of traditional strategy theory and bring it closer to the phenomena we wish to understand. We recognize that  complexity theory will challenge the most basic assumptions of the positivist-empiricist researcher, while also recognizing that complexity theory may hardly be a panacea. Specifically, while traditional economic theory, and economics inspired strategy theory has been parsimonious and relatively simple, we may no longer expect that with the application of complexity concepts to strategic management problems. The question is if the tradeoff between additional complexity in theory is to be accepted so we may have a more integrative and dynamic theory that is better at explaining the practical reality of strategic situations in firms. This is indeed a hard question, but nevertheless an important one to ask at the current juncture. 

\section{Essentials of Complexity Theory}\label{S:ComplexityTheory}
\cite{Eisenhardt2011} observe that there has been a paradigm shift from reductionist to a holistic perspective across disciplines, and that while pigeonholing was useful in the past, that it has had the eventual consequence of obscuring an understanding of emergent complex behavior (self-organization, nonlinear dynamics and power-law distributions of system-level phenomena). \cite{Maguire2011} further explain that organizations make up and operate in a Schumpeterian word of creative destruction - an under-determined social and economic reality which is not at, nor heading for, \textquotesingle equilibrium\textquotesingle \ and in which behaviors cannot be interpreted as being \textquotesingle optimal\textquotesingle\ .  Complexity science thus, conclude \cite{Maguire2011}, confirms that our world resembles an ecosystem or organism in the process of developing. By defining or clarifying the mechanisms by which micro-level interactions give rise to macro-level system structures, properties and behaviors, complexity theory provides the conceptual and methodological tools to handle emergence, self-organization, evolution and transformation \citep{Maguire2011}.  \cite{Cilliers1998} captures simply the cornerstones of complex system, and is reproduced here in Table ~\ref{tab:features}                                             
\setlength{\arrayrulewidth}{0.5mm}
%\setlength{\tabcolsep}{18pt}
\renewcommand{\arraystretch}{1.5}

\begin{table}[h!]
\begin{tabu} to 1.0\textwidth { |X[l]|} 
\hline
1. Complex systems consist of a large number of elements\\
2. These elements interact dynamically\\
3.	Interactions are rich; any element in the system can influence or be influenced by any other\\
4.	Interactions are nonlinear\\
5.	Interactions are typically short range\\
6.	There are positive and negative feedback loops of interactions\\
7.	Complex systems are open systems\\
8.	Complex systems operate under conditions far from equilibrium\\
9.	Complex systems have histories\\
10.	Individual elements are typically ignorant of the behavior of the whole system in which they are embedded\\
\hline 
\end{tabu}
\caption{Features of Complex Systems, Adopted from \cite{Cilliers1998}}
\label{tab:features}
\end{table}

\section{Philosophical Characterization of Complexity Theory}\label{S:PhilosophicalFoundationComplexity}
 \cite{Mckelvey2011} suggests that realism allows elements of positivism and relativism to flourish in social science. Specifically \cite{Mckelvey2011} identifies four characteristics, 1. dealing with metaphysical terms, 2. objectivist empirical investigation, 3. recognition of socially constructed meanings of terms, and 4. a dynamic process by which a multi paradigm discipline usually reduces to fewer but more significant theories. This classification directly addresses the shortcomings of the philosophical assumptions of the dominant theories in strategic management discussed earlier.

\subsection{A new view of Ontology and Epistemology}

"While classical science considered an ontology of isolated objects, complexity theory considers an ontology of connected entities i.e. a network which has links that change, nodes that change internally and capabilities that develop and change over time.  Complexity scholars recognize that often both the model and the modelers are not outside the system being modeled but within." Maguire et al. (2006: 197) suggest that the shift away from elegant mathematical representations of idealized processes to agent-based computational models allows organizational researchers to pursue the epistemological advantages of models and experiments while not having to assume away important or essential features of organizational reality simply to make the mathematics tractable. We may therefore now create models that assume a) idiosyncratic heterogeneity among individuals or firms (which had previously been assumed to be homogeneous to keep the mathematics manageable), b) interdependence among agents (which was previously eliminated, assuming agent independence), and c) the emergent nature of agent interactions (which was previously ignored to keep the variables at a single level of analysis). This therefore allows us to create rich, life-like models and therefore address the significant shortcomings of extant theories dominant in strategic management research.

\cite{Eoyang2011} reflects on complexity as an epistemological and ontological phenomenon suggesting that complex adaptive systems work against a backdrop of an ever-changing context, that therefore precludes the opportunity to separate what is happening from one\textquotesingle s ability to know what is happening. Therefore at such a time, the boundary between ontology and epistemology is seen to blur. \cite{Eoyang2011} therefore concludes that practitioners would see thinking and real-world causality merge when engaging in an adaptive complex environment.

In summary, with complexity theory we are able to break out of the barriers of the positivist-empiricist philosophical stance, and allow our agents to be modeled more richly, for emergent effects to surface, and for ontology and epistemology to also keep up with the changing world.

\subsection{Prediction vs. Explanation}
Positivism and realism differ in yet another dimension that informs science\textquotesingle s criterion for acceptance of a theory. While positivist views require predictive power of the theory as a necessary condition, the realist view is that the absence of predictive power does not threaten the scientific status of the discipline. Realism instead focuses on the explanatory power and associated understanding of the phenomena under study. This is an especially useful construct in the context of social sciences where conceptions of human agents can themselves serve to reshape the phenomena under study. In that sense, the future is not knowable in the form of accurate predictions. The future does not exist in the present in the sense that it is awaiting discovery by a human agent. Instead, under the realist worldview, multiple possible futures are thought to exist. \cite{Richardson2011} suggests that the adoption of complexity theory is likely to lead to a renewed interest in philosophy (including of ontology and epistemology) since causality is "complex, intricate, multi-ordered, and intractable in an absolute sense".

\subsection{Implications for Methodology}
The realist philosophical stance as applied to complexity theory described above has significant implications for the methodology for the conduct of research. First, with the possibility for ontology to change, and with it the epistemology, and for the distinction between the two to blur when engaging with an adaptive complex environment, traditional mathematical models may no longer be appropriate. Computer models and simulations may need to replace the traditional mathematical models. However we do not expect that agent level or other atomic level behavior be succinctly and parsimoniously described as a finite state automaton. Stochastic models and interventionist models may also be incorporated depending on the nature of the phenomenon under study. A number of options open up that allow researchers a finer level of understanding, though it must be noted that this increase in explanatory power comes with both lesser parsimony in the theorization, and less determinism about the detailed states. Is it possible that we drive a better understanding of the phenomena around us by leveraging the computing capacity in every pocket and on every desk?

\subsection{Pluralism}
The adoption of complexity theory lends itself to pluralism at multiple levels, primarily because of the philosophical flexibility that is inherent in the framework.
\cite{Richardson2011} suggests that complexity theory may be characterized as three schools: Neo-reductionist, Metaphorical, and Critical Pluralist each of which differ in the extent of plurality allowed, with the critical purist the most generous. The plurality, depending on the context may exhibit behavior as co-operation, competition, sustenance or survival depending on the conditions. Table ~\ref{tab:laws} reproduces a popular perspective on managing complex systems in all their plurality.

\begin{table}[h!]
\begin{tabu} to 1.0\textwidth { |X[l]|} 
\hline
1. Just because it looks like a nail doesn\textquotesingle t mean you need a hammer\\
2. Decisions made by the many are often better than those made by a few\\
3. Expect to be wrong (or at least not completely right)\\
4. Flip-flopping is OK\\
\hline 
\end{tabu}
\caption{The laws of complex organizational management, Adopted from \cite{Richardson2011}}
\label{tab:laws}
\end{table}

\cite{Eisenhardt2011} suggest that two principal propositions are central to complexity theory; first that there is an optimal amount of structure (rooted in the efficiency vs flexibility trade-off \citep{Davis2009}, and second concerning the relationship between optimal structure and the environment - As environmental unpredictability decreases, greater efficiency and therefore more structure become advantageous and vice versa. The optimal degree of structure (and the robustness of its range), therefore, depends upon the unpredictability of the environment \citep{Eisenhardt2001}.

\section{Limitations}\label{S:Limitations}
Human affairs are indeed incredibly complex, and any single individual would be hard stretched to believe that one could comprehend much of the complexity that is within business firms, much less provide practical models for their understanding. In presenting this study, we are keenly aware of our limitations as  agents attempting to understand the complex larger system. In keeping with complexity theory,  we hope to evolve, adapt and learn to perceive and interpret the complex social world of business firms in the years ahead. However, there are a few glaring limitations to the current study that need to be highlighted. 

Firstly, the study is not intended to belittle the monumental work done by towering scholars in the strategy area. If we have been able to see as far as we are today, we owe it to these scholars over the decades who have simplified, abstracted, and debated our construction of the reality of the business firm. 

Second, complexity theory is itself an elaborately broad topic that one may not expect to do justice to the many dimensions to this field of inquiry much less in a  paper on the philosophical foundations appropriate for strategy research. It is therefore only an initial attempt at assessing the attributes of complexity theory to the application in problems of strategy. 

Third, philosophy and philosophical constructs and indeed much of the metaphysics surrounding our endeavor as philosophers of science is extremely abstruse and indeed a slippery slope. We accept any errors made here with the utmost humility. 

Finally, we realize that it was an optimistic decision to both critique dominant theory as well as propose and expound complexity theory within the same paper. While we were  motivated by our interest in figuring out answers to questions that will become important in pursuit of a scholar\textquotesingle s life in the strategy area, this paper is in many ways is reflective the complexity involved in crafting an argument. 

\section{Conclusion}\label{S:Conclusion}
We started with an analysis of the philosophical moorings of the dominant theories in the strategy area. We proposed that complexity theory may provide a functional alternative without having to throw away most of what we have learnt. The possibilities of applying complexity approaches to strategy problems were discussed as holding  promising. Specifically, the availability of computing power, and the ability of personnel in firms to work with computers allows us the possibility that management theory can grow out of the traditional limitations of mathematics and economics, and expand to modeling on behavioral and complex adaptive systems. The philosophical assumptions in such an approach are less stringent, and more inclusive and plural. We hope that an appreciation for a diversity of approaches as seems to be the way of nature may also become reality in management scholarship. With the application of complexity theory with all its flexibility, one hopes that the strategists in both academia and in industry will be better served.

\newpage
\begin{singlespace}
\bibliography{/Users/aiyenggar/OneDrive/code/bibliography/ae,/Users/aiyenggar/OneDrive/code/bibliography/fj,/Users/aiyenggar/OneDrive/code/bibliography/ko,/Users/aiyenggar/OneDrive/code/bibliography/pt,/Users/aiyenggar/OneDrive/code/bibliography/uz} 
\bibliographystyle{aomlike}
\end{singlespace}

\end{document}
