%02-content-b-review-nickerson-silverman.tex
\documentclass[12pt]{article}
\usepackage{times}
\usepackage{amssymb,latexsym}
\usepackage[round,sort]{natbib}
\usepackage{fancyhdr}
\usepackage{lastpage}
\usepackage{graphicx}
\graphicspath{ {poss-term-paper-images/} }
\usepackage[T1]{fontenc}
\usepackage{mathptmx}
\usepackage{tabu}
\usepackage{textcomp}
\usepackage{stata}
\usepackage{listings}
\newenvironment{hypothesis}{
  	\itshape
  	\leftskip=\parindent \rightskip=\parindent
  	\noindent\ignorespaces}
	
\pagestyle{fancy}
\fancyhf{}
\fancyhead{}
\fancyfoot{}
\lhead{Review: Nickerson and Silverman (2003)}
\rfoot{Page \thepage  \ of \pageref{LastPage}}
\rhead{Ashwin Iyenggar (1521001)}

\begin{document}
\title{Review: Nickerson \& Silverman (2003)\\Why Firms Want to Organize Efficiently and what Keeps Them from Doing So}
\author{Ashwin Iyenggar  (1521001) \\ ashwin.iyenggar15@iimb.ernet.in} 


\maketitle
\thispagestyle{empty}

\section{Summary}\label{S:Summary}
While Transaction Cost Economics (TCE) has been and continues to be the predominant theory used to explain vertical integration in firms, the four articles we review this week each explores a certain nuance to the TCE in a particular empirical context. \cite{Nickerson2003} are concerned with the question of why and how organizations change. They note that prior organization theorists have, on the one hand assumed an adaptation based view of organizational change  where boundedly rational agents shape the organization toward most efficient organization form possible as explained by TCE theory. One the other hand, organization theorists have assumed selection-based approaches that contend that organizations rarely change inefficient governance mechanisms.The authors claim to bridge this divide with this article. With detailed data on large firms in the for-hire trucking industry, the authors demonstrate that adjustment costs inhibit the extent or organizational adaptation toward the efficient boundary prescribed by TCE theory.

Consistent with TCE theory, the authors recognize that costs vary with the characteristics of transactions. The for-hire trucking industry was a well selected empirical context to explore the question of how organizations change because the unexpected deregulation of the industry in 1980 threw up a variation in organizations\textquotesingle  \  responses that helped in dealing with the problem of unobserved heterogeneity that might have otherwise been an issue. Additionally, the for-hire trucking industry also provides two additional advantages given the authors\textquotesingle \ emphasis on asset specificity in their theorization: First the highly asset specific nature of the LTL (Less-than-truckload) business makes the measurement and exposition of the relationship between asset specificity of assets owned and adaptability much easier to demonstrate. Second, the fact that the for-hire trucking industry had itself another line of business,  TL (Truckload)  which is not asset specific but similar in other respects helps in providing necessary variance in the data but controlling for other factors that are outside the scope of the particular research question considered.

\section{Extension}\label{S:Extension}
By categorizing the problem of adaptation in the framework of content and process theories rather than stick to just one, the authors set a higher bar toward explaining the phenomenon. I would build on top of the authors\textquotesingle\  conclusion that while transaction cost economizing is desirable, that the decision does need to be balanced from costs of adjustment by constructing a three way model that studies the simultaneous effect of change in organizational features, performance and adaptation amongst a larger sample of firms across industries so as to be representative of the larger population of organizations.  I anticipate that such an empirical analysis would continue to emphasize the significance of transaction costs, but also simultaneously increase the significance of organizational features and performance outcomes on adaptation. In the spirit of Nickerson and Silverman, the quantification of the direction of the outcome would be another step forward.

\section{Connecting the Readings}\label{S:Conclusion}
The other three articles in this weeks\textquotesingle\ readings point toward other variables worth considering. \cite{Mayer2004} suggest that contracts serve as a repository of knowledge and that organizations learn to contract better with time. Thus, while the effects of the learning are in the direction predicted by TCE, the learning itself is consistent more with evolutionary and behavioral theories. \cite{Kalnins2004} use the context of pizza restaurant franchises to highlight the importance of local knowledge that is tacit, and that may therefore not be dispensed with. \cite{Novak2009} emphasize interdependence among various vertical integration decisions rather than seeing them as separate transactions. Therefore, while much of the Vertical Integration literature continues to be reliant on TCE, within the four readings here we observe that four different constructs - learning, tacit local knowledge, interdependence, adjustment costs - each help better explain the phenomenon under different conditions. A more complete understanding would therefore require a framework that allows for many of the above constructs to be included, but for only a subset of them to actually influence the outcome of vertical integration at any time.

\bibliography{/Users/aiyenggar/OneDrive/code/bibliography/ae,/Users/aiyenggar/OneDrive/code/bibliography/fj,/Users/aiyenggar/OneDrive/code/bibliography/ko,/Users/aiyenggar/OneDrive/code/bibliography/pt,/Users/aiyenggar/OneDrive/code/bibliography/uz} 
\bibliographystyle{apalike}

\end{document}
