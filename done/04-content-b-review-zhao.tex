%04-content-b-review-zhao.tex
\documentclass[12pt]{article}
\usepackage{times}
\usepackage{amssymb,latexsym}
\usepackage[round,sort]{natbib}
\usepackage{fancyhdr}
\usepackage{lastpage}
\usepackage[T1]{fontenc}
\usepackage{mathptmx}
\usepackage{tabu}
\usepackage{textcomp}
\usepackage{stata}
\usepackage{listings}
\usepackage[a4paper]{geometry}
\usepackage{longtable}
\geometry{
 total={170mm,257mm},
 left=20mm,
 top=20mm,
 bottom=20mm,
}
\setlength{\headheight}{12pt}
\newenvironment{hypothesis}{
  	\itshape
  	\leftskip=\parindent \rightskip=\parindent
  	\noindent\ignorespaces}
	
\pagestyle{fancy}
\fancyhf{}
\fancyhead{}
\fancyfoot{}
\lhead{Review: \cite{Zhao2006}}
\rfoot{Page \thepage  \ of \pageref{LastPage}}
\rhead{Ashwin Iyenggar (1521001)}

\begin{document}
\title{Review  : \cite{Zhao2006}\\Conducting R\&D in Countries with Weak Intellectual Property Rights Protection}
\author{Ashwin Iyenggar  (1521001) \\ ashwin.iyenggar15@iimb.ernet.in} 


\maketitle
\thispagestyle{empty}

\section{Summary}
Whereas the comparative advantage theory suggests that firms should locate their R\&D in locations that provide a comparative advantage, \cite{Zhao2006} provides a contradicting view suggesting that multinational enterprises may benefit from conducting R\&D in countries with weak IPR protection by  making up for the weaker IPR protection through better internal organization.

\section{Theory}
\cite{Zhao2006} argues that competing firms may have a lower ability to imitate when the value of technology is highly dependent on the proprietary firm\textquotesingle s internal resources. We contend that that assumption is weakened significantly if the knowledge of this technology is codified and published, and when published work on related technologies cite a common prior art. \cite{Zhao2006} herself cites \cite{Kogut1993} suggesting that difficult to codify knowledge lends itself to more efficient transfer within the firm. Drawing on \cite{Cohen2000}, we thereby conclude that firms would have a greater incentive to keep such highly dependent technology developed in weaker IPR countries secret, rather than make this knowledge public. While \cite{Zhao2006} highlights the strength of internal firm linkages in identifying and appropriating the knowledge generated in weak IPR location subsidiaries, she does not emphasize the importance of secrecy. Indeed, she uses patenting data from weaker IPR location offices to substantiate her hypothesis, which we believe actually weakens her argument for the strength of internal firm linkages.

\section{Methods}
We build on the argument made in the previous section by highlighting the paradox in the view that highly dependent components capture a lower value. To be able to claim that this is indeed the case, one has to demonstrate that the development of highly connected components does not require leakage of knowledge about the connected components. However \cite{Zhao2006} offers us with no such evidence. Additionally, if such were indeed the case, the followup question would be if firms would be motivated to patent such work rather than keep it secret. We find weak evidence or argument to be convinced that firms intent on internalizing knowledge would file patents on such technology and disclose the connections with other components. Instead we believe that that would that make imitating the larger technology easier. We are therefore unsure if the IPR environment is as weak as the author is suggesting.

\section{Results}
We ask here, the question about the direction of causality. Do firms develop better internal linkages to tap value from global subsidiaries or do you firms decide to outsource R\&D to a cheaper location because of existing internal linkages? To be able to demonstrate this conclusively, we would need to show that a significant proportion of multinational firms entered weak IPR countries for R\&D work directly, rather than grow into R\&D from prior work in the country. Our counter to the view presented by \cite{Zhao2006} is that firms might have expanded to weak IPR protection locations due to lower costs at higher scale, and that as the IPR protections began to improve, they moved some of their R\&D work there. Since she does not sample all highly dependent, lower individual value components and show that a significant number of them were moved to weak IPR protection countries but rather limited her sample to innovations from weaker IPR protection locations, her sample could be infected with a selection bias. Her results would be significantly weakened If only a few of the very highly dependent components move offshore.

Additionally, we find that the lack of significance for \verb|weak| in the cross-firm regressions potentially problematic as it takes away the primary result from the intra-firm regressions. While this may partly be due to the omitted controls for firm heterogeneity, much of the effect seems to have been explained by \verb|f_weak|. Table 2 in \cite{Zhao2006} displays an order of 10 difference between the number of patents developed in weak IPR countries and those developed in strong IPR foreign countries. This difference extends to a factor of 100 in the comparison between patents developed in weak IPR countries and those developed in the home country (US). This huge disparity raises questions about if the power of the test is itself suggestive of the statistical significance found.

\section{Extensions}
One  may need to nuance the argument between the extent of weakness of IPR protection and the cost/scale advantage obtained by firms operating in such locations. It may well be possible that these countries may be strengthening their IPR regime and that this, combined with lower costs and availability of talent has influenced firms to move R\&D offshore. The conclusions from \cite{Zhao2006} may be strengthened by rejecting the preceding hypothesis or vice versa. 

\section{Connections across readings}
\cite{Alcacer2006} makes a counter argument to \cite{Zhao2006} just months later, arguing that R\&D subsidiaries are indeed more concentrated and that firms tradeoff competition costs with agglomeration benefits. \cite{Pant2012} reverse the gaze and suggest that developing country multinationals work toward overcoming disadvantages due to country of origin in successfully executing their globalization strategy. \cite{Kostova1999} then consider question of the legitimacy among multinationals in general, and specifically how legitimacy spillovers occur over time. Finally, \cite{Chang2001} lend credibility to our argument that there may be a significant path dependence effect to globalization choices.

\bibliography{/Users/aiyenggar/OneDrive/code/bibliography/ae,/Users/aiyenggar/OneDrive/code/bibliography/fj,/Users/aiyenggar/OneDrive/code/bibliography/ko,/Users/aiyenggar/OneDrive/code/bibliography/pt,/Users/aiyenggar/OneDrive/code/bibliography/uz} 
\bibliographystyle{apalike}

\end{document}
