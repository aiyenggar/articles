\documentclass{beamer}
\usetheme{Singapore}
\usepackage[round,sort]{natbib}
\usepackage{tikz}
\usetikzlibrary{arrows,decorations.pathmorphing,backgrounds,fit,positioning,shapes.symbols,chains}
\usepackage{adjustbox}

\title{Institutions, Innovation \& Growth}
\subtitle{A Review of Readings}
\author{Ashwin Iyenggar}
\institute[Indian Institute of Management Bangalore] 
{
  Corporate Strategy and Policy\\
  Indian Institute of Management Bangalore
}
\date{21 January, 2017}
\subject{Review of Assigned Readings on Institutions, Innovation \& Growth}

% \pgfdeclareimage[height=0.5cm]{university-logo}{university-logo-filename}
% \logo{\pgfuseimage{university-logo}}

\AtBeginSubsection[]
{
  \begin{frame}<beamer>{Outline}
    \tableofcontents[currentsection,currentsubsection]
  \end{frame}
}

\begin{document}

\begin{frame}
  \titlepage
\end{frame}

\begin{frame}{Outline}
  \tableofcontents
  % You might wish to add the option [pausesections]
\end{frame}

\section{Overview}
\begin{frame}{Institutions, Innovation \& Growth}{}
\begin{itemize}
\item{\cite{Ogilvie2014}} - Using historical evidence to point to weaknesses in understanding effect of institutions on economic growth
\item{\cite{Acemoglu2005}} - Differences in economic institutions lead to differences in economic development
\item{\cite{Acemoglu2003}} - Empirical study of role of property rights institutions and contracting institutions on economic growth
\end{itemize}
\end{frame}

\section{\cite{Ogilvie2014}}
\begin{frame}{Historical Perspective}{Stylized Facts}
\begin{itemize}
\item{Private-order institutions vs Public-order institutions}
\item{Parliaments representing wealth holders}
\end{itemize}
\end{frame}

\begin{frame}{Lessons from History}{Examples}
\begin{itemize}
\item{Necessity of public-order institutions for markets to function}
\item{Maghribi traders supposed to have coalition based on Jewish, due to lack of effective legal system}
\item{Reputational mechanisms buttressed by public order institutions}
\item{Champagne fairs supposed that  private law-courts intermediated by law merchants}
\item{No evidence that any of the tribunals applied a private, merchant-generated law-code}
\item{Public-order institutions implications for economic growth is a two-edged sword}
\item{Role may be complementing not substituting}
\end{itemize}
\end{frame}

\begin{frame}{Lessons from History}{Examples}
\begin{itemize}
\item{Generalized (rules apply uniformly) vs Particularized (rules apply differently) Institutions}
\item{Wealth holders with coercive privileges used rents to obtain representation in parliament}
\item{At the expense of the rest of the economy. Ex: Corn laws, p. 424}
\item{Glorious Revolution of 1688 and economic discontinuity}
\end{itemize}
\end{frame}

\begin{frame}{Lessons from History}{Examples}
\begin{itemize}
\item{Linking of property rights institutions and contracting institutions}
\item{Considerations of insecurity - impact on economic growth}
\item{Generalized property rights are better}
\item{Degree of security of private property rights}
\item{Embeddedness of institutions - variation across societies and time-periods}
\item{Serfdom - implications for economic growth is divided}
\end{itemize}
\end{frame}

\section{\cite{Acemoglu2005}}
\begin{frame}{Institutions and Long Run Economic Growth}{Summary}
\begin{itemize}
\item{Institutions provide incentives and constraints to economic growth}
\item{Political power - de jure and de facto}
\item{Allocate political power to groups with interests in broad-based property rights enforcement}
\item{Constraints required on power holders}
\item{Power holders can capture rents}
\end{itemize}
\end{frame}

\begin{frame}{Income Differences}{Causes}
\begin{itemize}
\item{Institutions, Geography, Culture}
\item{Korean Example}
\item{Colonialization Example (pg. 412-413, 415)}
\item{The efficient institutions view - political Coase theorem}
\item{The ideology view}
\item{The incidental institutions view}
\item{Social conflict view}
\end{itemize}
\end{frame}

\begin{frame}{A theory of political institutions}{List}
\begin{itemize}
\item{Individuals have preferences over economic institutions}
\item{Disagreement in people's preferences}
\item{Commitment problem and inseparability of efficiency and distribution}
\item{Power determines equilibrium structure}
\item{Political power may be de jure or de facto}
\item{De facto power is influenced by distribution of resources in society}
\item{Endogeneity of political institutions}
\end{itemize}
\end{frame}

\section{\cite{Acemoglu2003}}
\begin{frame}{Unbundling Institutions}{Summary}
\begin{itemize}
\item{Proxy for property rights: Political Risk Services assessment of protection against government expropriation in a country, Polity IV's constraint on executive measure}
\item{Proxy for cost of enforcing private contracts: legal formalism data from Djankov et. al (2002, 2003)}
\item{Exogenous sources of variation - legal origin, mortality rates}
\item{Basic specification p. 14: Endogeneity, Measurement Error}
\item{multiple IV specification p. 16}
\item{Main results: Table 4, Table 5, Table 6, Table 7}
\item{Contracting institutions and legal rules have some effect of form of finance, and form of business regulation, but no effects on major economic outcomes}
\item{Legal rules and procedures affect contracting relationship}
\item{Property rights institutions have a major influence of long-run economic growth}
\end{itemize}
\end{frame}


\bibliography{/Users/aiyenggar/OneDrive/code/bibliography/ae,/Users/aiyenggar/OneDrive/code/bibliography/fj,/Users/aiyenggar/OneDrive/code/bibliography/ko,/Users/aiyenggar/OneDrive/code/bibliography/pt,/Users/aiyenggar/OneDrive/code/bibliography/uz}
\bibliographystyle{apalike}

\end{document}
