\documentclass{beamer}
\usetheme{Singapore}
\usepackage[round,sort]{natbib}
\usepackage{tikz}
\usetikzlibrary{arrows,decorations.pathmorphing,backgrounds,fit,positioning,shapes.symbols,chains}
\usepackage{adjustbox}

\title{Innovation, Economic Growth, and the Way Forward}
\subtitle{A Review of Readings}
\author{Ashwin Iyenggar}
\institute[Indian Institute of Management Bangalore] 
{
  Corporate Strategy and Policy\\
  Indian Institute of Management Bangalore
}
\date{14 January, 2017}
\subject{Review of Assigned Readings on Rate of Economic Growth and Innovation}

% \pgfdeclareimage[height=0.5cm]{university-logo}{university-logo-filename}
% \logo{\pgfuseimage{university-logo}}

\AtBeginSubsection[]
{
  \begin{frame}<beamer>{Outline}
    \tableofcontents[currentsection,currentsubsection]
  \end{frame}
}

\begin{document}

\begin{frame}
  \titlepage
\end{frame}

\begin{frame}{Outline}
  \tableofcontents
  % You might wish to add the option [pausesections]
\end{frame}

\section{Overview}
\begin{frame}{Economic Growth and Innovation}{}
Readings this week
\begin{itemize}
\item{\cite{Durlauf2005}} - Econometric tools used in studying economic growth
\item{\cite{Abramovitz1993}} - Challenges in understanding the sources of economic growth
\item{\cite{Young1994}} - Total Factor Productivity in East Asian Economies
\end{itemize}
\end{frame}

\section{\cite{Durlauf2005}}
\begin{frame}{Growth Econometrics}{Stylized Facts}
\begin{itemize}
\item{GDP per woker, Table 1, page 564}
\item{Growth miracles and disasters, Table 2-3, page 566-567}
\item{Does past growth predict future growth?, Table 4}
\item{Differences by region, Table 5, Page 572}
\end{itemize}
\end{frame}

\begin{frame}{Growth Econometrics}{Stylized Facts}
\begin{itemize}
\item{Most countries have grown richer, but income disparities remain}
\item{Past growth is a weak predictor of future growth, it is slowly}
\item{Growth slowdown is observed throughout most of the income distribution.}
\item{Convergence hypothesis: efects of initial conditions eventually disappear.}
\end{itemize}
\end{frame}

\begin{frame}{Growth Econometrics}{Statistical Tools}
\begin{itemize}
\item{Beta convergence: Choice of variables, Identification and non-linearity, Endogeneity, Measurement error, Effects of linear approximation}
\item{Distributional approaches: sigma-convergence, ...}
\end{itemize}
\end{frame}

\begin{frame}{Growth Econometrics}{Statistical Issues}
\begin{itemize}
\item{Outliers}
\item{Measurement Error}
\item{Missing Data}
\item{Heteroskedasticity}
\item{Cross-seciton error correlation}

\end{itemize}
\end{frame}

\begin{frame}{Growth Econometrics}{Limitations}
\begin{itemize}
\item{Model Uncertainty - identification of empirically salient determinants of growth when the range
of potential factors is large relative to the number of observations}
\item{Standard inference procedures based on a single model can grossly overstate the precision of inferences about a given phenomenon}
\item{Cross country data}
\end{itemize}
\end{frame}

\section{\cite{Abramovitz1993}}
\begin{frame}{The Search for the Sources of Growth}{Summary}
\begin{itemize}
\item{Standard growth accounts of economic historians is misleading}
\item{Abramovitz (1955): per capita input of labor and capital accounted for 10\% growth of net output per capita. Residual = Technological Progress?}
\end{itemize}
\end{frame}

\begin{frame}{The Search for the Sources of Growth}{Issues with the Residual}
\begin{itemize}
\item{Measure of factor inputs was incomplete}
\item{Intangibles - education, on-the-job training, R\&D were neglected}
\item{Other missing categories of business spend}
\item{Denison showed a) neglected elements, b) effect of residual sharply lower}
\end{itemize}
\end{frame}


\begin{frame}{The Search for the Sources of Growth}{Discussion}
\begin{itemize}
\item{Standard growth accounting - sources of growth operate independently of one another}
\item{Schumpeter - net capital accumulation would fall to zero in the absence of innovation}
\item{Nelson 1964 - embodiment question - by reducing the age of capital stock}
\item{David and Abramovitz - capital using technological progress increases demand for capital relative to labor}
\item{Arrow - learning by doing}
\item{Rosenberg - learning by using}
\end{itemize}
\end{frame}


\begin{frame}{The Search for the Sources of Growth}{Discussion}
\begin{itemize}
\item{Comparison of 19th and 20th century character of economic growth}
\item{Growth of capital intensity was larger source of labor productivity growth in 19th century than in the 20th}
\item{Rates of total factor productivity was low}
\item{Dilemma: If labor quality, urban migration and scale contributed to growth, there is little room for technological progress. If you assume technological progress, there is no room for education, better resource allocation}
\item{19th century - physical capital heavy, which weakened in the 20th century}
\item{distinguish raw labor from labor augmented by education and investment in knowledge}
\end{itemize}
\end{frame}


\begin{frame}{The Search for the Sources of Growth}{Discussion}
\begin{itemize}
\item{Tangible Capital vs Intangible Capital}
\item{Esteemed economists held: economies of scale = technological progress}
\item{Urbanization, Immigration added scale to economy}
\item{Long period estimates - advantages and disadvantages}
\end{itemize}
\end{frame}

\begin{frame}{The Search for the Sources of Growth}{Discussion}
\begin{itemize}
\item{Technology dependent rubric of capital accumulation}
\item{Shift toward intangible capital}
\item{Rise in educational level, investment in R\&D}
\item{School levels rose due to human capital using bias in the composition of output}
\item{Shift from blue collar to white collar work}
\end{itemize}
\end{frame}

\begin{frame}{The Search for the Sources of Growth}{Summary}
\begin{itemize}
\item{Interdependence of proximate sources of economic growth}
\item{Interdependence runs both ways - largely ignored}
\end{itemize}
\end{frame}

\section{\cite{Young1994}}
\begin{frame}{The Tyranny of Numbers}{Summary}
\begin{itemize}
\item{How does factor accumulation explain post war growth in east Asia}
\item{Output growth and Manufacturing growth is unprecedented}
\item{Total factor productivity growth is similar to OECD nations}
\end{itemize}
\end{frame}

\begin{frame}{The Tyranny of Numbers}{Countries}
\begin{itemize}
\item{Hong Kong}
\item{Singapore}
\item{South Korea}
\item{Taiwan}
\end{itemize}
\end{frame}

\begin{frame}{The Tyranny of Numbers}{Summary}
\begin{itemize}
\item{Premise that productivity growth in Asian countries has been extraordinarily high is incorrect}
\item{Increases in output brought through rise in participation rates, investment to GDP ratios, educational standards and transfer of labor from agriculture to other sectors}
\item{South Korea}
\item{Taiwan}
\end{itemize}
\end{frame}



\bibliography{/Users/aiyenggar/OneDrive/code/bibliography/ae,/Users/aiyenggar/OneDrive/code/bibliography/fj,/Users/aiyenggar/OneDrive/code/bibliography/ko,/Users/aiyenggar/OneDrive/code/bibliography/pt,/Users/aiyenggar/OneDrive/code/bibliography/uz}
\bibliographystyle{apalike}

\end{document}
