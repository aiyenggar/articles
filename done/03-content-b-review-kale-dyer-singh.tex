%03-content-b-review-kale-dyer-singh.tex
\documentclass[12pt]{article}
\usepackage{times}
\usepackage{amssymb,latexsym}
\usepackage[round,sort]{natbib}
\usepackage{fancyhdr}
\usepackage{lastpage}
\usepackage[T1]{fontenc}
\usepackage{mathptmx}
\usepackage{tabu}
\usepackage{textcomp}
\usepackage{stata}
\usepackage{listings}
\usepackage[a4paper]{geometry}
\usepackage{longtable}
\geometry{
 total={170mm,257mm},
 left=20mm,
 top=20mm,
 bottom=20mm,
}
\setlength{\headheight}{12pt}
\newenvironment{hypothesis}{
  	\itshape
  	\leftskip=\parindent \rightskip=\parindent
  	\noindent\ignorespaces}
	
\pagestyle{fancy}
\fancyhf{}
\fancyhead{}
\fancyfoot{}
\lhead{Review: Kale, Dyer and Singh (2002)}
\rfoot{Page \thepage  \ of \pageref{LastPage}}
\rhead{Ashwin Iyenggar (1521001)}

\begin{document}
\title{Review\footnote{I have been unable to incorporate the learnings received from the feedback for the previous submission while writing this review. I will do so from the next written submission}  : Kale, Dyer \& Singh (2002)\\Alliance Capability, Stock Market Response, And Long-Term Alliance Success}
\author{Ashwin Iyenggar  (1521001) \\ ashwin.iyenggar15@iimb.ernet.in} 


\maketitle
\thispagestyle{empty}

\section{The Alliances Landscape}
The five research articles this week on strategic alliances highlight that firms enter into alliances with other firms on various contractual terms and for various reasons. While the researcher has a much larger data set of alliances available for the last twenty-five years, the topic has been a fertile ground for several research studies.  In Table ~\ref{table:theoreticallens}, we capture the various theoretical lenses adopted by scholars (among the assigned readings this week), and the stage of the alliance that they focus on. We note that the phenomenon of inquiry (alliances) has not been dominated by a single strong paradigm. Therefore, the area continues to serve as a fertile ground for ongoing research. We specifically highlight the potential for portfolio studies as most previous studies  have focussed on single alliances \citep{Wassmer2010, Kale2009}

\begin{center}
\begin{longtable}{|p{0.20\textwidth}|p{0.30\textwidth}|p{0.25\textwidth}|p{0.10\textwidth}|}
\hline \textbf{Article}&\textbf{Theoretical Lens}&\textbf{Stage of Focus}&\textbf{Portfolio}\\\hline
\endfirsthead
\hline \textbf{Article}&\textbf{Theoretical Lens}&\textbf{Stage of Focus}&\textbf{Portfolio}\\\hline
\endhead

\cite{Broschak2014}&Transaction Cost Economics&Alliance Dissolution&Single\\\hline
\cite{Gulati1995}&Social Network Theory&Alliance Formation&Single\\\hline
\cite{Kale2002}&Dynamic Capabilities, Knowledge Based View, Evolutionary Economics&Alliance Performance&Portfolio\\\hline
\cite{Lavie2006}&Behavioral Theory (Exploration/Exploitation)&Alliance Formation&Single\\\hline
\cite{Zollo2002}&Evolutionary Economics&Alliance Performance&Single\\\hline

\caption{Theoretical Lens and Focus of Analysis Among Articles}
\label{table:theoreticallens}\\
\end{longtable}
\end{center}

\section{Review of \cite{Kale2002}}
\cite{Kale2002} note that while alliances have been understood to be able to create value for firms undertaking the alliances, that roughly half of all alliances end in failure. The article has attempted to answer the question of how firms may maximize their probability of success in alliances. Specifically, \cite{Kale2002} identify that while prior research has demonstrated that experience matters in strategic alliances, literature had not answered the question of how prior experience in alliances translated into a capability to succeed at alliances.

\cite{Kale2002} make a significant contribution by overturning the (then) existing understanding of the role of alliance experience as variable determining alliance performance (or success). Since \cite{Gulati1995}, the received wisdom was that firms perform better at alliances with greater alliance experience (as measured by the count of the number of alliances entered into by the firm). \cite{Kale2002} distinguish alliance capability from alliance experience and demonstrate that when self selection is excluded, the effect of alliance experience of either stock market performance, or manager perception of long term performance  is insignificant.

On the other hand, \cite{Kale2002} demonstrate that alliance capability as measured by the presence of a dedicated alliance function within the firm is a strong predictor of alliance performance (short term stock market return), and alliance success (long term managerial perception of alliance success).

We believe that a significant contribution of \cite{Kale2002}  was that it was able to empirically demonstrate that the stock market was doing a good job of predicting the long term success of an alliance, and thereby supported the efficient market hypothesis.

\section{Extension}
Much alliance literature has focussed on a single alliance as the subject of study. As \cite{Kale2009} mention, the phenomenon in several industries is that firms manage multiple alliances (also known as an alliance portfolio) simultaneously. We would assume that such a portfolio of alliances may be characterized by the presence of significant tacit knowledge that the market may be unable to value accurately despite the presence of a designated alliance function within the firm. We therefore suggest that the work of \cite{Kale2002} be extended to a portfolio of alliances and the effectiveness of market valuations to alliance announcements be tested in that context. We expect that the results would be significantly weaker than that obtained in \cite{Kale2009}.

\bibliography{/Users/aiyenggar/OneDrive/code/bibliography/ae,/Users/aiyenggar/OneDrive/code/bibliography/fj,/Users/aiyenggar/OneDrive/code/bibliography/ko,/Users/aiyenggar/OneDrive/code/bibliography/pt,/Users/aiyenggar/OneDrive/code/bibliography/uz} 
\bibliographystyle{apalike}

\end{document}
