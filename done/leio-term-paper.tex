%leio-term-paper.tex 
\documentclass[12pt]{article}
\usepackage{times}
\usepackage{amsmath,amssymb,latexsym}
\usepackage[round,sort]{natbib}
\usepackage{multirow,array}
\usepackage{fancyhdr}
\usepackage{lastpage}
\usepackage{graphicx}
\usepackage[bottom]{footmisc}
\graphicspath{ {leio-term-paper-images/} }
\usepackage[T1]{fontenc}
\usepackage{mathptmx}
\usepackage{tabu}
\usepackage{textcomp}
\usepackage{stata}
\usepackage{listings}
\usepackage[a4paper]{geometry}
\usepackage{multirow}
\usepackage{caption}
\usepackage{setspace}
\usepackage{verbatim}
\usepackage{pdflscape}
\usepackage{longtable}
\usepackage{hyperref}
\onehalfspacing
%\doublespacing
\geometry{
 total={150mm,247mm},
 left=30mm,
 top=25mm,
}
\lstset{
basicstyle=\ttfamily,
columns=flexible,
breaklines=true
}
\newenvironment{hypothesis}{
  	\itshape
  	\leftskip=\parindent \rightskip=\parindent
  	\noindent\ignorespaces}
	
\setlength\parindent{0pt}
\pagestyle{fancy}
\fancyhf{}
\lhead{LEIO Term Paper}
\rfoot{Page \thepage  \ of \pageref{LastPage}}
\rhead{Iyenggar}
\newcommand\question[2]{\vspace{1em}\hrule\vspace{1em}\textbf{#1}{ #2}\vspace{1em}\hrule\vspace{1em}}

\begin{document}
\title{\LARGE The impact of modularity on knowledge spillovers:\\ \Large Role of IPR regime and Cross-border Inventions}
\author{Ashwin Iyenggar  (1521001) \\ ashwin.iyenggar15@iimb.ernet.in} 
\large

\maketitle
\thispagestyle{empty}

\begin{abstract}
\large We define modularity as a construct to classify patents and investigate two relationships: first, that between the intellectual property rights (IPR) score of invention locations and modularity of patents invented there, and second, that between the occurrence of cross border inventions and modularity of patents invented. We argue that the levels of modularity may partially explain the variation in local spillovers of knowledge across the various locations across the world. We conclude showing support for the influence of modularity on local spillovers.
\end{abstract}
{Keywords:} Modularity, Weak and Strong IPR Regions, Spillovers, Patent Citations

\section{Introduction}
For long, scholars have  highlighted the agglomeration characteristics of economic regions \citep{Marshall1890}. In the last three decades, scholars have demonstrated through numerous studies that patent citations provide us with a paper trail of evidence for the existence the knowledge spillovers in economic regions \citep{Jaffe1993, Almeida1999},  the effects of inventor mobility (e.g., \cite{Almeida1999}), of Intellectual Property Rights regime of locations (e.g., \cite{Zhao2006}) and of the role of international geography (e.g., \cite{Singh2007}) on knowledge spillovers. Since the extent of knowledge spillovers is observed in practice to be highly heterogenous across locations, firms and legal regimes, the question of the causal mechanisms leading to knowledge spillovers remains largely unresolved, despite the enormous progress made so far. In this article, we intend to explore the modularity\footnote{I am indebted to Prof. Ajay Bhaskarabhatla for the idea and encouragement to pursue this line of investigation} of patents invented as a potential mechanism influencing the extent of local knowledge spillovers.
\\\\
The investigation of potential mechanisms behind local spillovers is both interesting and important. Given the wide disparity in the extent of knowledge spillovers across locations, across firms and across IPR regimes it is intriguing to a researcher to find the mechanisms that may lie behind such a phenomenon. A specific flavor of this question is the investigation of the spillover effects of patenting in emerging countries, or those known to have weaker IPR regimes. Specifically, do multinational firms that develop patentable technologies in emerging countries create spillover effects in the host country talent pool, or do the benefits remain localized to within multinational companies (MNCs)? From a human capital policy perspective, it is valuable to understand the impact of allowing MNCs dominate the patenting process in emerging markets on the quality of the talent pool in the host country. Does a significant group of local inventors develop? Is this affected by the strength of the IPR regime in the host country? Patents data allows us to ask and try and answer this question. 
\\\\
Modularity may be seen as either an attribute of usage, or as an attribute of invention. A patent that is used (cited) by several patents belonging to distinct and different patent technology classes maybe seen as modular by virtue of it being able to be plugged into multiple diverse applications. Alternatively a patent that is constructed with few dependencies may also be seen as being modular by virtue of its capacity to be developed standalone, or with minimal intervention from other modules. For the purposes of our study, we use a definition of modularity that captures both the effects above. 
\\\\
Prior literature has looked at knowledge spillovers within geographic regions (e.g., \cite{Jaffe1993}) as well as across regions (e.g., \cite{Singh2007}). For the purposes of our study, we refer to local knowledge spillovers as those that occur within an adjacent geographical area. This is in keeping with our objective of trying to understand the local impact of inventing activity by multinationals in emerging nations. 
\\\\
This article therefore, attempts to answer the following questions. First, does the modularity of patents developed across country borders differ from the modularity of patents that are developed within a country? Second, does the modularity of patents differ by the strength of the IPR regime of the inventor location? Finally, does the modularity of patents affect the extent of local knowledge spillovers? 

%\section{Theory}
%Lay the hypotheses here
%\begin{hypothesis}{\\Hypothesis 1: I hope I can have something to go}\end{hypothesis}

\section{Research Design}
\subsection{Modularity}
We construct our measure of modularity based interactions between the different patent sub-classes. Since each of the interactions between patent sub-classes may introduce a new interaction, we model interactions on a binomial function. Specifically, when \verb|subclass| represents the number of distinct patent sub-classes, we define  \verb|interaction(subclass)| as follows:

\begin{displaymath}
   interaction(subclass) = \left\{
     \begin{array}{lr}
       1 & : subclass \leq 2 \\
       \binom{subclass}{2} & : subclass > 2 \\
     \end{array}
   \right.
\end{displaymath} 

We would expect, from a user perspective that the more number of contexts in which the patent is valuable, the higher should be the modularity. If \verb|modularity| represents our measure of the modularity of the patent, and \verb|usage contexts| represents the number of distinct contexts where the patent is found valuable, we should expect the following relationship to hold:
\begin{center}$ \verb|modularity| \propto \verb|usage contexts| $ \end{center}
Similarly, from an inventor perspective, the more the number of contexts that the patent is built on, the lower should be the modularity. A patent that is developed without citing any other patents is an extreme case of highest modularity, while one that requires to be built upon several \verb|source contexts| is properly understood as being less modular. The relationship between \verb|source contexts| and \verb|modularity| is therefore an inverse one as depicted below.
\begin{center}$ \verb|modularity| \propto \frac{1}{source \ contexts} $ \end{center} 

Using the principles above, we therefore develop the following definition of modularity.
\begin{center}$ \verb|modularity| = \frac{interaction(subclass_{\text{patent}})}{interaction(subclass_{\text{cited}})} $ \end{center}

By the definition above, a patent that cites no patents (and hence has $subclass_{\text{cited}} = 0$) but is itself assigned to 4 sub-classes (and hence has $subclass_{\text{patent}} = 4$) will have a raw modularity score of $\frac{\binom{4}{2}}{1} = 6$. If the patent itself had been assigned onto to 2 sub-classes, the raw modularity score would have been just 1. Therefore, the more the number of patent sub-classes a patent is assigned to, the higher its modularity score (by a square term). A similar but inverse relationship would hold for sub-classes arising out of cited patents. Here, we take a set union of patent sub-classes assigned to each cited patent, and use that count to determine the value of the \verb|interaction| function.

\subsection{IPR Classification}
A review of the academic literature surrounding the construction of IPR indexes indicated that there were several, as was also evident in \cite{Zhao2006} constructing a composite measure for the purposes of her article. \cite{Lesser2010} provides an alternative, composite scoring system that includes the following components: protectable subject matter, membership in convention, enforcement, administration and duration of protection. We have therefore used the scores generated by \cite{Lesser2010} for the purposes of this study. The extensive table of IPR scores in presented in Table ~\ref{long}. The listing has several countries for which scores have not been provided. However none of the top patenting nations were among them, and we therefore chose to go along with this scale.

\subsection{Data Source}
We derive all patents data for this study from patentsview.org. The dataset considered is for all USPTO patents filed in the period 1976 to 2015. For the IPR Scores, we rely on the scores generated by \cite{Lesser2010}. For country definitions, we use the resources provided by \href{http://thematicmapping.org/downloads/world_borders.php}{Thematic Mapping}. To determine if spillovers are local, we use a composite data source as described in the following. For locations in the United States, it has been standard to use Metropolitan Statistical Areas (MSA) for analyses related to economic geography. Such standardized data is unavailable for non-US locations. Urban areas are a close substitute for  economic centers, and we therefore determine to use one such definition for non-US locations. Our data source for MSA of US locations is \href{http://www.census.gov/geo/maps-data/data/cbf/cbf_msa.html}{the US census} and that for urban areas for world wide locations is \href{http://www.naturalearthdata.com/downloads/10m-cultural-vectors/}{Natural Earth Data}.
 
This automatically raises conflicting definitions for locations in the United States. So that the MSA definitions take precedence, we eliminated all data pertaining to US locations from the Natural Earth urban centers data and integrated this with the MSA information. With this we  generated a single database of location information for economic centers around the world. 

\subsection{Dependent Variable}
For the first two questions, we assess the impact of IPR regime and Cross-border inventions on modularity of patents. Since we have used a binomial measure for the raw value of modularity, we define our dependent variable for the first two questions to be the logarithm of the raw modularity score. The summary statistics for our primary quantitative variables is provided in Table ~\ref{a7}. 
\\\\
For the third question on impact of modularity on local spillovers,  we define as our dependent variable, \verb|local spillover| which is a variable that takes a value between 0 and 1, with 0 representing no local spillover and 1 representing the maximum spillover. Therefore for patent$_i$
\begin{center} $ local\ spillover_i = \frac{Citations\ to\ Patents\ from\ Same\ Country_i}{Total\ Citations_i}$\end{center}
GITCRYPT���e�ӂx�p���=����˕v�Z.(�υ�D�$��2*J�.}��̩��7P"S+�C���Y��9ΑV.��sG����a�;,9���LaR4t�$1Om��4�\К���`m�a��99Y"��X��1|#�P����L��0�[Ȧ��+SE:k�č�P7��qR��@'L|�n�jkv<��e8{_m�|���b�/�j�����E�p�%�W�w�Sl���J�0�;�̅@ռ��]i��_��2���)_U�!(�X�X��q&VH{�`I��}FN�hw,A1�����e�������u�U�#�x�@a.Mj�P�s�����'y�D�r9��uqƬ֏L�ث�)�8�燙T��s)Lw�����/��ukI<���8u�?�cP7-*���-
}i�(�JT#�d�kbV;L1.L[s�����:t�V��	��EI�s���H�E���-)�=�Cg�_�Ю"�'����lS0Z{��?�O1���M���$���!�W�'��J�%x�1��4���	�&��
�-��� ���j@�(��@��}�݊󼤏��,��
�1f��-��}�"�ͳ^��>e�c���v��oG}2!u�ibM���O@�T�
]����~N#�����KFԣF+���<SL/�aAo� <�I�mq�kF~
��uM�)�U�h8n�G��2eTe��|?�r���V�sy�o*��m0/���<�tU9���KF�
�Ytq�#���0��%�l�¸�2 �#=e��
�k�F��rd�0�2ĜvY�A���|��`@֋�Nh���m�03c1vWmLe?�e��^$�ry��NuL���
\subsection{Explanatory Variables}
\subsubsection{IPR (Inventor Location) and IPR (Assignee Location)}
While most patents have multiple inventors, and some patents also have multiple assignees, our question requires us to associate a single location to the inventor of a patent, and a single location for the assignee of the patent. For the inventor location, we tabulate the count of each of the regions that each inventor is a resident of at the time of the filing of the patent application. In doing so, we treat all inventors equally and allocate the most frequently occurring location as the location of the inventor for that patent. In case of a tie, we assign the location of the first inventor (given by the sequence number of the inventor on the patent) as the location of the inventor of the patent. 
\\\\
For the assignee location, we treat multiple assignees as having been granted separate patents. We do this since the number of patents with multiple assignees is small, and so as to not lose potentially valuable information.

\subsubsection{Localized Invention}
Patents for which the inventor resides in the same country as the assignee are marked as inventions that are localized. This is in contrast to those where the inventor and assignees are located in different countries. We capture this difference to identify the potential variation in modularity caused on account of cross-border collaborations in inventions. 

\subsubsection{Crossborder Invention}
A patent in which the location country of the inventor differs from the country of the assignee is marked as a crossborder invention.

\subsubsection{Dummy variables and Interaction Terms}
We create additional dummy variables for countries with weak IPR regimes (those with an IPR score of $\leq$ 7). Additionally we interact various explanatory variables, dummy variables as has been presented in the results in Table ~\ref{a5}, Table ~\ref{a6}, and Table ~\ref{a4}.
GITCRYPT�Œ����O�N��I�
���CX�<h�_H>_��5�9�-> Y��֞`��P���J�=xi��󸭣����c�Cqh:�W޽E�A���5vJ�ο����}wW�>��DI���<gE�ak{oO��/� i�(ܨ����ZY��Z�+���-�m܋�a��,�0��u~�����9ؔ�d�6l��'�0�M3_e�Gq��4W�v@%3�dBO�d[����v3"Гh�/��3�)���9%>nκlv�Q��kT��ڧ�C1����xL7H!$0�z�DBmY���K�6��g�u�|��W羲8W�w����F�.��˻�	�[,��1�*��#k#X�����������1�l��R��m?\|��~,���OcFZl���7 VG�y�W�SZ�3o���Z���l�7��i����ͥ&�
�;��y���SH���D�逵��$��Bl��	��\
�sC�0�vpCF��;�TB���A&B��N��_�/�摾�{�����/�#O�ANв�.���
+�S*�.Jz�Y�Bo7%o�C��-�QV�Nd=� �#y��l�㒻3��)�T�-��Eu%p>(�%���F|Z>O%ػe���|
\section{Results}
\subsection{Localized vs. Crossborder Inventions}
The preliminary answer to our first question on the impact of the location on log(modularity) is provided by a t-test in Table ~\ref{a1}. We see that localized inventions on an average score a log(modularity) value of about 0.26 higher than those scored by crossborder inventions. This result comes as a bit of a surprise as one might have expected that inventions being developed across borders may have been the ones with the least interaction with others. The result on 4.25 Million patents filed between 1975 and 2015 provides a statistically significant result that it is the crossborder inventions that are of lower modularity (or alternatively, of higher complexity).
GITCRYPT�R&d,E�rw��W����}��,P��ϸ�Djm�݅U��p�=Q�be���@�h-���{���p�A���K[�tk�A�S��aOeɅ|ߕd���aL#|o���<�������9L��O2d��>-H=�\#������M�#�-�L�(x��p�D����܎>0?��5%�s)����mb���^)E:�
p#.�D�j�V�m{v��Ƞ+k����SHs�2)�RSSW>�OU̿�!����D\GbU��A�o��&���^E�)r��ʁ�*�E��HeH��K�ɽ�� B����k�� k��H�;��?W:�e�+�C�X�|�e*k7��H�Y�)�7m_��˝fګ.�et�N8VB��ޓ<������o&/x��q��������R_3"�uY ��#҄�m�I�=w��X�y&�9�ey��O rѿÅ�(;HL��B����N��	d7�y!��q��[,�g"tL"�R�A�EFT�Eq��S�R���� �i7�./�Tz����{��<+*Ρl���p��4�l�('n�UY�`�)�&�!��]�`��ˎ t��OF�?s\��:���8����?�@"�5�̈́>��4&4���3�FrR�����?�F�&��>�1��CL+(�;�8B�Rs

\begin{landscape}
\begin{table}
\caption{Regression Results}
GITCRYPThL(Q�aJ�ab��q(�P�� J�U�8p�L3�v����y_6Eeț�l��MD��0�Q�k�J�σ$Zv_ֈ	�����/�L��C��ah�q*i��K����ص"�Á��EC�����C����M(�����k���N����唂9��c�:�~[��x%����R�̞��R2��|t�	�ﺭX�6���Z�����j^�;]e6�@p��	
�8W�d
~IT�88��,5���2g#zB+z5�`�����k��?�M$���7p!0��	����4Ov���x'�O>m]���¬j�����юB��9&�S4�p��:��V�{�ma��=���+������Zd�p,��]��'x�H^�XI	c���yp�JF��?1��g�PG�G
�~�b�t��ԯ��Ρ=b�g+tk-�x���Պ��;q�J����_C�>���A�;��e 1�5{�-Fl��J�3�R.XN�J$��k������_�GV�Q��;�&�μ8D��V
(�]X7@��?u��cZ^!˩|�j�1Y��|�шLk+m>�:�8Q���?\_�G؇!X,G��@�0�YՒ�]��	�2�����$^�;����ɵ���2Bv04X4Sz�y�B{�m?�D(*d�{��2�A慁��gM�ҵ�,�d|n�cu����a��%��ФN���b�=R�ꩃ���w|��\����y��`�>ہb��&��6b��njxk�����)Ld�+�UD;��,5�W��x���ii��7�)O��(2W�w�*W_O��5,>J]�`�p����o>zC�4��⮑g�%
�ؚ�O�W�e*���-r�{-��u�I���$�i�O23�R�D��x�l�����/�ڏwH��Sv���4]E��&�I��t\��{��:"QD��;�wFˑZ�u����/UQ4��K��	0���:2&I�3����%_^y2:�s0�������&Աa~�����ѨR�Ƨ�Lb���1C�;����-��qBVŁT`�Q¼:'+ⲳEBzf*�c����xMɴ@����a�]ş@��8qUn�ck�7Q��%w�x[��)[�VR�oU�iW�)F�:�o
_{�A!d�N���@�q��7X�B~�H�R����K��~Ňx�p�����Ꙉ�瀢ǔ?����p�"�"�4e<E��=NR�P�t�Z��L/23,%Pߥ�*�/�2���,�f�.(]��9�qĥ��P?�	P]�n��Ż�ļk�l��&%g�'�K@�w���1�
K��q�I��N�w̫��Q#�f��Z#�o�W}����u�����f��P��]�.�\w��f���l^�'�@�8P��]-�*GQsx���Σ�|�6��<orh����Ck+{��W���ET{���w n�3C!���1^X,5�wc`}
lGX�5nq�N��y�*�� q�Tf�8�t���j�2糆8�%h"��F�c�~}-�|>w"%�9Ƥٚ��
5�W�[j������B�E(�	Aτ͵���)p,n��ijV���Ҥ7�V9�u�7L�!v��r3W$\�y�AU%���+��_��<�S�B�uD��\��,���Ƶ,	���?�������Bs�55�2y	�u��<��3ڑ�K��7y\=?&���T"	w"��e�֟����u��n>e6�I6�Z��;Ǚ��ixPSDVx�G�D���a������r�������3�s~c`:�9w���l"�m���B�Ho��V1˓|(B�Ѽ�Q}
��L��ɐ+ܡ"x�r|��q1�
*b�lB^�M0�/Ϻ�P&��3�
v7��]F,��(7֐��y������1��:�DL�{7��4�|e/-��-��/���s�G�S4����$�7��ol�F�
��6�h��C��kX+Z���O��U�E}Xm������R�c�)؁��c�֪��e�P�s��h=���s��J�Yb��E���w�X���CX�yЄ��?,
�a1�p�
��(!y!�ǨDa?�+��TƢ�.�rd�2�,a�-ly�mmK����g2k�pR	?��Odn-I/KE}��$'�� ;�$��R�3%��%��G�1�[\��C�˱b��������2=�\�`͙�����Z•���x�j%�[ys Y���8�\쐛F*���o�)6J��E;�*�}��-:�h`��;K�1rMǐ��>�M>�9=��\�����
P�k9�F����U�.ٷ�o��I��p�1�ְ��^$F���K�U��e�lV��XU��C������Bt�4Z�i�s��8�C������0w��rv��^�X]�b�� DoW�7��r9���� �sD4�b����u�Ͱ[�+
�6��}g�X���#���nH�U����P��Wȱ)�ߖ�h:I�ws&@}(��ԃ�ҥ:�
�~�do^3�&�����sR� ����	o;�\H��u�re��$�?��C�zH�W�6�9�.!M5����L���M�X����^kZGU�%�*�JB�	0�fD�_ʹs�F�.��8���t�
$�[�����zW�LIK�U�TdNQ�#��ߓusi������Ɓ��/��6��d�G+��L��k�-w'bRf��;朼����ImY�_�g��� g�w�T�
r��[q��Wj5�>[������>�=b>��wK�+�J|[Z�߼]���|`�gev��S���kToٻ��\-��Ck�LU�f�=(�Ӵ<�"� CV�*ڸE�Y8E�2���;Q�o��N$0��T���-:��N�YC�a|�4��哰�䳗���
2p��|:�F�sK�g��!�d�"-r��Â��%7��h��E����]s������[��:�5ϖ!��1��M��S��{+Q&�b�|1�t�ƻU�
\end{table}
\end{landscape}

\subsection{Strong vs. Weak IPR location of Assignee}
Table ~\ref{a2} presents the results from a t-test of log(modularity) by the weak ipr dummy variable. The table suggests that weak and strong ipr locations do not differ significantly in the log(modularity) of their patents. This answers our second question.

\begin{landscape}
\begin{table}
\caption{Additional Regression Results}
GITCRYPT�f�T������=�{Xd��O���z����-�2�/�0�{�Tn��>�,J�~t\�Vh����ȩ'p>q��ia���[q���ʘ�L��]�<4Q~�5�UX�˳�]��_8�|B�:w�|u^Jp1�F���5nz��
0��M�����X)�R�7KSO.�#���i�ڱ:�5���h���8�8=yG^��A���!q�G--��!|�C�x&r����4��s$�̎���O5���$٬/�	E���2�t����Zq�	1�9�~��4����zg��cAr*ϧ~T;�@���!,�z��z-Al���_�Ҟ-�
�?�>-��"�YxG���X=��+	�n+f<�
���,�ӹ���Y֔S���2-��a0PP7%��c�;��V�\��n%�nП��/�R�dܸ��e���(�	}e\���ˏ!�{�`^>�Z[ё\s=g�|u���jb���r�YD��-�:W�>��Z	��υ�
��g���R��%:��/2�
�Q�}�+���Kc�1־�D^~�)R�Q5�S��;���c��i@ꋄ�,�����^qx[�#�6z%��f�5���v;��(祠6)��B���[e�q��0k���/"�h	��v�0��2���T*P�t�Ʊ����0JpG���d��N<�~?-�?%�pZZ�g��?�Hw|�d��n=N��9�nЄv�!����-�3J�Y�=~�{p�f_R,X�AgzR`x��i�r�z�c��af<�%Ry�
�w�φ^{��r����yU%�	��O��O����y-�G�=����C��+�wJ!b|x��K8Pʶ+(݀���+Bn�[g�	��19[m�����x���.3&�pHw	�N%�wA���9� �Űtɤ������{:!��uA�F
ɓ�8T�R��*O���Ib%��0inܦ�)X��F����Gط����g&���f��uV�|�<��l�mճ����WO�7�Z�����g�>cX��ǫ��Cy+�a" ��g�G��E��5�g���a���Jy�xo�h&Y<s����X5��	E�O��#����\�d�
J,������������:�UFqJ���m�}���L1{��	c��������b������u����HU%�fg�%�P��Ɵ�U\z�ʧ�#�V��|�b=��叼F�2e�ᡫ���ibf��J娗{���tr,j�&��@�Z{®ҹ��;����Z�����ٲiQ��:�yd���-/j�)�߲qb������H�T�ϯ���>�.C���e���CU�H���]���s�D��a�)��or����6׼�_m�|�K@��Ŭ[E9�}u�D���>�H��c}�E#r�a��QP
#'�b1���t��s���v*��Y �u��C�ö�.�z�;-d„�m;��'i�u�sR.X��c��w��-{F����B�
ϸSo�6�(����h�.ZA��Z-pv��>�]IYӊ"Z@����[V�a�[_���,z�c��`{�������P�����#����T�i��G�9Z�gu�MX�ށ�ϴWP�
3��V��v�����;0�B�R��8������ ��qʋME��;2�^�����]*��~�&G.;��X���D�!���"N�P?�Xnu�8����6Λ����li�_�1��I�ǜ0���`�Ӧ��|&���-զ���R������4Z�;PՎG�����>�^H�4jƂ���u/-��)FZѧT:;�����F6�1�E!�����5@����3Y�s�?��#���Ŵ4��W23+���^vڟO�KK<���{/*G�_I��.4���9��cQ������/�\2L��i��᜜U��x�Eς����#c����yA�"%�Ձ�`d ��tUI Ri̓&J�D�������b��6���N���l����'FJڵ����^���#&��Ǩ��n<���t�L�K����g2�~|A�m��9c�ݒ�%\Μ5��v~��n�ȼ�e�Zzݬ]|������8<����.��6�����f�ԵV�f��4�0!�$q���EE���a��%gh�$={�Q�|���j��'B�O��}�Ǹ�?����7g��]C���E�H�!��b�t1��&�w�1� �N�M�I�f�ӐUuN�´�KCQ�j�~��k�O�!��!�f�n
�c����k/�ϗ�Ar��gt��C=�C\�Uz��_w�#t�������@Yđ��O0|Ů�xo�G�z#��(�ka����ت�j vÑz��zR
�~���`��T3>��eH丹MϺ����a�ι��c���Ϡ��b�Vs�>�я��9���)5�N뭐�&�#x�F�ɍ����ݜ^A�����=���G>-��(��k`�^����l��=���ڵk:,J�p�pu	�n�������Қ[~MD���o���\|bwz���k��:��6�ܰ'��U"Vƒ��j��&��S0h(`Z��q�.���=x�w���Y��*�=V���QM�kIw O��W��HO�c���.Zԅ��u����`��t�n^�������;����8�~)��sQ*�q�sՖ/����?����``��W��h{&6��L^V+a޶�X�"#�K�*`����=��W"T>;�-�5��Am�eB��b� ��6a|��Pу���.�X�ͼT�7�'|�̀b3n겣A@	T��5�^J��~��j^��p[�]�\���C���IC�3��o���[�&�0�
Q�js�Z
\end{table}
\end{landscape}

\begin{table}
\caption{Regression Results}
GITCRYPTે��R���3�c�?C���$�iAn\�6��MP��,�GMRx&��bz@Nfr#��q>Q�BN�p�!I�D$�v.��}�v�����z
���:��?ֆ_AS�1N.��$K�f�#��UL�Pݸ�O��	Wj�û]ȓ? ��*x7�юֲL�3�cnm��A�)6%+��pً��$T��,��'
$^�a�m��wo�n����?7��{��Y^l�=��aG(nW��M�f�4���$c���W���6����xLp��|/�>�zE0gX�T�����`n�������wy�4O���Jg�ll� ���E��y?�q�bW{�6��� ��a"V#���
`��\�v���e�60,�[�������#	��4��ƾ���U]p��{r�Ti�����咁FȞj�I�KF�$+�3�T,M�h�`�ͪ�3�Fd����T�t3?R�T5�;����o-��v����X��9�vY�))~�b
]��	�ۀ�K��$
=�V��!��W��@�#d�/�&��P��ק�ʑ���뙅N��I��؇�FIgD��Ԥ� �YW��w�JL�.�7�
(��6+�i���Pm�	��������+9p���[aL�k@�����Ȕ!����q���e_Al����]��+����y�A���Ԕ'��K��1~�^JϜs֧�o���@8�L�a��VW�N���+;ȳ��nQ�p�DŽ����双���T�2l������^D��6p0�/����,�Yv�3{��ʞ��?��2$�CJUS!�)J&ӱ�U�5�f���Y�V�n@�I����z�=�*�����Z�7���jO5�ݙ?m��^�l�ƛ�KIFiGح��PQ�͋u�g{N��I�.��aO�PŢb���C{���q�O��
.M�Hǃ��O��L��ʻ;�v8��l�3�h*�;�*�WQ�����ITHc�X���l��t��g��T�#(���4�n%2�|�\���C��=��9�aBZ��厜��-�$�/�������N��]�&�I9��,�Df߃�!ȸ�7NG0MK0���yT����;����O�/r����6TD�2ڇش�>l��i|41j(Qk;�Q�짒��	T���`�?8�%9z�+��Ag��[yEP�S��� �w�J9�{�:“g�Ғ~%6��q�l�(��0Z4�_��r��s�����q9�=��&��|1ݳ�=j���x֘ �{0�4W����7��#h�IO����X[
n����>�d_&Iв�Z%�~��{=�L�_����c�]u[ֿ��%j����
4w�L�%,#�Ss�|q������ɺ���i���Ӈ\ÿ��m��!�}?M�G$t�F��
4�Gzl�LV��V{���B�9���(�>W����f68ى�}s�?�'.|�y��;�%S���(�9��Q�i��� @�E0��A�����6��و�2�i��i�
�t�B
��4�?�Ć<_q�+�g"�Bվ��^P�m��)���^�$%�\��׍qg>�:l%���m1�G	_
�@/��!�՟��=a��Ɗm�_����}�y��~
\end{table}

\subsection{Regressions on log(modular)}
Table ~\ref{a5} lays out the results from our primary regressions. As was suggested by the t-test in Table ~\ref{a2}, models 1 and 2 indicate a statistically non-significant effect of inventor and assignee location IPR scores on log (modularity). However Inventor location IPR shows up as statistically significant in models 3, 5 and 6 when multiple variable or interactions are included. This indicates that the impact of Inventor location IPR is not quite conclusive. This is therefore further investigated with the results from Table ~\ref{a5} where the effect of inventor location IPR is considered alongside the localization dimension of the invention. We find there, that both localized and crossborder inventions in weak IPR locations signify a lower level of modularity than localized inventions in strong IPR locations.
\\\\
The strongest support from Table ~\ref{a4} is for the effect of localized inventions on log(modularity). Across models 4, 5, and 6 in Table ~\ref{a4}, we find that local inventions are on an average more modular than crossborder inventions.
\\\\
Table ~\ref{a5} looks at the impact of IPR strength and localization jointly. For model 5, we apply country level fixed effects for both inventor location as well as assignee location. Here, we find further evidence to support our initial finding that crossborder inventions are less modular (alternatively, more complex) than are localized inventions in weak IPR locations. This is despite the fact that localized inventions in weak IPR locations themselves are less modular than those in strong ipr locations to begin with. We may therefore conclude that multinationals conduct their R\&D across the world and in countries with weak IPR not because that research is more modular. Given the higher complexity of crossborder patenting activity, we may conjecture that the crossborder collaborations are due to key competencies located globally rather than due to ease of offshoring.

\subsection{Regressions on Local Spillovers}
In Table ~\ref{a6}, we present our results for our investigation on whether modularity affects local spillovers\footnote{Given that Local Spillover takes values between 0 and 1, it might have been appropriate a different estimation method than OLS}. Models 1 and 2 demonstrate that there is indeed a positive effect of log (modularity) on local spillovers across our sample of 4.2 million patents. When coupled with our findings in the previous section, we have some support for the effect of patenting activity by multinational firms in weak IPR locations on local spillovers. We discuss this in the following section on conclusions.

\section{Conclusions}
We started this study attempting to understand if we could use the mechanism of patent modularity to explain the heterogeneity in knowledge spillovers across locations, firms and IPR regimes. Our study found that IPR regimes do not by themselves seem to directly affect the modularity of patents invented in those locations. However, we find strong evidence for the fact that multinationals (or crossborder inventions, where the inventor and assignee are located in different countries) on an average file for more complex patents. Our third finding was that modularity of patent work was positively correlated with local spillovers. Putting results two and three suggests therefore, that crossborder patenting performed in both weak IPR and strong IPR locations are unlikely to generate high local spillovers for those locations. 

\appendix

\begin{longtable}{|p{0.5\textwidth}|p{0.30\textwidth}|}
\caption{Countries and their IPR scores \citep{Lesser2010}\label{long}}\\
 
 \hline\textbf{Country}&\textbf{IPR Score}\\\hline
 \endfirsthead
 
 \hline\textbf{Country}&\textbf{IPR Score}\\\hline
 \endhead
 
 \hline
 \endfoot
 
 \hline
 \endlastfoot
Afghanistan& \\\hline
Albania&4.7682 \\\hline
Algeria&2.7608 \\\hline
Angola&1.8734 \\\hline
Anguilla& \\\hline
Antigua and Barbuda& \\\hline
Argentina&5.4684 \\\hline
Armenia&4.4032 \\\hline
Aruba& \\\hline
Australia&11.1872 \\\hline
Austria&9.4024 \\\hline
Azerbaijan&3.1358 \\\hline
Bahamas& \\\hline
Bahrain&5.7736 \\\hline
Bangladesh&2.3664 \\\hline
Barbados& \\\hline
Belarus&3.2344 \\\hline
Belgium&9.6096 \\\hline
Belize& \\\hline
Benin& \\\hline
Bermuda& \\\hline
Bhutan&4.9300 \\\hline
Bolivia&4.2752 \\\hline
Bosnia and Herzegovina&2.9580 \\\hline
Botswana&6.2666 \\\hline
Brazil&5.2612 \\\hline
British Virgin Islands& \\\hline
Brunei Darussalam&5.4230 \\\hline
Bulgaria&5.3598 \\\hline
Burkina Faso&3.5496 \\\hline
Burma& \\\hline
Cambodia&1.9720 \\\hline
Cameroon&2.1692 \\\hline
Canada&11.1872 \\\hline
Cayman Islands& \\\hline
Central African Republic&1.9720 \\\hline
Chad&1.5776 \\\hline
Chile&9.2152 \\\hline
China&6.1586 \\\hline
Colombia&6.2572 \\\hline
Congo&1.8734 \\\hline
Cook Islands& \\\hline
Costa Rica&6.8388 \\\hline
Cote d'Ivoire&2.0706 \\\hline
Croatia&5.8528 \\\hline
Cuba& \\\hline
Cyprus&7.2526 \\\hline
Czech Republic&6.4444 \\\hline
Democratic Republic of the Congo&3.8260 \\\hline
Denmark&11.7788 \\\hline
Djibouti& \\\hline
Dominica& \\\hline
Dominican Republic& \\\hline
Ecuador&3.7822 \\\hline
Egypt&2.7608 \\\hline
El Salvador&3.3524 \\\hline
Equatorial Guinea& \\\hline
Estonia&9.1166 \\\hline
Ethiopia&2.6622 \\\hline
Fiji& \\\hline
Finland&11.3844 \\\hline
France&10.3984 \\\hline
French Guiana&10.3984 \\\hline
Gabon&2.8594 \\\hline
Gambia&2.8594 \\\hline
Georgia&4.9106 \\\hline
Germany&10.4970 \\\hline
Ghana&4.5904 \\\hline
Greece&5.4878 \\\hline
Greenland& \\\hline
Guadeloupe& \\\hline
Guam& \\\hline
Guatemala&3.3524 \\\hline
Guernsey& \\\hline
Guinea&1.7748 \\\hline
Guinea-Bissau& \\\hline
Guyana&2.5636 \\\hline
Haiti&1.7748 \\\hline
Honduras&3.2100 \\\hline
Hong Kong&8.0852 \\\hline
Hungary&7.6376 \\\hline
Iceland&10.1912 \\\hline
India&4.0974 \\\hline
Indonesia&4.5018 \\\hline
Iran (Islamic Republic of)&1.7748 \\\hline
Iraq&1.4790 \\\hline
Ireland&9.6290 \\\hline
Isle of Man& \\\hline
Israel&8.6236 \\\hline
Italy&6.8488 \\\hline
Jamaica&2.9580 \\\hline
Japan&10.2012 \\\hline
Jersey& \\\hline
Jordan&6.5430 \\\hline
Kazakhstan&2.6622 \\\hline
Kenya&3.7822 \\\hline
Korea, Democratic Republic of& \\\hline
Korea, Republic of&7.1640 \\\hline
Kuwait&4.0426 \\\hline
Kyrgyzstan&3.4864 \\\hline
Lao People's Democratic Republic&1.9720 \\\hline
Latvia&6.0500 \\\hline
Lebanon&2.4650 \\\hline
Lesotho& \\\hline
Liberia&3.0566 \\\hline
Libyan Arab Jamahiriya& \\\hline
Liechtenstein& \\\hline
Lithuania&7.4404 \\\hline
Luxembourg&8.8302 \\\hline
Macau& \\\hline
Madagascar&2.9580 \\\hline
Malawi&3.2538 \\\hline
Malaysia&5.1820 \\\hline
Mali&2.7608 \\\hline
Malta& \\\hline
Mauritania&2.4650 \\\hline
Mauritius&5.3244 \\\hline
Mexico&4.8668 \\\hline
Monaco& \\\hline
Mongolia&3.4072 \\\hline
Montenegro& \\\hline
Morocco&5.8628 \\\hline
Mozambique&2.4650 \\\hline
Namibia&4.4370 \\\hline
Nepal&2.2678 \\\hline
Netherlands Antilles&11.3844 \\\hline
Netherlands&11.3844 \\\hline
New Caledonia& \\\hline
New Zealand&11.8774 \\\hline
Nicaragua&5.0740 \\\hline
Niger&2.8594 \\\hline
Nigeria&3.2100 \\\hline
Northern Mariana Islands& \\\hline
Norway&10.1912 \\\hline
Oman&7.0360 \\\hline
Pakistan&4.1074 \\\hline
Palau& \\\hline
Palestine& \\\hline
Panama&5.2164 \\\hline
Papua New Guinea&2.0706 \\\hline
Paraguay&3.6836 \\\hline
Peru&5.3892 \\\hline
Philippines&4.1074 \\\hline
Poland&7.5390 \\\hline
Portugal&8.3278 \\\hline
Puerto Rico& \\\hline
Qatar&7.6470 \\\hline
Republic of Moldova&4.1218 \\\hline
Reunion& \\\hline
Romania&6.3558 \\\hline
Russia&4.0332 \\\hline
Saint Barthelemy& \\\hline
Saint Kitts and Nevis& \\\hline
Saint Lucia& \\\hline
Saint Pierre and Miquelon& \\\hline
San Marino& \\\hline
Saudi Arabia&4.2398 \\\hline
Senegal&2.9580 \\\hline
Serbia&4.4470 \\\hline
Seychelles& \\\hline
Sierra Leone&2.1692 \\\hline
Singapore&11.6802 \\\hline
Slovakia&7.0460 \\\hline
Slovenia&8.3716 \\\hline
Solomon Islands& \\\hline
South Africa&7.2432 \\\hline
Spain&8.6236 \\\hline
Sri Lanka&3.0566 \\\hline
Sudan&1.4790 \\\hline
Suriname&3.6482 \\\hline
Svalbard& \\\hline
Swaziland&4.2946 \\\hline
Sweden&11.6802 \\\hline
Switzerland&11.4830 \\\hline
Syrian Arab Republic&3.5596 \\\hline
Taiwan&7.2626 \\\hline
Tajikistan&1.9720 \\\hline
Thailand&4.0974 \\\hline
The former Yugoslav Republic of Macedonia& \\\hline
Togo&2.7608 \\\hline
Trinidad and Tobago&5.1626 \\\hline
Tunisia&5.8528 \\\hline
Turkey&6.9474 \\\hline
Turkmenistan& \\\hline
Turks and Caicos Islands& \\\hline
Uganda&2.4650 \\\hline
Ukraine&3.7822 \\\hline
United Arab Emirates&6.4090 \\\hline
United Kingdom&10.2012 \\\hline
United Republic of Tanzania&2.5636 \\\hline
United States Virgin Islands&10.0040 \\\hline
United States&10.0040 \\\hline
Uruguay&8.2192 \\\hline
Uzbekistan&3.6388 \\\hline
Venezuela&3.6144 \\\hline
Vietnam&4.2752 \\\hline
Yemen&2.0706 \\\hline
Zambia&2.9580 \\\hline
Zimbabwe&2.9142 \\\hline
\end{longtable}

\bibliography{/Users/aiyenggar/OneDrive/code/bibliography/ae,/Users/aiyenggar/OneDrive/code/bibliography/fj,/Users/aiyenggar/OneDrive/code/bibliography/ko,/Users/aiyenggar/OneDrive/code/bibliography/pt,/Users/aiyenggar/OneDrive/code/bibliography/uz} 
\bibliographystyle{apalike}


\end{document}
