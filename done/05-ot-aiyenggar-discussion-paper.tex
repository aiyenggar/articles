%05-ot-aiyenggar-discussion-paper.tex
\documentclass[12pt]{article}
\usepackage{amssymb,latexsym}
\usepackage[round,sort]{natbib}
\usepackage{multirow,array}
\usepackage{fancyhdr}
\usepackage{lastpage}
\usepackage{graphicx}
\usepackage[bottom]{footmisc}
\graphicspath{ {05-ot-aiyenggar-discussion-paper-images/} }
\usepackage[T1]{fontenc}
\usepackage{mathptmx}
\usepackage{tabu}
\usepackage{textcomp}
\usepackage{tikz} 
\usepackage{adjustbox}
\usepackage{longtable}
\usetikzlibrary{arrows,decorations.pathmorphing,backgrounds,fit,positioning,shapes.symbols,chains}
\usepackage[a4paper]{geometry}
\usepackage{caption}
\setlength\parindent{0pt}	
\setlength{\parskip}{1em}
\usepackage[a4paper]{geometry}
\usepackage{pdflscape}


\captionsetup[longtable]{skip=3pt}

\geometry{
 total={170mm,237mm},
 left=20mm,
 top=30mm,
}

\newenvironment{hypothesis}{
  	\itshape
  	\leftskip=\parindent \rightskip=\parindent
  	\noindent\ignorespaces}
	
\pagestyle{fancy}
\fancyhf{}
\lhead{Readings on Institutional Theory}
\rfoot{Page \thepage  \ of \pageref{LastPage}}
\rhead{Iyenggar}


\begin{document}
\title{Readings on Institutional Theory:\\  Session 05, Organization Theory 2016}
\author{Ashwin Iyenggar  (1521001) \\ ashwin.iyenggar15@iimb.ernet.in} 


\maketitle
\thispagestyle{empty}


\begin{quotation}  
The empirical reality is that organizations often behave in ways that defy economic logic or norms of rational behavior. And institutional theory offers a paradigm devoted to understanding that.\\
\null\hfill \textit{\cite{Suddaby2010}}
\end{quotation}

\abstract
In this discussion article, we explore the antecedents of institutional theory and attempt to place it in the context of other organizational theories discussed so far. We then map the assigned articles and identify the key constructs and themes in institutional theory by highlighting connections and contradictions. We conclude that institutional theory is rich in potential opportunities for research with newer opportunities coming from the interaction of agency and structure, and from the potential for pluralism in logics and relationships.

\section{Origins of Institutional Theory}
Institutional theory may be traced back to  \cite{Selznick1957} where he posits that practices and routines become institutionalized when they are "infused with value beyond the technical requirements at hand". Selznick had discovered that structures and practices survived even when they no longer achieved the goals for which they had been designed. This had lead to the startling insight that instead of changing their structures, organizations adopt new goals suited to existing structures.

1977 witnessed two of the three seminal articles that institutional theory was later built upon. First,  \cite{Meyer1977} observed that "within any given sector or industry, organizations use similar organizational forms". The influence of the embedded social context was used to explain this. \cite{Meyer1977} suggested that "organizations increase their legitimacy and survival prospects by incorporating practices and procedures that are defined by prevailing rationalized conceptions of organizational work that is institutionalized in society, independent of the immediate efficacy of of the acquired practices and procedures." \cite{Meyer1977} essentially pointed out to theorists that organizations are not simply production systems but social and cultural systems embedded within a institutional context that is comprised of the state, professions, interest groups and public opinion.  Second, \cite{Zucker1977} described the micro processes by which authority becomes institutionalized in organizations. The article focussed on how actors use cues from their organizational environment to attribute meaning to events. We note that both the 1977 works were highly ideational rather than structural in their treatment of organizational behavior. In the third article, \cite{Dimaggio1983} extend the  ideational elements  presented by \cite{Meyer1977, Zucker1977} by suggesting that organizations that adopt a similar structural position in an organizational field eventually become isomorphic with their common institutional environment \citep{Suddaby2010}.

\section{The Tenets of Institutional Theory}\label{S:Review}
\cite{Lincoln1995} in \cite{Suddaby2010} suggests that an important principle in institutional theory is "the tendency for social structures and processes to acquire meaning and stability in their own right rather than as instrumental tools for the achievement of specialized ends." Institutional theorists have therefore proposed that formal organizational structure reflects not only technical demands and resource dependencies, but is also shaped by institutional forces, including rational myths, knowledge legitimated through the educational system and by the professions, public opinion, and the law \citep{Powell2007}. The growth of institutional theory itself has seen each of these factors being neglected at some stages, and elaborated at others as is described in the following sections.

\subsection{"Old" Institutional Theory}
Early accounts identified institutional effects as being concerned principally with social stability. The following quote from Weber illustrates the frame that sociologists construed the reality of organizations.
\begin{quotation} 
The "iron cage" traps individuals in systems based on rational calculation, teleological efficiency and bureaucratic control.\\ 
\null\hfill \textit{\cite{Weber2002}}
\end{quotation}

In one of the most cited and celebrated articles during the early phase of institutional theory, \cite{Dimaggio1983} pointed to coercive, normative, and mimetic processes of reproduction that were causing organizations to resemble each other (the term used was isomorphism). Coercive factors involved political pressures and the force of the state in providing regulatory oversight and control; normative factors stemmed from the  influence of the professions, professional networks, universities and the role of education; and mimetic forces drew on habitual, taken-for-granted responses to circumstances of uncertainty. \cite{Powell2007} suggest in retrospect that they omitted evangelizing efforts, where institutional entrepreneurs champion the adoption or influence of specific practices, one that was corrected in the 1988 article \citep{Dimaggio1988}. Later, \cite{Scott1995} suggested three pillars of the institutional order: regulative, normative, and cultural/cognitive. Regulative elements emphasize rule setting and sanctioning, normative elements contain an evaluative and obligatory dimension, while cultural/cognitive factors involve shared conceptions and frames through which meaning is understood. Each of these pillars offered a different rationale for legitimacy, either by virtue of being legally sanctioned, morally authorized, or culturally supported. \cite{Powell2007} suggests that these two key treatments of institutional mechanisms underscored that it is critical to distinguish whether an organization complies out of expedience, from a moral obligation, or because its members cannot conceive of alternative ways of acting \citep{Powell2007}.

\subsection{"New" Institutional Theory}
The "iron cage" metaphor used in \cite{Dimaggio1983} had been misinterpreted to mean that all organizations will eventually become isomorphic. The introduction of the notion of agency in the form of institutional entrepreneurship in \cite{Dimaggio1988} resulted in a new focus on the power of the actors within the organizational field. This heightened recognition that institutionalization is a political process. Attention was also placed on  internal influences and the heterogeneity of responses. This increased concern with the role of agency in institutionalization amongst scholars. New institutionalism has laid focus at the field level, based on the insight that organizations operate amidst both competitive and cooperative exchanges with other organizations. The formal structure should therefore be ideally seen as an adaptive product, responsive to environmental influences, including cultural definitions of propriety and legitimacy \citep{Selznick1996}.  \cite{Philips2009} suggest that institutional entrepreneurship is an important idea because it considers how actors can attain their goals by intentionally constructing and/or altering the institutional structures in which they are embedded. However \cite{Suddaby2010} cites that research in institutional theory is plagued with the problem of organizations now being presented as "hypermuscular supermen" instead of the "passive cultural dopes" earlier. In our readings this week, most of which come after this admonition from Suddaby, we note that there is a greater integration of the old and new as is depicted in Table ~\ref{assigned}

In the following section, we step back and attempt to place institutional theory in the context of other organizational theories that we have encountered thus far (both in the course and in other courses).
\begin{landscape}
\begin{center}
\section{Placing Institutional Theory}
 \begin{longtable}{|p{0.18\textwidth}|p{0.30\textwidth}|p{0.20\textwidth}|p{0.20\textwidth}|p{0.15\textwidth}|p{0.20\textwidth}|}
 \caption{Comparison of Theories Explored\label{long}}\\
 
 \hline\textbf{Theory}&\textbf{Main Idea}&\textbf{Level of Analysis}&\textbf{Organizations}&\textbf{Environment}&\textbf{Change}\\\hline
 \endfirsthead
 
\hline\textbf{Theory}&\textbf{Main Idea}&\textbf{Level of Analysis}&\textbf{Organizations}&\textbf{Environment}&\textbf{Change}\\\hline
 \endhead
 
 \hline
 \endfoot
 
 \hline
 \endlastfoot

Structural Contingency Theory&Contingencies determine organizational form&Organization&React to Context&Determined Organization&Implicit\\\hline
Strategic Choice \citep{Child1972}&Organizations could choose not to adapt&Organization&Shaped the context&&\\\hline
Configuration Theory&Strategies, structures and processes should be considered holistically&Organization&&&\\\hline
Behavioral Theory&Organizations are conceptualized as information processing systems that use routines to cope with ambiguous streams of information&Organization&Routines shape responses&Influences Organization&Emphasized organizational adaptation\\\hline
Resource Dependence Theory&Organizations seek to influence and dominate the environment, not simply adapt to it&Organization&Shaped the context&Influenced by Organizations&Cognitive frames and distribution of power determine detterance of change\\\hline
Institutional Theory&Organizations increase legitimacy and survival prospects by incorporating practices and procedures institutionalized in society independent of immediate efficacy&Organizational Field&Influenced by environment but also shape it&Influences organization and is influenced by it&Field impedes organizational change\\\hline
Ecological Theory&Organizational survival is a product of fit between form and market forces&Populations of Organizations&Structural inertia impedes adaptation&Selected Organizations by fit&Extremely difficult to achieve\\\hline
Network Theories&Networks may be structures of opportunity, constraints or or embedded relationships&Network of Organizations&Complexly influenced by the environment&&Change by bridging holes\\\hline
 \end{longtable}
 \end{center}
 \end{landscape}


\section{Constructs in Institutional Theory}

\subsection{Organizational Fields}
Fields are a community of organizations that partakes of a common meaning system and whose participants interact more frequently and fatefully with one another than with actors outside the field \citep{Scott1995}. \cite{Dimaggio1983} define organizational fields as those organizations that in the aggregate constitute a recognized area of institutional life: key suppliers, resource and product consumers, regulatory agencies, and other organizations that produce similar services or products. \cite{Dimaggio1983} suggest that the virtue of such a definition of the organizational field is that it directs attention to the totality of relevant actors. This is a salient aspect of institutional theory as compared with the other theories depicted in Table ~\ref{long}

\subsection{Institutional Logics}
"Institutional logics are socially constructed rules, norms and beliefs constituting field membership, role identities and patterns of appropriate conduct. Logics, conveyed through regulatory, normative and cognitive processes, shape how actors interpret reality and define the scope of socially legitimate conduct"\citep{Friedland1991}.  \cite{Friedland1991} further suggests that institutional logics may consist of both symbolic and material carriers, where symbolic carriers are the rules, norms and belief systems embedded in an institutional logic, and material carriers are the routines, relationship systems, and artifacts that materialize and reproduce them. 

As an essential construct in institutional theory, logics is modeled in numerous studies. Among the articles reviewed for this discussion paper, \cite{Helms2012} hypothesize that the number of distinct logics faced by the organization is the critical dimension of logics that influences the likelihood of organizational settlement on a new institutional arrangement. While \cite{Dunn2010} suggest that multiple logics are often in contestation, while \cite{Quattrone2015} uses a rich historical analysis of the accounting practices of the Jesuit Order to illustrate the role of unfolding rationality in the emergence of procedural logics.


\subsection{Structuration}
The notion of structuration is attributed to \cite{Giddens1979}, and is defined by \cite{Dimaggio1983} as consisting of  four parts. First, an increase in the extent of interaction among organizations in the field. Second, the emergence of sharply defined inter-organizational structures of domination and patterns of coalition. Third, an increase in the information load with which organizations in a field must contend. Finally, the development of a mutual awareness among participants in a set of organizations that they are involved in a common enterprise. The main premise in \cite{Dimaggio1983} is that organizational isomorphism is caused due to the structuration of organizational fields.


\subsection{Institutional Isomorphism}
Max Weber used the term "iron cage of rationality" to describe what he viewed as a trend in society to move towards a form of bureaucratic rationality that would not realize universal freedom, but rather create an "iron cage" from which there would be no escape. The cause of this trend, Weber believed, stemmed from the expectations and hopes of the Enlightenment thinkers who felt that it was necessary to maintain a strong linkage between the growth of rationality, science and human freedom. Weber however saw this as an ironic, bitter illusion. According to \cite{Dimaggio1983}, societies and governments intend greater diversity but the empirical reality is that organizations seem to be growing increasingly isomorphic. As discussed earlier,  \cite{Dimaggio1983} identify three mechanisms coercion, mimetic (imitation), and normative processes to explain the isomorphization of organizations.

Weber suggested that bureaucratization resulted from three causes: competition among capitalist firms in the marketplace, competition among states, and bourgeois demands for equal protection under the law. Of the three, the most important for Weber was the competitive marketplace. \cite{Dimaggio1983} argue that the bureaucratization of the corporation and the state have been achieved; and that structural change in organizations is less and less driven by competition or by the need for efficiency but out of the structuration of organizational fields. This is effected largely by the state and professions, where individual efforts to deal rationally with uncertainty and constraint often lead in the aggregate to homogeneity in structure, culture and output.


\subsection{Legitimacy}
Legitimacy is seen as an organizational "imperative" that is both a source of inertia and a summons to justify particular forms and practice \citep{Selznick1996}. Identifying legitimacy as critical for organizational survival, \citep{Kostova2008} suggest that legitimacy is achieved primarily through isomorphism, where organizations become similar to other organizations in their organizational field.  \cite{Harmon2015} suggest that despite the critical role played by legitimacy in institutional arrangements, little is understood about the processes of legitimation. They go on to identify that the specific ways in which communication strategies shape and reflect social actors\textquotesingle \ assumptions of legitimacy remains underspecified. In their article, \cite{Harmon2015} determine that the differences in the underlying structure of communication strategies may be used to understand the processes of legitimation.

\subsection{Decoupling}
Organizations engage in ceremonial adoption of institutionalized structures and practices while at the same time decoupling themselves from the environment by actually using different structures and practices they view as more economically efficient \citep{Kostova2008}. In the context of internationalization, \cite{Kostova2008} suggest that a given MNC sub-unit has to be approved and accepted by many actors, externally and internally, each of who might perceive the unit to be part of different organizational fields and expect it to adopt different institutionalized standards. \cite{Kostova2008}  argue that MNCs cannot function without practicing decoupling and ceremonial adoption of certain legitimating standards.


\subsection{Myth, Rhetoric and Ceremony}
While early rhetorical theory had emphasized the speaker\textquotesingle s persuasion as the inspiration for social action, recent rhetorical theory has emphasized the role of audience in affecting the way rhetoric shapes social action. \cite{Harmon2015} argue that intrafield level rhetoric tend to restrict and suppress challenges to legitimacy and therefore tend to promote institutional reproduction and maintenance. On the other hand, interfield level rhetoric tends to amplify the challenges to legitimacy and are therefore likely to promote institutional change.


\subsection{Institutional Work}
Institutional approaches to organization theory have traditionally focused attention on the relationships among organizations and the  fields in which they operate. This has helped provide strong accounts of the processes through which institutions govern action. The study of institutional work reorients these traditional concerns, shifting the focus to understanding how action affects institutions \citep{Lawrence2009}. \cite{Lawrence2009} in reference to \cite{Suddaby2010} suggests that a significant part of the promise of institutional work as a research area is to establish a broader vision of agency in relationship to institutions, one that avoids depicting actors either as "cultural dopes" trapped by institutional arrangements, or as hypermuscular institutional entrepreneurs. 

\subsection{The Position of the Individual}
\cite{Suddaby2010} notes that the individual has been missing in much institutional theory work and that institutional logics must have a perceptual component that operates cognitively at the level of individuals. He distinguishes between "the old institutionalism," in which "issues of influence, coalitions, and competing values were central, along with power and informal structures", and "the new institutionalism," which emphasizes "legitimacy, the embeddedness of organizational fields, and the centrality of classification, routines, scripts, and schema" (\cite{Philips2009} citing \cite{Greenwood1996}). 

Among our readings this week, \cite{Zietsma2010} highlights the necessity for actor "embedded agency". While much prior work has identified external sources that predict deviation from institutional pressures for isomorphism, \cite{Lepoutre2012} demonstrate that a combined analysis of both symbolic and material aspects of institutional change is necessary in understanding the mechanism of deviating logic.

\subsection{Institutional Complexity}
Scholars have suggested that social actors are sometimes confronted with incompatible prescriptions from "multiple institutional logics", leading to an environment of institutional complexity. Scholars have suggested that individuals may experience complexity in different ways and that this may lead to respond differently. \cite{Harmon2015} suggest that analyzing the structural use of rhetoric may provide insight into how individuals may experience varying levels of institutional complexity.

\newpage
\section{Assigned Readings}
The assigned readings for the session on institutional theory includes a wide variety of perspectives and contexts. In Table ~\ref{assigned}, we attempt to map the assigned readings along two dimensions: First, we consider the stage of the organizational phenomena studied (this is captured horizontally). Second, we identify the institutional constructs deployed by scholars in understanding organizational phenomena. This, we capture vertically. Table ~\ref{assigned} captures the potential for institutional theory to capture multiple and often contradictory constructs.

\begin{center}
 \begin{longtable}{|p{0.13\textwidth}|p{0.18\textwidth}|p{0.18\textwidth}|p{0.18\textwidth}|p{0.18\textwidth}|}
 \caption{Mapping out Assigned Readings\label{assigned}}\\
 
 \hline\textbf{Construct}&\textbf{Creation}&\textbf{Maintenance}&\textbf{Change}&\textbf{Non-Conformity}\\\hline
 \endfirsthead
 
 \hline\textbf{Construct}&\textbf{Creation}&\textbf{Maintenance}&\textbf{Change}&\textbf{Non-Conformity}\\\hline
 \endhead
 
 \hline
 \endfoot
 
 \hline
 \endlastfoot


Rhetoric&&\cite{Harmon2015}&\cite{Harmon2015}&\\\hline
Immunity&&&&\cite{Lepoutre2012}\\\hline
Reflexive Normality&&\cite{Lok2013}&&\\\hline
Negotiation&\cite{Helms2012}&&\cite{Lok2013}&\\\hline
Ritual&&\cite{Dacin2010}&&\\\hline
Relational Pluralism&&&\cite{Raffaelli2014}&\\\hline
Structuration&&\cite{Dimaggio1983}&&\\\hline
Boundary and Practice Work&&\cite{Zietsma2010}&\cite{Zietsma2010}&\cite{Zietsma2010}\\\hline
Plural Logics&&&\cite{Dunn2010}&\\\hline
Unfolding Rationality&&\cite{Quattrone2015}&\cite{Quattrone2015}&\\\hline
 \end{longtable}
 \end{center}


\section{Salient Perspectives in Institutional Theory Research}

\subsection{Efficiency vs. Legitimacy}
Organizations first adopt innovations for efficiency, but later do so for legitimacy. As an innovation spreads, a threshold is reached beyond which adoption provides legitimacy rather than improves performance \citep{Meyer1977}. Institutional theory is based on the observation that organizations often seem to behave in ways different from that that would improve efficiency. As we note in our readings this week, once the organizational fields develop, legitimacy concerns are seen as being more important.

\subsection{Positivist vs. Interpretivist Apporach}
A review of the empirical literature on institutional theory indicates that the structural elements such as isomorphism and decoupling have dominated the research agenda\citep{Suddaby2010}. \cite{Suddaby2010} suggests that rationalized myths, legitimacy and taken-for-grantedness are re-entering the literature. We note from our readings that ideational components are critical in a fully specified institutional theory, and that the stream will be better served with the operationalizing of some of the intangible and ideational constructs identified by \cite{Suddaby2010}

\subsection{Outcomes vs. Process}
Along the lines of the previous observation, institutional theorists have tended to study the outcomes or products of institutional influences on organizations. While the focus has traditionally, been outside the organization \citep{Suddaby2010}, understanding the internal processes of change and maintenance in the context of institutional work may likely lead to a more complete understanding of organizational phenomena.

\subsection{On the Paradox of Embedded Agency}
The paradox of embedded agency refers  to the tension between institutional determinism and agency. Specifically, how can organizations or individuals innovate if their beliefs and actions are determined by the institutional environment they wish to change? \citep{Scott1987}. 

\cite{Harmon2015} conclude that rhetoric functions as the theoretically identifiable and empirically observable factor that restricts what actors can say or object to, therefore observing that institutions operate as a nested system where it constrains actors at one level while enabling them at another. \cite{Helms2012} find that embedding oneself in decision making among logically diverse participants implied a lower likelihood of settlement to new institutional arrangements. They explain this rather contradictory result by suggesting that exposure to new logics or perspectives may call into question the actors\textquotesingle existing worldviews, and therefore reduce conformity. The implications for the paradox of embedded agency from the \cite{Helms2012} study is that embeddedness is a very complex social phenomenon that  may have consequences for limited influence, disillusionment or exposure to alternative views.

\subsection{Symbolic Environment vs. Material Environment}
This debate is triggered by the question of why organizations engage in activities that are legitimate in the symbolic realm rather than in the material realm. Why do organizations adopt behaviors that conform to normative demands but conflict with the rational attainment of economic goals? \cite{Suddaby2010} suggests that organizational phenomena are characterized by both an ideational element and a structural element. However since \cite{Dimaggio1983},  the structural element held primacy including due to issues of measurability. In our readings this week, \cite{Lepoutre2012} introduce the notions of symbolic and material immunity as attributes that predict firm-level non-conformity. Here immunity is modeled in the material and symbolic contexts, and is used to theorize how non-conformists deal with legitimacy issues.

\section{Opportunities building on Institutional Theory}
\subsection{Phenomena}
The area of international business provides us interesting opportunities to apply and extend institutional theory. The exchanges between \cite{Kostova2008} and \cite{Philips2009}, indicate that international business research has tended to emphasize the "old" institutional theory and that there are opportunities to bring in recent advances in institutional theory research to better understand organizational phenomena in the international context. 

\subsection{Methodology}
As noted in the sections on the positivist approach and outcome studies, institutional research has historically maintained a quantitative focus. This has meant that hard to measure constructs in the ideational domain had been left out. \cite{Suddaby2010} suggests that institutional theory has largely failed to retain methodologies that are consistent with their need to attend to meanings, systems, symbols, mythos and the processes by which organizations interpret their institutional environments. In order to correct for this imbalance, \cite{Suddaby2010} suggests that  there is a need to move from strictly positivist research to include interpretivist methods. \cite{Suddaby2010} calls for in-depth case studies as a potential option to correct for the imbalance. \cite{Quattrone2015} suggests that an alternative to pursuing a positivist description of beliefs and assumptions is to inquire about what is not represented. He suggests that research should be methodologically inspired by a search for what "cannot be categorically framed", what he calls the procedural approach, and use that to rearticulate the sources of legitimacy, authority and control.

\section{Conclusion}
While institutional theory has been among the most important and prolific theories researched in understanding organizational phenomena, the field continues to offer exciting opportunities for further work. \cite{Powell2007} suggest that three themes capture the topics of contemporary interest in institutional theory. First, institutional theorists could develop studies  accounting for both institutional heterogeneity as well as institutional homogeneity. Second, there is much work to be done in the direct measurement of institutional effects, and \cite{Quattrone2015} is an interesting inspiration for more such. Finally, the competing, multi-level and nested processes within organizational fields and across countries offers significant opportunity for further work.


\bibliography{/Users/aiyenggar/OneDrive/code/bibliography/ae,/Users/aiyenggar/OneDrive/code/bibliography/fj,/Users/aiyenggar/OneDrive/code/bibliography/ko,/Users/aiyenggar/OneDrive/code/bibliography/pt,/Users/aiyenggar/OneDrive/code/bibliography/uz} 
\bibliographystyle{apalike}

\end{document}
