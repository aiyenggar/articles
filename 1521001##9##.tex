%1521001##9##.tex
\documentclass[12pt,letterpaper]{article}
\usepackage{mathptmx}
\usepackage[margin=1in]{geometry}

\usepackage{setspace}
\singlespacing
  
\usepackage{amssymb,latexsym}
\usepackage[round,sort]{natbib}
\usepackage{fancyhdr}
\usepackage{lastpage}
\usepackage{graphicx}
\graphicspath{ {qe9/} }

% Bold Table and Figure captions
\usepackage{caption}
\captionsetup{figurename=FIGURE}
\captionsetup{tablename=TABLE}
\captionsetup[figure]{labelfont=bf}
\captionsetup[table]{labelfont=bf}
  
% Turns off all section numbering
\setcounter{secnumdepth}{0} 

  % Places all tables at end of document and creates AOM-style table-here placeholders
  \usepackage[nolists]{endfloat} % Places all figures and charts at end of manuscript and adds 'insert table x about here' lines.
  \renewcommand{\figureplace}{
    \begin{center}
    \begin{singlespace}
    ------------------------------------\\
    Insert \figurename \ \thepostfig\ about here.\\
    ------------------------------------
    \end{singlespace}
    \end{center}}
  \renewcommand{\tableplace}{%
    \begin{center}
    \begin{singlespace}
    ------------------------------------\\
    Insert \tablename \ \theposttbl\ about here.\\
    ------------------------------------
    \end{singlespace}
    \end{center}}

  \usepackage{titlesec}
   \titleformat{\title}
       {\filcenter\normalfont\bfseries\uppercase}{\thetitle}{1em}{}
  \titleformat{\section}
    {\filcenter\normalfont\bfseries\uppercase}{\thesection}{1em}{}
  \titleformat{\subsection}
    {\normalfont\bfseries}{\thesubsection}{1em}{}
  \titleformat{\subsubsection}[runin]
   {\normalfont\bfseries\slshape}{\thesubsubsection}{1em}{\hspace*{\parindent}}
       
\usepackage{tabu}
\usepackage{textcomp}
\usepackage{listings}
\usepackage{hyperref}
\usepackage{verbatim}
\usepackage{tabu}
\hypersetup{
    colorlinks=true,
    linkcolor=blue,
    filecolor=cyan,      
    urlcolor=cyan,
    citecolor=blue,
}

\usepackage{etoolbox}

\makeatletter

% Patch case where name and year are separated by aysep
\patchcmd{\NAT@citex}
  {\@citea\NAT@hyper@{%
     \NAT@nmfmt{\NAT@nm}%
     \hyper@natlinkbreak{\NAT@aysep\NAT@spacechar}{\@citeb\@extra@b@citeb}%
     \NAT@date}}
  {\@citea\NAT@nmfmt{\NAT@nm}%
   \NAT@aysep\NAT@spacechar\NAT@hyper@{\NAT@date}}{}{}

% Patch case where name and year are separated by opening bracket
\patchcmd{\NAT@citex}
  {\@citea\NAT@hyper@{%
     \NAT@nmfmt{\NAT@nm}%
     \hyper@natlinkbreak{\NAT@spacechar\NAT@@open\if*#1*\else#1\NAT@spacechar\fi}%
       {\@citeb\@extra@b@citeb}%
     \NAT@date}}
  {\@citea\NAT@nmfmt{\NAT@nm}%
   \NAT@spacechar\NAT@@open\if*#1*\else#1\NAT@spacechar\fi\NAT@hyper@{\NAT@date}}
  {}{}

\lstset{
basicstyle=\ttfamily,
columns=flexible,
breaklines=true
}
\newenvironment{hypothesis}{
  	\itshape
  	\leftskip=\parindent \rightskip=\parindent
  	\noindent\ignorespaces}

\fancypagestyle{plain}{
  \renewcommand{\headrulewidth}{0pt}
  \fancyhf{}
}	


\begin{document}
\title{Methodological Critique of \cite{Funk2014}\vspace{-2cm}}
\date{}
\maketitle

\pagestyle{fancy}
\fancyhf{}
\lhead{Methodological Critique of \cite{Funk2014}}
\rhead{\thepage}

Funk, R. J. 2014. Making the most of where you are: Geography, networks, and innovation in organizations. \textit{Academy of Management Journal}, 57(1): 193-222.\vspace{1cm}

Russell Funk argues that while geographical proximity to industry peers improves the innovation output of firms, this relationship is moderated by the efficiency of intra-firm information networks. Funk suggests that the underlying mechanism driving that relationship is that diversity facilitates information processing. I found the study interesting for two reasons. First, the geographical effects of innovation is an open area for research and there is still a lot we do not quite understand about the role of geographical proximity in fostering innovation. Second, Funk frames the question as an interaction of two factors geographical proximity and intra-firm network, both of which are neither completely exogenous nor completely endogenous to firm performance. I have structured my comments in the four following sections. For each I provide a summary of my critique and follow it up with numbered comments. The comments are ordered in decreasing order of importance (i.e., the most important comments come first).


\section{Theory}
Start with a summary, and then list issues numbered in order of importance.
\begin{enumerate}
  \item Firms may self-select into locations. Are location choices exogenous?
  \item Could other firms move exogenously to the focal firm?
  \item Other firms may be guided strategically by their managers. Is there an assumption that all industry peers are required to take a similar strategic direction?
  \item Do intra-organizational information networks evolve over time? Has an assumption been made to the contrary?
  \item What is the evidence for the presence of knowledge spillovers. \cite{Arora2017a} suggest that patent citations may not be indicative of knowledge flows. It seems that Funk assumes that local spillovers take place, building on earlier work. Recent work is challenging this assumption.
  \item Physical proximity does not imply more frequent interactions. AoM meetings maybe the only time neighbors meet.
  \item Consider the complement vs compete tension
    \item Funk argues that proximity allows firms to capture large volumes of knowledge through spillovers from nearby organizations. This would be incomplete without also considering absorptive capacity of recipient firms
\end{enumerate}

\section{Empirical Methodology}
We started out attempting to improve our understanding of the mechanisms behind the embedded agent - institutional field engagement. 
\begin{enumerate}
  \item Sample selection
    \item If links considered are of co-inventor links from successful patents, could there be a case of selecting on the dependent variable. What about all the knowledge flows that happened but did not end in co-inventor links on a successful patent application.
      \item firms that do not list at least one inventor with an address in the same region as a firm\textquotesingle s main research facility are excluded. Is this sufficient?
  \item Potential bias from eliminating all satellite R\&D facilities. I would like to see if the size of the satellite facilities dropped vis-a-vis those retained.
  \item Authors shifting location. Not clear how this is addressed.
  \item Potential simultaneity or endogeneity among control variables
  \item The question of globally sourced R\&D. I would like to see statistics about how many of the inventors on the patents on his panel were resident in the US
  \item Combinations calculated based on USPTO class 100000 of them means that the chance of diversity is higher. Much of that may be noise. May endup overstating the effect of inefficiency
  \item There seems to be a confounding of the location of the facility and that of the inventor. It is not clear which is affecting the results. Would be better to just look at inventor locations.
  \item Euclidean distance may be problematic since Houston is much more spread out than is Boston. 5 miles may mean different things in different places.
  \item The California control - questionable, especially is it appropriate in the context of nano technology industry?
\end{enumerate}

\section{Results}
Summarize the results
\begin{enumerate}
  \item Can the relationship hypothesized in Hypothesis 2a changed in direction and still hold? Is there a simultaneity here?
\end{enumerate}

\section{Other Comments}
Summarize other comments
\begin{enumerate}
  \item Time effect or a time trend
  \item A lot of work, collaboration and learning happens over the Internet rather than due to physical proximity. Ask of the sample if virtual proximity matters more than physical proximity
  \item Does proximity to peer firms indeed increase the innovation outcome of firms or is it just changing the goal post (like inflation) and not particularly changing the relative standings. It is running faster to stay where you are or is it really improving outcome?
  \item Proximity benefits may just be due to knowledge flows from common customers and common suppliers rather then from geography
  \item Participation in community clubs, children activities and other local events increases opportunities for employees of different firms to interact: What kind of conversations are likely to ensue. How likely are they to be instruments of innovation
  \item Are all firms assumed to be homogenous in all other ways? What are the implications of relaxing that assumption?
  \item Could geographically distant peer firms have other advantages, including a superior capability to sense and address local markets. Are product markets assumed to be homogenous? How about factor markets, especially the human capital market?
  \item Is it assumed that geographical proximity leads to greater alliance opportunities? Are distant firms at a necessary disadvantage while allying?
  \item Does distancing from one group of peer firms, not potentially allow for greater proximity to other firms?

  \item Shock or Quasi-experiment to tease out causality
  \item Knowledge flows from proximity may help in prevention of early extinction (use the right term), rather than help improve outcomes directly
  \item Do cohesive networks naturally better off in accommodating varied views. Can highly cohesive groups tend toward herding rather than diversity?



  \item Random effects
  \item Mechanism behind the inefficient network approach (could there also be a negative effect out of such little commonality) Prior studies may have suggested an inverted-U relationship.
\end{enumerate}

\begin{singlespace}
\renewcommand{\refname}{REFERENCES}
\bibliography{/Users/aiyenggar/code/bibliography/aiyenggar} 
\bibliographystyle{ai-amjlike}
\end{singlespace}

\end{document}
