%1521001##9##.tex
\documentclass[12pt,letterpaper]{article}
\usepackage{mathptmx}
\usepackage[margin=1in]{geometry}

\usepackage{setspace}
\singlespacing
  
\usepackage{amssymb,latexsym}
\usepackage[round,sort]{natbib}
\usepackage{fancyhdr}
\usepackage{lastpage}
\usepackage{graphicx}
\graphicspath{ {qe9/} }

% Bold Table and Figure captions
\usepackage{caption}
\captionsetup{figurename=FIGURE}
\captionsetup{tablename=TABLE}
\captionsetup[figure]{labelfont=bf}
\captionsetup[table]{labelfont=bf}
  
% Turns off all section numbering
\setcounter{secnumdepth}{0} 

  % Places all tables at end of document and creates AOM-style table-here placeholders
  \usepackage[nolists]{endfloat} % Places all figures and charts at end of manuscript and adds 'insert table x about here' lines.
  \renewcommand{\figureplace}{
    \begin{center}
    \begin{singlespace}
    ------------------------------------\\
    Insert \figurename \ \thepostfig\ about here.\\
    ------------------------------------
    \end{singlespace}
    \end{center}}
  \renewcommand{\tableplace}{%
    \begin{center}
    \begin{singlespace}
    ------------------------------------\\
    Insert \tablename \ \theposttbl\ about here.\\
    ------------------------------------
    \end{singlespace}
    \end{center}}

  \usepackage{titlesec}
   \titleformat{\title}
       {\filcenter\normalfont\bfseries\uppercase}{\thetitle}{1em}{}
  \titleformat{\section}
    {\filcenter\normalfont\bfseries\uppercase}{\thesection}{1em}{}
  \titleformat{\subsection}
    {\normalfont\bfseries}{\thesubsection}{1em}{}
  \titleformat{\subsubsection}[runin]
   {\normalfont\bfseries\slshape}{\thesubsubsection}{1em}{\hspace*{\parindent}}
       
\usepackage{tabu}
\usepackage{textcomp}
\usepackage{listings}
\usepackage{hyperref}
\usepackage{verbatim}
\usepackage{tabu}
\hypersetup{
    colorlinks=true,
    linkcolor=blue,
    filecolor=cyan,      
    urlcolor=cyan,
    citecolor=blue,
}

\usepackage{etoolbox}

\makeatletter

% Patch case where name and year are separated by aysep
\patchcmd{\NAT@citex}
  {\@citea\NAT@hyper@{%
     \NAT@nmfmt{\NAT@nm}%
     \hyper@natlinkbreak{\NAT@aysep\NAT@spacechar}{\@citeb\@extra@b@citeb}%
     \NAT@date}}
  {\@citea\NAT@nmfmt{\NAT@nm}%
   \NAT@aysep\NAT@spacechar\NAT@hyper@{\NAT@date}}{}{}

% Patch case where name and year are separated by opening bracket
\patchcmd{\NAT@citex}
  {\@citea\NAT@hyper@{%
     \NAT@nmfmt{\NAT@nm}%
     \hyper@natlinkbreak{\NAT@spacechar\NAT@@open\if*#1*\else#1\NAT@spacechar\fi}%
       {\@citeb\@extra@b@citeb}%
     \NAT@date}}
  {\@citea\NAT@nmfmt{\NAT@nm}%
   \NAT@spacechar\NAT@@open\if*#1*\else#1\NAT@spacechar\fi\NAT@hyper@{\NAT@date}}
  {}{}

\lstset{
basicstyle=\ttfamily,
columns=flexible,
breaklines=true
}
\newenvironment{hypothesis}{
  	\itshape
  	\leftskip=\parindent \rightskip=\parindent
  	\noindent\ignorespaces}

\fancypagestyle{plain}{
  \renewcommand{\headrulewidth}{0pt}
  \fancyhf{}
}	


\begin{document}
\title{Methodological Critique of \cite{Funk2014}\vspace{-2cm}}
\date{}
\maketitle

\pagestyle{fancy}
\fancyhf{}
\lhead{Methodological Critique of \cite{Funk2014}}
\rhead{\thepage}

Funk, R. J. 2014. Making the most of where you are: Geography, networks, and innovation in organizations. \textit{Academy of Management Journal}, 57(1): 193-222.\vspace{1cm}

Russell Funk argues that while geographical proximity to industry peers improves the innovation output of firms, this relationship is moderated by the efficiency of intra-firm information networks. Funk suggests that the underlying mechanism driving that relationship is that diversity facilitates information processing. I found the study interesting for two reasons. First, the geographical effects of innovation is an open area for research and there is still a lot we do not quite understand about the role of geographical proximity in fostering innovation. Second, Funk frames the question as an interaction of two factors - geographical proximity and intra-firm network, both of which are neither completely exogenous nor completely endogenous to firm performance. Driving causal relationship in such an environment is extremely hard and challenging. I have structured my comments in the four following sections. For each I provide a summary of my critique and follow it up with numbered comments. The comments are ordered in decreasing order of importance (i.e., the most important comments come first).


\section{Theory}
While it is commonly believed that localization of patent citations are indicative of knowledge flows, the empirical support for this is not particularly convincing. I would have liked a rigorous theoretical basing for the knowledge flow argument with explanation of underlying mechanisms. Second, it is ambiguous what Funk's assumptions are about managerial behavior. For the most part it seemed like he had assumed boundedly rational agents, but I would like that to be clarified. My detailed comments follow.
\begin{enumerate}
  \item While Funk focuses on the options ahead of the focal firm in terms of geographical proximity and intra-firm information networks, are other firms assumed to be guided strategically by their managers. Is there an assumption that all industry peers are required to take a similar strategic direction?
  \item It was not clear if Funk assumed if intra-organizational information networks evolve over time. In explaining the network as driving innovation outcomes, it seemed to me that an assumption been made to the contrary.
  \item Funk does not provide evidence for the presence of knowledge spillovers. \cite{Arora2017a} suggest that patent citations may not be indicative of knowledge flows. It seems that Funk assumes that local spillovers take place, building on earlier work. Recent work is challenging this assumption. I would have liked to see either theoretical justification or snapshots in the data to support this assumption.
  \item Funk seems to assume that physical proximity necessarily leads to more frequent interactions. In the Internet age, many people are more likely to be conferencing with, working with and collaborating with in their area of specialty who do not live anywhere near. Additionally annual conference meetings may sometimes be the only time neighbors meet. The assumption that living in the same city or county should automatically increase frequency of meetings is problematic. 
  \item Funk assumes that a diverse information network necessarily leads to positive knowledge recombination activities. However other authors have found an inverted-U relationship between network diversity and effectiveness. Why would this result hold any differently?
   \item Funk argues that proximity allows firms to capture large volumes of knowledge through spillovers from nearby organizations. This would be incomplete without also considering absorptive capacity of recipient firms. Infact, the argument is probably stronger the other way. Those with strong absorptive capacity may reach out even if they are not geographically proximate.
\end{enumerate}

\section{Empirical Methodology}
I highlight in this section my concerns around the endogeneity of location choice and intrafirm information networks to firms\textquotesingle \ performance. I also highlight concerns around assumptions about the impact of geography on people\textquotesingle s  interactions and nature of conversations there in.  I would have liked to see some stylized facts or statistics that lend support to one of Funk\textquotesingle fundamental assumptions. 
\begin{enumerate}
  \item I am concerned that the sample selection methodology chooses to drop all satellite centers from the sample. Specifically there is no mention about the size characteristics of these satellite centers. I am particularly concerned about the sample dropping large satellite centers and retaining smaller HQ centers. There is a potential bias introduced from eliminating all satellite R\&D facilities. I would like to see how the size of the satellite facilities dropped vis-a-vis of those retained.
  \item If intrafirm information links are exogenous to firm innovation performance, one would expect that all information links those that lead to joint patent authorship and those that failed to receive a patent authorship would be considered. However, it seems to me that links are assumed of co-inventor links from successful patents, Could there be a case of a sample selection bias? Would not all the knowledge flows that happened but did not end in co-inventor links on a successful patent application be dropped from the analysis?
    \item Combinations are calculated based on USPTO classification system that includes over 100000 subclasses. Using this measure as the means of measuring diversity implies that the diversity may be overstated. Much of those subclasses may map to the same broad technology area. This may end up overstating the effect of inefficiency and may therefore call into question the primary argument in the article.
  \item I am concerned that firm locations and intrafirm information networks may be endogenous and that firms may self select into regions and employees into intrafirm information networks. I would like to see an exogenous shock of a location move by some firms in the sample to identify causal effects of location and intrafirm information networks on innovation outcomes.
   \item Funk determines that firms that do not list at least one inventor with an address in the same region as a firm\textquotesingle s main research facility are excluded. This may be problematic on two counts. For many firms in many locations that operate a global R\&D model, a majority of patent inventors may reside outside the US. This strategy may over count US flows. Additionally, it is not clear if the nanotechnology industry builds on globally sourced R\&D. I would like to see statistics about how many of the inventors on the patents on his panel were resident in the US
  \item Patent authors may sometimes move location. The paper is not clear on how shifting locations is handled.
  \item I found that some of the control variables are not independent of each other. Potential simultaneity or endogeneity among control variables may lead to biased estimates.
  \item There seems to be a confounding of the location of the facility and that of the inventor. It is not clear the location of which is affecting the results. It may have been better to consider inventor locations alone to determine knowledege spillover effects.
  \item Using Euclidean distance to measure proximity may be problematic since different cities and regions in the US are clustered or spread out differently. For example, Houston is much more spread out than is Boston. 5 miles may mean different things in different places. It may have been better to use a city or metro area definition.
  \item The California control is not particularly convincing. For it to be valid, the assumptions about Silicon Valley and San Diego would have to be valid for the nanotechnology technology?
\end{enumerate}

\section{Results}
The results of intrafirm networks\textquotesingle \ interaction effects on innovation outcomes are questionable for two reasons: first because the measure for intrafirm diversity may be over stated, and second because the direct effect is insignificant but the interaction term is insignificant. Why would that be?
\begin{enumerate}
  \item Can the relationship hypothesized in Hypothesis 2a changed in direction and still hold? How would it be possible to eliminate the possibility of simultaneity here?
  \item Since random effects model is used but no statistics or fixed effects results are reported even in robustness checks, I wonder what would the results be if a fixed effects model were employed. How would that change the interpretation of the results?
\end{enumerate}

\section{Other Comments}
I describe here some of my concerns and comments regarding alternative explanations
\begin{enumerate}
  \item Does proximity to peer firms indeed increase the innovation outcome of firms or is it just changing the goal post and not particularly changing the relative standings. This could have been demonstrated using pre-post analysis around an exogenous event like a new invention or by anchoring returns off an exogenous base like the historical patenting performance rate across industries.
  \item Proximity benefits may just be due to knowledge flows from common customers and common suppliers rather then from geography.
  \item Funk suggests that participation in community clubs, children activities and other local events increases opportunities for employees of different firms to interact: Is it necessary that productive work related conversations are likely to ensue. How likely are they to be instruments of innovation?
  \item Funk seems to assume that all firms are  homogenous in all other ways other than their intrafirm information network divesity? What are the implications of relaxing that assumption?
  \item Could geographically distant peer firms have other advantages, including a superior capability to sense and address local markets. Are product markets assumed to be homogenous? What if distant location is a mechanism to improve labor force productivity (due to fewer distractions, less attrition)?
  \item Is it assumed that geographical proximity leads to greater alliance opportunities? Are distant firms at a necessary disadvantage while allying? Why should that be the case?
  \item Does distancing from one group of peer firms, not potentially allow for greater proximity to other firms, and thereby create other advantages?
  \item Knowledge flows from proximity may both help and cause overload, rather than help improve outcomes directly. How can this argument be refuted?
  \item Why would cohesive networks be naturally better off in accommodating varied views. Can highly cohesive groups tend toward herding rather than diversity? Why would that not apply in this case? Prior studies may have suggested an inverted-U relationship with both cohesion and diversity.
  \item Mechanism behind the inefficient network approach (could there also be a negative effect out of such little commonality) \end{enumerate}

\begin{singlespace}
\renewcommand{\refname}{REFERENCES}
\bibliography{/Users/aiyenggar/code/bibliography/aiyenggar} 
\bibliographystyle{ai-amjlike}
\end{singlespace}

\end{document}
