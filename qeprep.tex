%qeprep.tex
\documentclass[12pt,letterpaper]{article}
\usepackage{mathptmx}
\usepackage[margin=1in]{geometry}

\usepackage{setspace}
\doublespacing
  
\usepackage{amssymb,latexsym}
\usepackage[round,sort,longnamesfirst]{natbib}
\usepackage{fancyhdr}
\usepackage{lastpage}
\usepackage{graphicx}
\graphicspath{ {qeprep/} }

% Bold Table and Figure captions
\usepackage{caption}
\captionsetup{figurename=FIGURE}
\captionsetup{tablename=TABLE}
\captionsetup[figure]{labelfont=bf}
\captionsetup[table]{labelfont=bf}
  
% Turns off all section numbering
\setcounter{secnumdepth}{0} 

  % Places all tables at end of document and creates AOM-style table-here placeholders
  \usepackage[nolists]{endfloat} % Places all figures and charts at end of manuscript and adds 'insert table x about here' lines.
  \renewcommand{\figureplace}{
    \begin{center}
    \begin{singlespace}
    ------------------------------------\\
    Insert \figurename \ \thepostfig\ about here.\\
    ------------------------------------
    \end{singlespace}
    \end{center}}
  \renewcommand{\tableplace}{%
    \begin{center}
    \begin{singlespace}
    ------------------------------------\\
    Insert \tablename \ \theposttbl\ about here.\\
    ------------------------------------
    \end{singlespace}
    \end{center}}

  \usepackage{titlesec}
   \titleformat{\title}
       {\filcenter\normalfont\bfseries\uppercase}{\thetitle}{1em}{}
  \titleformat{\section}
    {\filcenter\normalfont\bfseries\uppercase}{\thesection}{1em}{}
  \titleformat{\subsection}
    {\normalfont\bfseries}{\thesubsection}{1em}{}
  \titleformat{\subsubsection}[runin]
   {\normalfont\bfseries\slshape}{\thesubsubsection}{1em}{\hspace*{\parindent}}
       
\usepackage{tabu}
\usepackage{textcomp}
\usepackage{listings}
\usepackage{hyperref}
\usepackage{verbatim}
\usepackage{tabu}
\usepackage{titling}
\usepackage{titlesec}
\hypersetup{
    colorlinks=true,
    linkcolor=blue,
    filecolor=cyan,      
    urlcolor=cyan,
    citecolor=blue,
}
\usepackage{etoolbox}

\makeatletter

% Patch case where name and year are separated by aysep
\patchcmd{\NAT@citex}
  {\@citea\NAT@hyper@{%
     \NAT@nmfmt{\NAT@nm}%
     \hyper@natlinkbreak{\NAT@aysep\NAT@spacechar}{\@citeb\@extra@b@citeb}%
     \NAT@date}}
  {\@citea\NAT@nmfmt{\NAT@nm}%
   \NAT@aysep\NAT@spacechar\NAT@hyper@{\NAT@date}}{}{}

% Patch case where name and year are separated by opening bracket
\patchcmd{\NAT@citex}
  {\@citea\NAT@hyper@{%
     \NAT@nmfmt{\NAT@nm}%
     \hyper@natlinkbreak{\NAT@spacechar\NAT@@open\if*#1*\else#1\NAT@spacechar\fi}%
       {\@citeb\@extra@b@citeb}%
     \NAT@date}}
  {\@citea\NAT@nmfmt{\NAT@nm}%
   \NAT@spacechar\NAT@@open\if*#1*\else#1\NAT@spacechar\fi\NAT@hyper@{\NAT@date}}
  {}{}

\lstset{
basicstyle=\ttfamily,
columns=flexible,
breaklines=true
}
\newenvironment{hypothesis}{
  	\itshape
  	\leftskip=\parindent \rightskip=\parindent
  	\noindent\ignorespaces}

\fancypagestyle{plain}{
  \renewcommand{\headrulewidth}{0pt}
  \fancyhf{}
}	


\begin{document}
\setlength{\droptitle}{-5em}
\title{Preparation notes for Strategy comprehensive exam}
\date{\vspace{-12ex}}
\maketitle

\begin{comment}
\begin{abstract} 
\normalsize 
\end{abstract}
{\textbf{Keywords:} \\\indent }
\end{comment}

\pagestyle{fancy}
\fancyhf{}
\chead{Preparation notes for Strategy comprehensive exam}
\rhead{\thepage}
\lhead{1521001}

\section{Phenomena}

\subsection{Cognition}
\subsubsection{Process Theories}

\subsection{Decision Making}
\subsubsection{Process Theories}

\subsection{Innovation}
\subsubsection{Process Theories}
\subsubsection{Content Theories}

\subsection{Entrepreneurship}
\subsubsection{Process Theories}
\subsubsection{Content Theories}

\subsection{Strategy Formulation}
\subsubsection{Process Theories}

\subsection{Strategic Change}
\subsubsection{Process Theories}

\subsection{Inter-organizational Strategy}
Maybe this is not different from Alliances and Vertical Integration. Check this
\subsubsection{Process Theories}

\subsection{Diversification}

\subsection{Acquisitions}

\subsection{Divestitures}

\subsection{Vertical Integration}

\subsection{Alliances}

\subsection{International Strategy}

\subsection{Emerging Economy Strategy}

\subsection{Organization Knowledge}

\subsection{Organization Learning}

\subsection{Human Capital}

\subsection{Competitive Dynamics}

%--------------------------------------------------
\newpage
%--------------------------------------------------

\section{Content Theories}
\subsection{Industrial Organization Economics}

\subsection{Resource Based View}

\subsection{Dynamic Capabilities}

\subsection{Evolutionary Perspectives}

\subsection{Transaction Cost Economics}

\subsection{Institutional Perspectives}

\subsection{Network Theory}

\subsection{Complexity Theory}


%--------------------------------------------------
\newpage
%--------------------------------------------------

\section{Process Theories}



%--------------------------------------------------
\newpage
%--------------------------------------------------

\section{Classics}
\subsection{A Behavioral Theory of the Firm}
\subsection{The Theory of the Growth of the Firm}
\subsection{The External Control of Organizations}
\subsection{Managing the Resource Allocation Process}
\subsection{Strategy and Structure}
\subsection{Strategy, Structure and Process}
\subsection{The Concept of Corporate Strategy}

%--------------------------------------------------
\newpage
%--------------------------------------------------

\section{International Business}

\subsection{Clusters}
What are clusters?
The economic geography perspective: location of economic activity 
The strategic management perspective: the organization of economic activity

Two perspectives in economic geography
von Thunen, The Isolated State, 1826 with rings of specialty
Hoteling EJ 1929, sellers agglomerate and minimize differentiation (linear transportation costs incentivises this)

(Agglomeration) MAR \cite{Marshall1890} increasing returns to scale due to supplier, \cite{Arrow1962}, learning by doing, \cite{Romer1986} Knowledge returns increasing returns to scale
(Agglomeration) Porter 1998 HBR, Knowledge is important
(Urbanization) \cite{Jacobs1969} inter-industry diversity

Bathelt et al., 2004
* Lorenzen and Mudambi, JEG, 2013
Connectivity through personal or organizational

*Shaver and Flyer, SMJ, 2000
*Cantwell and Mudambi, GSJ, 2011
Strategic deterrence vs. physical attraction (social networks)

*McCann and Mudambi, Environ Plann A,  2005
The table suggesting three different types of agglomerations and their properties
Consequences for rents. For development of human capital. For the development of knowledge or know how.

For MNCs, there is the spatial footprint. Then there is the organizational architecture. You can look at antecedents and consequences.

\subsection{Global Value Chains}
What are clusters?
The economic geography perspective: location of economic activity 
The strategic management perspective: the organization of economic activity
\citep{Mudambi2008, Mudambi2010}
Value capture and value migration

Disaggregation of the value chain into ever more specialized activities – “fine slicing” \citep{Mudambi2008}
Trade in activities or “tasks” rather than goods

Orchestrators and Specializers
Lower value in standardized (repititive) activities (the middle of the smile). What will it take for R\&D to become standardized. Or marketing for that matter?

More and more activities are getting standardized due to fine slicing *Andersson, et al., J World Bus, 2015. When will the slice get too thin for sufficient value capture or for increased risk of failure?

\subsubsection{How does fine slicing work?} \citep{Jacobides2005b}

\textbf{Standardization of Information}
\indent Ubiquitous understanding of information \cite{Arrow1974, Argyres1999a}
\indent Ability to send information across boundaries (Monteverde, 1995; Malone et al 1987)
\indent Reduction of the Lemons problem (Akerlof, 1970; Barzel, 1982)
\indent Reduction of asset specificity \citep{Williamson1985, Klein1978}
\par
\textbf{Simplification of Coordination}
\indent Reduction of asset specificity \citep{Williamson1985, Klein1978}
\indent Ability to solve coordination problems across units \citep{Tushman1978, Puranam2003}
\indent Ability to independently manage each segment of the value chain \citep{Thompson1967, Baldwin2000, Robertson1995}
\par


%--------------------------------------------------
\newpage
%--------------------------------------------------

\section{Industrial Organization}
This entire course is on theories that help define the boundaries and the structure and conduct of firms.
\subsection{Transaction Cost Economics}
\subsection{Information Economics}
\subsection{Agency Theory}
\subsection{Property Rights Theory}
\subsection{New Institutional Economics}
\subsection{Implications to Managerial Decision Making}


%--------------------------------------------------
\newpage
%--------------------------------------------------

\section{Economics of Innovation}
\subsection{The History and the Philosophy of Economic Growth}
\subsection{Economic History and Innovation}
\subsection{Methods in Studying Innovation}
\subsection{Institutions and Innovation}
\subsection{Diffusion of Innovation}
\subsection{Firm Effects in Innovation Research}
\subsection{Appropriating Value from Innovation}


%--------------------------------------------------
\newpage
%--------------------------------------------------

\section{Strategic Networks and Alliances}


%--------------------------------------------------
\newpage
%--------------------------------------------------

\section{Methodological Critique}
\textbf{General Areas to Cover}
In addition to commenting on the theoretical development of a submission and the technical correctness of the methodology, you should also consider the overall value-added contribution the submission offers. Does the submission pass the so what test? Also, consider whether the submission has any practical value, and comment on its implications for the practice community.
Specific Areas to Consider
The following points are some suggested criteria that might help you structure your evaluations of the submissions sent to you.
\par
\subsection{Introduction}
Is there a clear research question, with a solid motivation behind it?
Is the research question interesting?
After reading the introduction, did you find yourself motivated to read further?
\par
\subsection{Theory}
Does the submission contain a well-developed and articulated theoretical framework? Are the core concepts of the submission clearly defined?
Is the logic behind the hypotheses persuasive?
Is extant literature appropriately reflected in the submission, or are critical references missing?
Do the hypotheses or propositions logically flow from the theory?
Alternate explanation
Problem with unit of analysis
\par
\subsection{Method (for empirical papers)}
Are the sample and variables appropriate for the hypotheses?
Is the data collection method consistent with the analytical technique(s) applied?
Does the study have internal and external validity?
Are the analytical techniques appropriate for the theory and research questions and were they applied appropriately.
Sample + data
construct validity
endogeneity
survivor bias or sampling on the dependent variable
choice of estimation model
RELogit - Cannot do fixed effects. What and why? \cite{Zhou2011}
Use Logit when you have [0,1]
\par
\subsection{Results (for empirical papers)}
Are the results reported in an understandable way?
Are there alternative explanations for the results, and if so, are these adequately controlled for in the analyses?
fixed effects
choice of estimation model
one-tail vs two-tail
economic significance
For count variable, use xinb, poisson or negative binomial

\subsection{Contribution}
Does the submission make a value-added contribution to existing research?
Does the submission stimulate thought or debate?
Do the authors discuss the implications of the work for the scientific and practice community?

\subsection{Extensions}
\subsection{Coming up with ideas for research}
Following are some ideas\par
Literature says that X leads to Y. You work on demonstrating what leads to X.\par
Theory predicts outcome A. But you do not see outcome A, but an A'\par
Pit theory X against theory Y. Add contingency variables to explain.\par
Puzzling empirical fact (Ashwin: Why don’t all factors that can affect a situation, actually do?)\par


%--------------------------------------------------
\newpage
%--------------------------------------------------

\begin{comment}
\section{Acknowledgements}
I am heavily indebted to Abhoy Ojha for having set the bar for this paper high. I am indeed the biggest beneficiary of the expectation of as full a paper as possible. My ability to do this work was also contributed to by Sai Yayavaram, who introduced me in June 2015 to the wonderful world of computational modeling, and to Phanish Puranam who helped me with thinking about simple agent based models for modeling organizational phenomena. All mistakes made here are completely mine. 
\end{comment}

\newpage
\begin{singlespace}
\bibliography{/Users/anu/code/bibliography/aiyenggar} 
\bibliographystyle{ai-amjlike}
\end{singlespace}

\begin{comment}
\newpage
\appendix
\begin{singlespace}
\section{APPENDIX A: Simulation Code}
\lstinputlisting{amjlike/matchingEmbeddedAgency.py}
\end{singlespace}

\begin{hypothesis}
{Hypothesis 1a: When the institutional field is open to influence, slow learning adversarial agents will raise overall performance higher than slow learning agents with a neutral orientation\\}
\end{hypothesis}

\begin{hypothesis}
{Hypothesis 2a: For the same initial outcome preferences,  the overall performance score varies curvilinearly with difference in the rates of learning of the agent and the institutional field\\}
\end{hypothesis}

\end{comment}

\end{document}
