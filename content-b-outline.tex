%content-b-outline.tex
\documentclass[12pt]{article}
\usepackage{times}
\usepackage{amssymb,latexsym}
\usepackage[round,sort]{natbib}
\usepackage{fancyhdr}
\usepackage{lastpage}
\usepackage{graphicx}
\usepackage[T1]{fontenc}
\usepackage{mathptmx}
\usepackage{tabu}
\usepackage{textcomp}
\usepackage{stata}
\usepackage{listings}
\usepackage[a4paper]{geometry}
\geometry{
 total={160mm,247mm},
 left=25mm,
 top=25mm,
}
\newenvironment{hypothesis}{
  	\itshape
  	\leftskip=\parindent \rightskip=\parindent
  	\noindent\ignorespaces}
	
\pagestyle{fancy}
\fancyhf{}
\fancyhead{}
\fancyfoot{}
\lhead{Outline: Internationalization Problem}
\rfoot{Page \thepage  \ of \pageref{LastPage}}
\rhead{Ashwin Iyenggar (1521001)}

\begin{document}
\title{Outline: Internationalization Problem}
\author{Ashwin Iyenggar  (1521001) \\ ashwin.iyenggar15@iimb.ernet.in} 


\maketitle
\thispagestyle{empty}

\section{Abstract}


\section{Method}

\question{1}{ State the research question you are exploring in the seminar.}
How do the patenting patterns of inventors whose first patent was co-invented with someone living abroad compare with that of inventors whose first patent was co-invented with someone living in the same location.

I am interested in understanding the impact of cross-national inventions on subsequent patenting career of inventors in weak IPR (or emerging) countries. Do they end up patenting more from the weak IPR country or do they eventually move to a strong IPR country. Do they develop more local inventors or do they patent more with international collaborators?

\question{2}{ What is the motivation for your study? Why should anyone care about your research? Is there a fundamental reference in the literature for your particular study? What is it?}
I am interested in understanding the spillover effects of patenting in emerging countries. Specifically, do multinational firms that develop patentable technologies in emerging countries end up creating spillover effects in the host country talent pool, or do the benefits remain localized to within MNCs. My focus in on human capital, and I wish to understand the impact of a allowing MNC's dominate the patenting process in emerging markets on the quality of the talent pool in the host country. Specifically, is there a significant group of local inventors who develop? Do they then move around to cross-pollinate to other firms? Or do domestic firms get completely left out. Does the talent pool demonstrate a clear MNC vs Domestic company division, i.e., to employees who have previously worked at MNCs move only to other MNCs, and those who had previously worked at Domestic Firms only continue to do so? We may try and understand this from patent data though this may be a large question to be asked for all roles. But the specific context of highly skilled, technologically intensive inventive roles are definitely interesting to look at.

\question{3}{ What is the main dependent variable in your study? What are your alternative choices of dependent variables?}
My dependent variable will be the propensity to switch between MNC and Domestic Firms or vice versa (differentiating between the two could be another study)

\question{4}{ What is the main explanatory variable in your study? What are your alternative choices for the main explanatory variable?}

\question{5}{ What are the control variables in your study?}

\question{6}{ What econometric method do you propose to use for the analysis and why?}

\question{7}{ Explain the dataset in use for your study. What is the source of the data? How was it put together?}

\question{8}{ Explain the identification strategy for your study. What variation are you exploiting in your estimation?}

\question{9}{ Given your choices of the econometric method and identification strategy, is the dataset appropriate for your analysis? Does the manner in which the data were put together induce any biases?}

\question{10}{ What are the potential alternative explanations for your hypothesized / expected results.}

\question{11}{ How do you plan to address the potential confounding effects of the alternative explanations?}

\question{12}{ Are your expected results relevant to any stakeholders such as the businesses, governments, or consumers? Explain.}

\question{13}{ What is your data visualization strategy? In particular, what will you show in Figure 1 to draw the reader's attention.}

\question{14}{ What is your data tabulation strategy? What will you show in Table 1 of your paper that communicates your story?}

\question{15}{ The first regression table is the most important component of your evidence. Have you thought about what kinds of analyses you plan to show in this table? How many rows of variables will you show? How many columns will be present? What additional information do you intend to show? Explain your thought process.}

\question{16}{ Research is about discovery. It begins with asking a novel question. You have been handed a brief by the instructor at the beginning of the seminar. Have you largely stuck to it or have you made any efforts to go beyond the brief and formulate a novel research question. Have you thought about novel mechanisms and effects that may be at work in your research setting and context? Have you thought about how you can examine them and isolate individual effects? If you could continue to work on the research in your Master's thesis and collect additional data, what would you do? Finally does your study tell me, the instructor, anything new?}


\bibliography{/Users/aiyenggar/OneDrive/code/bibliography/ae,/Users/aiyenggar/OneDrive/code/bibliography/fj,/Users/aiyenggar/OneDrive/code/bibliography/ko,/Users/aiyenggar/OneDrive/code/bibliography/pt,/Users/aiyenggar/OneDrive/code/bibliography/uz} 
\bibliographystyle{apalike}

\end{document}
