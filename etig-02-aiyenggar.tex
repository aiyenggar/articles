\documentclass{beamer}
\usetheme{Singapore}
\usepackage[round,sort]{natbib}
\usepackage{tikz}
\usetikzlibrary{arrows,decorations.pathmorphing,backgrounds,fit,positioning,shapes.symbols,chains}
\usepackage{adjustbox}

\title{Economic History and Innovation}
\subtitle{A Review of Readings}
\author{Ashwin Iyenggar}
\institute[Indian Institute of Management Bangalore] 
{
  Corporate Strategy and Policy\\
  Indian Institute of Management Bangalore
}
\date{07 January, 2017}
\subject{Review of Assigned Readings on Economic History}

% \pgfdeclareimage[height=0.5cm]{university-logo}{university-logo-filename}
% \logo{\pgfuseimage{university-logo}}

\AtBeginSubsection[]
{
  \begin{frame}<beamer>{Outline}
    \tableofcontents[currentsection,currentsubsection]
  \end{frame}
}

\begin{document}

\begin{frame}
  \titlepage
\end{frame}

\begin{frame}{Outline}
  \tableofcontents
  % You might wish to add the option [pausesections]
\end{frame}

\section{Overview}
\begin{frame}{Economic History and Innovation}{}
Articles this week
\begin{itemize}
\item{\cite{Temin1997}}
\item{\cite{Epstein1998}}
\item{\cite{Gray2013}}
\item{\cite{Khan2001}}
\item{\cite{Khan1993}}
\item{\citet{Khan2004}}
\item{\cite{Mokyr2010}}
\item{\cite{Schumpeter1947}}
\item{\cite{Moser2013}}
\end{itemize}
\end{frame}

\section{\cite{Mokyr2010}}
\begin{frame}{The Contribution of Economic History to the Study of Innovation and Technical Change}{Summary}
\begin{itemize}
\item{Industrial revolution changed dynamics of how innovation comes about \& the speeds of invention \& diffusion}
\item{Technological component of economic modernity caused by a set of intellectual \& ideological changes that altered the way Europeans interacted with their physical environment. Not due to growth of foreign trade / Growing use of coal / Emergence of urban bourgeoisie}
\end{itemize}
\end{frame}

\begin{frame}{The Contribution of Economic History to the Study of Innovation and Technical Change}{Technology and economic modernity}
\begin{itemize}
\item{Technology is produced within the system by men}
\item{It is non-rivalrous. Debate on how best to establish optimal incentives in innovative activity}
\item{It is produced under uncertainty - Unknown outcomes \& Unintended consequences}
\item{Sets the agenda for scientists}
\item{Declining access costs for technology helps economic modernity}
\end{itemize}
\end{frame}

\begin{frame}{The Contribution of Economic History to the Study of Innovation and Technical Change}{Technology in a "Malthusian Economy"}
\begin{itemize}
\item{Traditional societies - "Culture of improvement"}
\item{18th century transition of handling useful knowledge. Empirical, unsystematic, tacit set of "understandings" --> Collecting \& Analysing in systematic, organised fashion}
\item{Paradox: Productivity growth fails to lead to long term improvements in living standards. "Iron law" of wages rules in the long run}
\item{Most inventions made by artisans. Organised as conservative guilds}
\end{itemize}
\end{frame}

\begin{frame}{The Contribution of Economic History to the Study of Innovation and Technical Change}{A new approach in the first industrial revolution}
\begin{itemize}
\item{Industrial Enlightenment. Bacon's dream that useful knowledge would become a "rich storehouse for the Glory of the Creator and relief of Man's estate"}
\item{Most of 18th century's natural philosophy consisted of 3 Cs: counting, cataloguing \& classifying}
\item{Malthusian \& epistemic constraints broken because propositional knowledge got better at informing technology; feedback from improved technology into more knowledge}
\item{Push for progress on a wide front in second half of 18th century}
\item{First Industrial Revolution's importance is due to Western economies' ability to sustain technological progress and avoid negative feedbacks \& hard constraints}
\end{itemize}
\end{frame}

\begin{frame}{The Contribution of Economic History to the Study of Innovation and Technical Change}{The transition to modern growth, 1830-1880}
\begin{itemize}
\item{Growth in transport technology- most spectacular. Railroad technological history, an eg. of "hybrid" technology}
\item{Special purpose tools used}
\item{Immediate impact of Lavoisier revolution in chemistry on industrial practices}
\item{Useful knowledge in form of "Mechanical science" in early 19th century Britain} 
\end{itemize}
\end{frame}

\begin{frame}{The Contribution of Economic History to the Study of Innovation and Technical Change}{The second industrial revolution}
\begin{itemize}
\item{Large scale electricity generation}
\item{French, American adoption of automobile technology}
\item{Ship design changes, Screw propellor}
\item{Development of food preservation \& preparation - reduced food borne diseases}
\end{itemize}
\end{frame}

\begin{frame}{The Contribution of Economic History to the Study of Innovation and Technical Change}{A suggested interpretation}
\begin{itemize}
\item{Dynamic of innovation began to change in 18th century in the West. Made possible due to a set of institutional developments}
\item{A market for ideas- Efficiency judged by consensus, contestability, cumulativeness}
\item{An open source System of knowledge creation emerged in Europe before Industrial Revolution}
\item{In 18th century, Coercion and Repression were relegated to marginal roles in the market for ideas; Accommodation between religion and the search for useful knowledge; Britain didn't expropriate the profits of innovators \& entrepreneurs}
\item{Technological advances, a result of both discrete quantum leaps in knowledge \& of small incremental and cumulative microinventions}
\end{itemize}
\end{frame}


\section{\cite{Moser2013}}
\begin{frame}{Patents and Innovation: Evidence from Economic History}{Summary}
\begin{itemize}
\item{What is the optimal IPR system to encourage innovation?}
\item{Is there one?}
\end{itemize}
\end{frame}

\begin{frame}{Patents and Innovation: Evidence from Economic History}{Patent laws and the rate of innovation}
\begin{itemize}
\item{Innovation outside the patenting system - 1851 Crystal Palace}
\item{Alternative: secrecy and lead time}
\item{Contradictory effects of patenting on encouraging economic growth}
\end{itemize}
\end{frame}

\begin{frame}{Patents and Innovation: Evidence from Economic History}{Plant patents in 1930 and innovation}
\begin{itemize}
\item{Plant Patents with over half between 1930 - 1970 for roses}
\end{itemize}
\end{frame}

\begin{frame}{Patents and Innovation: Evidence from Economic History}{Secrecy and the direction of technical change}
\begin{itemize}
\item{Secrecy is effective}
\item{Countries without patent laws used secrecy}
\end{itemize}
\end{frame}

\begin{frame}{Patents and Innovation: Evidence from Economic History}{Diffusion of innovation}
\begin{itemize}
\item{Diffusion effects have been largely ignored}
\item{Patent laws seem to affect direction of technological change and diffusion, but the effect is not causal or necessary}
\end{itemize}
\end{frame}

\begin{frame}{Patents and Innovation: Evidence from Economic History}{Patent pools and the mechanism to modify patent laws}
\begin{itemize}
\item{Patent pools expected to weaken the intensity of competition}
\item{Results show that improvements slowed after pooling, only to recover after breaking it down}
\end{itemize}
\end{frame}

\begin{frame}{Patents and Innovation: Evidence from Economic History}{Compulsory Licensing}
\begin{itemize}
\item{Trading with the enemy act, 20\% increase in domestic patenting}
\item{Effect may be delayed}
\item{Problems with uncodified knowledge}
\end{itemize}
\end{frame}

\begin{frame}{Patents and Innovation: Evidence from Economic History}{Conclusion}
\begin{itemize}
\item{Many inventors have avoided patents when possible}
\item{Granting strong IP rights to early generations of inventors may discourage innovation}
\end{itemize}
\end{frame}


\section{\cite{Schumpeter1947}}
\begin{frame}{The Creative Response in Economic History}{Summary}
\begin{itemize}
\item{Adaptive response in the presence of changing conditions}
\end{itemize}
\end{frame}

\section{Stepping Back}
\subsection{Assessing the landscape}
\begin{frame}{Perspectives}{}
\begin{itemize}
\item{Multiple perspectives on effectiveness of patenting in encouraging innovation, diffusion}
\end{itemize}
\end{frame}


\bibliography{/Users/aiyenggar/OneDrive/code/bibliography/ae,/Users/aiyenggar/OneDrive/code/bibliography/fj,/Users/aiyenggar/OneDrive/code/bibliography/ko,/Users/aiyenggar/OneDrive/code/bibliography/pt,/Users/aiyenggar/OneDrive/code/bibliography/uz}
\bibliographystyle{apalike}

\end{document}
