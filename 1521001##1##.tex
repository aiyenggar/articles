%1521001##1##.tex
\documentclass[12pt,letterpaper]{article}
\usepackage{mathptmx}
\usepackage[margin=1in]{geometry}

\usepackage{setspace}
\singlespacing
  
\usepackage{amssymb,latexsym}
\usepackage[round,sort]{natbib}
\usepackage{fancyhdr}
\usepackage{lastpage}
\usepackage{graphicx,multirow}
\graphicspath{ {qe1/} }

% Bold Table and Figure captions
\usepackage{caption}
\captionsetup{figurename=FIGURE}
\captionsetup{tablename=TABLE}
\captionsetup[figure]{labelfont=bf}
\captionsetup[table]{labelfont=bf}
  
% Turns off all section numbering
\setcounter{secnumdepth}{0} 

  % Places all tables at end of document and creates AOM-style table-here placeholders
  \usepackage[nolists]{endfloat} % Places all figures and charts at end of manuscript and adds 'insert table x about here' lines.
  \renewcommand{\figureplace}{
    \begin{center}
    \begin{singlespace}
    ------------------------------------\\
    Insert \figurename \ \thepostfig\ about here.\\
    ------------------------------------
    \end{singlespace}
    \end{center}}
  \renewcommand{\tableplace}{%
    \begin{center}
    \begin{singlespace}
    ------------------------------------\\
    Insert \tablename \ \theposttbl\ about here.\\
    ------------------------------------
    \end{singlespace}
    \end{center}}

  \usepackage{titlesec}
   \titleformat{\title}
       {\filcenter\normalfont\bfseries\uppercase}{\thetitle}{1em}{}
  \titleformat{\section}
    {\filcenter\normalfont\bfseries\uppercase}{\thesection}{1em}{}
  \titleformat{\subsection}
    {\normalfont\bfseries}{\thesubsection}{1em}{}
  \titleformat{\subsubsection}[runin]
   {\normalfont\bfseries\slshape}{\thesubsubsection}{1em}{\hspace*{\parindent}}
       
\usepackage{tabu}
\usepackage{textcomp}
\usepackage{listings}
\usepackage{hyperref}
\usepackage{verbatim}
\usepackage{tabu}
\hypersetup{
    colorlinks=true,
    linkcolor=blue,
    filecolor=cyan,      
    urlcolor=cyan,
    citecolor=blue,
}

\usepackage{etoolbox}

\makeatletter

% Patch case where name and year are separated by aysep
\patchcmd{\NAT@citex}
  {\@citea\NAT@hyper@{%
     \NAT@nmfmt{\NAT@nm}%
     \hyper@natlinkbreak{\NAT@aysep\NAT@spacechar}{\@citeb\@extra@b@citeb}%
     \NAT@date}}
  {\@citea\NAT@nmfmt{\NAT@nm}%
   \NAT@aysep\NAT@spacechar\NAT@hyper@{\NAT@date}}{}{}

% Patch case where name and year are separated by opening bracket
\patchcmd{\NAT@citex}
  {\@citea\NAT@hyper@{%
     \NAT@nmfmt{\NAT@nm}%
     \hyper@natlinkbreak{\NAT@spacechar\NAT@@open\if*#1*\else#1\NAT@spacechar\fi}%
       {\@citeb\@extra@b@citeb}%
     \NAT@date}}
  {\@citea\NAT@nmfmt{\NAT@nm}%
   \NAT@spacechar\NAT@@open\if*#1*\else#1\NAT@spacechar\fi\NAT@hyper@{\NAT@date}}
  {}{}

\lstset{
basicstyle=\ttfamily,
columns=flexible,
breaklines=true
}
\newenvironment{hypothesis}{
  	\itshape
  	\leftskip=\parindent \rightskip=\parindent
  	\noindent\ignorespaces}

\fancypagestyle{plain}{
  \renewcommand{\headrulewidth}{0pt}
  \fancyhf{}
}	


\begin{document}
\title{Inducing Strategic Initiatives at a Startup Firm:\\Understanding the Role of the Co-founding Team}
\date{}
\maketitle

\begin{abstract} 
\normalsize I propose here, a study to understand the process of inducing strategic initiatives at a startup firm. Building on the model of strategy as a process of guided evolution \citep{Lovas2000}, I suggest that co-founders at a startup firm induce strategic initiatives by orchestrating influence on variation, retention and selection of ideas, people and projects over time. Using a comparative case study of Paypal and Billpoint at around the turn of the 20th century, I hope to enrich the \cite{Lovas2000} model in the empirical context of the digital economy.
\end{abstract}


{\textbf{Keywords:} \\\indent co-founders, process model, strategic initiatives, variation selection retention}

\newpage
\pagestyle{fancy}
\fancyhf{}
\lhead{Inducing Strategic Initiatives at a Startup Firm}
\rhead{\thepage}

\begin{center}
\textbf{Inducing Strategic Initiatives at a Startup Firm: Understanding the Role of the Co-founding Team}\vspace{1cm}
\end{center}

\begin{quotation}
\textit{I disagree and commit all the time. We recently greenlit a particular Amazon Studios original. I told the team my view: debatable whether it would be interesting enough, complicated to produce, the business terms aren\textquotesingle t that good, and we have lots of other opportunities. They had a completely different opinion and wanted to go ahead. I wrote back right away with \textbf{``I disagree and commit and hope it becomes the most watched thing we\textquotesingle ve ever made."} Consider how much slower this decision cycle would have been if the team had actually had to convince me rather than simply get my commitment.}\par
\null\hfill \textbf{\textit{Jeff Bezos}}, reflecting on a recent \textit{strategic initiative} at Amazon \citep{Bezos2016}.
\end{quotation}

There has been a debate among scholars about whether strategic initiatives are induced or are autonomous. Present the two points of view. In a startup firm, a certain set of things are at play. Therefore one may very well assume that all strategic initiatives are autonomous. If such is the case, how to leaders at startup firms guide the strategic initiatives in the direction of the co-founders\textquotesingle strategic intent?

The tradition of process research applied to strategic initiatives goes at least as far back as \cite{Bower1970}. What did Bower do. What did Mintzberg suggest. Quinn. Burgelman 1983. Refer to their main models here (provided in the appendix).

Lead into the evolutionary framing of process models in the variation selection retention framework. Mention \cite{Burgelman1991, Burgelman1994, Noda1996, Lovas2000}

Define strategic initiatives and strategic intent in the variation selection retention framework. Suggest why \cite{Lovas2000} model is an appropriate one to study the process of inducing strategic initiatives at firms. Suggest without reference that co-founding team at a startup firm execute the role of middle and higher levels of management in larger firms. Refer to characteristics of startup firms without citation.

\section{Research Proposal}
\subsection{Research Question}

\subsection{Data and Method}
What is the sample. Why is it chosen. How is it going to contribute.

\subsection{Hypotheses}

\section{Limitations}
Suggest how this study may help inform the literatures that it is drawing from, and the interesting research avenues it will open up. Discuss level of generalizability.

\section{Summary}
Recap and motivate interest in framework, in theoretical value as well as in the particular empirical setting.

\begin{comment}
\section{Notes}
\section{Introduction}

\subsection{Strategic Initiatives}
\cite{Lovas2000} define strategic initiatives as ``a deliberate effort by a firm at creating or appropriating economic value from the environment, which is organized as an independent project with its own profit and loss responsibility". They additionally note that that not all strategic initiatives survive the initial stages to become a part of the firm's product line. A strategic initiative could be canceled during, or after, product development, during attempts to establish a commercially viable manufacturing system, or even after market introduction, due to lack of customer interest. Strategic initiatives are subject to evolutionary processes and ecological forces in both an external and an internal market. 

In the external market, the evolutionary process is defined by the introduction of a new product or service in the marketplace
(variation), its initial capacity for appropriating resources in the external environment (selection), and its capacity to do so in later time
periods (retention). These processes take place in the context of the other products and services (competition). The internal processes differ from the external in two main ways. First, competition is not between strategic initiatives competing to appropriate
resources in the external environment, but between the alternative strategic initiatives a firm may choose to invest their resources in pursuing. Second, the processes of variation (which strategic initiatives are suggested), selection (which are started), and retention (which are retained) are not determined by market forces, but guided by beliefs about how a strategic initiative will perform (variation and selection) or informed by knowledge of how well it is actually performing.

\subsection{Characteristics of a startup firm}


\subsection{Role of co-founders in a startup firm}


\subsection{Model of Inducing Strategic Initiatives}
Strategic initiatives may be induced in a firm through a series of actions. Recruitment. Financing. Spinning of as an independent project.
Argue why strategic initiatives are well suited to an evolutionary model - because there is an adequate role for both managerial intervention as well as environmental selection. Seondly, there will be enough variance for the internal selective system to operate
on.

In the form of a detailed literature review, describe the \cite{Lovas2000} model and compare it to prior models of \cite{Burgelman1991, Burgelman1994}. Earlier work include \cite{Bower1970}, \cite{Mintzberg1978}, \cite{Quinn1980}, \cite{Burgelman1983b} Highlight the key aspects of the framework, how it is different and what salient aspects stand out that you then build on top of in this article. Suggest why you think \cite{Lovas2000} model is appropriate. Draw out a table Table ~\ref{payoffmatrix} that compares the various theories and their assumptions, traditions, and recommendations.

\begin{table}
\begin{centering}
\caption {Comparison}
\label{payoffmatrix}
{\tabulinesep=1.4mm
\begin{tabu}{|c|c|c|c|}
\hline
          & Bower Model & Burgelman Model & Lovas and Ghoshal Model \\
\hline   
    Tradition & Theta & Beta  & God \\
\hline    
    Assumptions & Alpha & Hexa  & None \\
\hline    
    Recommendations & Rationalist & Behavioral & Prgamatist \\
\hline 
\end{tabu}}

\end{centering}
\end{table} 

\section{Theory}
You also need to bring in literature discussing the phenomenon (startup firm strategy). Maybe look up strategy entrepreneurship journal.

\subsection{The \cite{Lovas2000} Model}
either redraw or paste the picture
\subsection{Variation, Selection, Retention}
Bring in the model here.

This should lead to the research question in the form of a gap in the current literature. Describe the question, and why it is important to study it (what additional insight will it provide, what policy/strategic implications will it have).

Develop hypotheses. Use theoretical arguments to lay out an interesting conundrum and then attempt to answer that in the form of propositions.
\subsection{On the topic of the general hypotheses}
 Figure ~\ref{fig:3a} lays out the average score charts for four agent-field combinations while enforcing the field to start in Right of Center (this is the same as saying $p_{0,F}^0 = 0.75$). 
\subsubsection{Leading into H1a}
We do so since the scale is symmetric across the Center (C), any initial mapping 

\begin{hypothesis}
{Hypothesis 1a: When the institutional field is open to influence, slow learning adversarial agents will raise overall performance higher than slow learning agents with a neutral orientation\\}
\end{hypothesis}

\subsubsection{Leading into H2a}
This trend is confirmed further in Figure ~\ref{fig:3a} where the learning rates of agents are increased even further to \textquotesingle Fast\textquotesingle .

\begin{hypothesis}
{Hypothesis 2a: For the same initial outcome preferences,  the overall performance score varies curvilinearly with difference in the rates of learning of the agent and the institutional field\\}
\end{hypothesis}

\section{Method}
Then describe an empirical setting in the form of a quasi-experimental setup where this will be examined.
\section{Sample Selection}
Use ideas from \cite{Burgelman1991} Implications

\section{Limitations and Future Work}
Suggest how this study may help inform the literatures that it is drawing from, and the interesting research avenues it will open up. Discuss level of generalizability.


\subsection{\cite{Lovas2000} Model}
The model consists five main elements, viz., (1) strategic initiatives and human and social capital, which are the units of selection; (2) strategic intent, which defines the objective function; (3) administrative systems, which facilitate the evolutionary process; sources of variation; and (5) agents of selection and retention in the evolutionary process, both which potentially include every employee of company.

\begin{figure}[h]
\begin{centering}
  \caption{The five elements of guided evolution, adopted from \cite{Lovas2000}}
  \includegraphics[width=\textwidth]{Lovas2000a}
  \label{fig:Lovas2000a}
\end{centering}
\end{figure}

\begin{figure}[h]
\begin{centering}
  \caption{A model of strategic management as guided evolution, adopted from \cite{Lovas2000}}
  \includegraphics[width=\textwidth]{Lovas2000b}
  \label{fig:Lovas2000b}
\end{centering}
\end{figure}

\subsection{Role of management and employees}
source: \cite{Lovas2000} The members of the top management group had five main responsibilities: (1) to develop and articulate strategic goals which defined the strategic intent of the organization; (2) to sponsor strategic initiatives; (3) to allocate financial capital to strategic initiatives; (4) to recruit people to the organization; and (5) to take responsibility for the development of one area of functional expertise and knowledge in the organization. All employees who were not members of the top management group had two main responsibilities: (1) to work on at least two strategic initiatives at any given point of time; and (2) to have experience in two or more functional areas, in at least one of which he or she had to be an expert. In addition, some employees served as project managers, but then only as a temporary role for the duration of the project.

\subsection{Administrative Systems}
In the guided evolution model, the purpose of administrative systems is role. Its purpose is not to control the retention of predefined strategies, but to help manage the coevolution of strategic initiatives and human and social capital on a distributed basis. More specifically, the intention is to ensure that the variation, selection, and retention of strategic initiatives and human and social capital are informed by the local knowledge of people within the firm \citep{Lovas2000}.

\subsection{Key Problem}
A central problem of evolution in cultural and social systems is the tension between the creation of new variants (strategic initiatives) versus the retention of previously selected variants (strategic initiatives).

\subsection{Strategic Intent}

\cite{Lovas2000} suggest that By 'strategic intent' we mean those long-term goals that reflect the preferred future position of the firm, as articulated by its top management (Prahalad and Doz, 1987).

initiatives. In any given time period, the results and cash flows associated with strategic initiatives that compete for resources in the external environment provide insight into the appropriateness of a firm's strategic intent \citep{Lovas2000}

Strategic intent includes the ability to envision a desired leadership position, to establish the cri- terion used to chart organizational progress towards that end, and the active management process required to accomplish the intent. This process focuses attention on the essence of winning, motivates people by communicating the target, recognizes individual and team contributions, sustains enthusiasm and consistently guides resource allocations.

Strategic intent includes the ability to envision a desired leadership position, to establish the criterion used to chart organizational progress towards that end, and the active management process required to accomplish the intent. The concept ?strategic intent? was first popularized in a 1989 Harvard Business Review article by \cite{Hamel1989}.

According to them, strategic intent included the ability to envision a desired leadership posi- tion, to establish the criterion used to chart organizational progress, and the active management process required to accomplish that intent. They described this process as one that focused the organization?s attention on winning; motivated people by communicating the value of the target; recognized individual and team contributions; sustained enthusiasm by providing new operational definitions as circumstances changed; and used intent to consistently guide resource allocations.


The academic strategic management literature has commonly evaluated the empirical reality of strategic intent through the lens of the resourcebased view theory \citep{Barney1991}, or dynamic capabilities theory \citep{Teece1997}. These theories emphasize the need of the firm to con- sider where it intends to be in the future and to develop or acquire the resources and capabilities needed to attain their intended competitive position. As strategic intent was envisioned by \cite{Hamel1989} as a process, it is not surprising that academic research that has used the term strategic intent has typically considered the strategy-making process in organizations that use more qualitative methods. For example, \cite{Noda1996} contrasted the evolution of two regional Bell holding companies created by the breakup of the Bell system to investigate how top management\textquotesingle s strategic intentions set the strategic and structural context that defines the environment for front-line and middle managers. Likewise, \cite{Lovas2000} studied the interrelationship between strategic decisionmaking and administrative systems at a Danish hearing aid company to develop a model of strategy as guided evolution. The limited academic work that has directly focused on strategic intent as a construct suggests that future studies of the link between strategic intent and resource and capability development/acquisition and the strategic decision-making process within organizations would enrich the academic strategic management literature.

\subsection{Optimizing}
As March (1994) has noted, problem in attempting to 'engineer' or guide evolutionary processes in social systems is to specify what part of the system one is to optimize. This is a problem because social systems are nested in space; i.e., they consist of many different parts,
which are interrelated with one another. Because what might be best for one part of the system (e.g., the engineering department) may not be what is in the best interest of another part of the system (e.g., the marketing department), it is necessary to specify clearly what part of the system one wants to optimize on.

\subsection{Role of top management}
\subsubsection{Positive side}
There are two important implications of such coevolution. On the positive side, to the extent top management can influence where and how employees use their time and energy, they can also influence what human and social capital is created and maintained. In the proposed model, this is done in two ways: first, by relying on a strategic intent to guide the evolution of strategic initiatives, thereby influencing the production and replication of the human and social capital that coevolve with them; and, second and related, by
relying on a strategic intent to signal what human and social capital top management expects to be valuable in the future, thereby influencing what skills, knowledge, and business relationships people are motivated to build and maintain \citep{Lovas2000}. 

\subsubsection{Negative side}
On the negative side, to the extent the strategic intent is not providing effective selection pressures in the internal ecological environment, the result may be random drift when undirected changes in the firm's stock of human and social capital accumulate from one time period to another (McKelvey, 1982; Hannan and Freeman, 1989). As a consequence, valuable human and social capital may be gradually lost. Likewise, if the strategic intent is changed, but does not guide the coevolution between strategic initiatives and
human and social capital in an adaptive direction, existing valuable human and social capital may be lost. Finally, if the strategic intent is changed too often, a firm may lose existing human and social capital through too frequent variation (e.g., mutation, recombination, hybridization), and not be able to focus long enough on a certain set of issues to develop and retain valuable human and social capital in any particular. 

\subsection{Guided Evolution in the Pantheon of Strategy Process Research}
In contrast to this view in which autonomous strategic initiatives serve to challenge the formal strategy of the firm, guided evolution is based on the experiences of a firm that has attempted to replicate a natural selection environment within itself. As a consequence, the distinction between induced and autonomous strategic initiatives is not as salient or as useful here as in Burgelman's model. For example, in a process of guided evolution, all strategic initiatives are autonomous in the sense that someone in the organization
initiates them. Yet, they are all induced, in the sense that the process of variation selection retention is guided by a strategic intent that is
defined by top management. As a result of this difference, guided evolution posits a role of top management that is very different from the role one can infer from Burgelman's model. In Burgelman's model, the key task of top management is to resolve the tension between the autonomous and induced strategy processes by acting as the selection filter-i.e., through resource allocation. Yet, as the work of
March and Simon (1958), Quinn (1980), Lindbloom (1959), and others have shown, in practice this role of top management is severely
constrained. These constraints are clearly acknowledged by researchers within the Bower-Burgelman tradition: as has been noted by Noda and Bower (1996: 186), top management's role in shaping the strategic context tends to be retroactive rationalization and their influence on structural context is believed to be severely constrained because of inertial forces. Paradoxically the reasons for most of these constraints are to be found in the institutionalized administrative systems and processes. In other words, the control systems developed to ensure efficient implementation of past strategies end up constraining top management's discretion in later time periods.
In the model of guided evolution, in contrast, the role of top management is primarily twofold: (i) to create a set of administrative systems that would replicate the processes of natural selection within the organization; and (ii) to guide those processes by defining the strategic intent and the units of selection in the evolutionary process. In other words, top management has traded off direct control through the structural context (i.e., the implementation of predefined product-market strategies) against greater control of the strategic intent.

\subsection{\cite{Burgelman1991} Propositions}
PROPOSITION 1. Firms that are relatively successful over long periods of time, say ten years or more, will be characterized by top managements that are concerned with building the quality of the organization's induced and autonomous strategic processes as well as with the content of the strategy itself.

\begin{figure}[h]
\begin{centering}
  \caption{Intraorganizational Ecology of Strategy Making and Organizational Adaptation, adopted from \cite{Burgelman1991}}
  \includegraphics[width=\textwidth]{Burgelman1991}
  \label{fig:Burgelman1991}
\end{centering}
\end{figure}

PROPOSITION 2. Firms that are relatively successful over long periods of time, say ten years or more, will be characterized by maintaining top driven strategic intent while simultaneously maintaining bottoms-up driven internal experimentation and selection processes.

PROPOSITION 3. The population of firms with successful strategic reorientations will contain a significantly higher proportion of firms whose strategic reorientations were preceded by internal experimentation and selection processes than the population of firms with failing strategic reorientations.

\subsection{\cite{Burgelman1991} Conclusions}
The intraorganizational perspective on strategy making also extends frameworks presented by Mintzberg (1978) and Quinn (1982) in the strategic management literature. It does so by documenting more explicitly some of the sources of emergent strategy, by further elucidating the organizational decision processes through which emergent strategies become part of realized strategies (strategic context determination), by identifying feedback mechanisms between realized and intended strategy, and by providing some evidence that logical incrementalism is likely to be variation reducing and may need to be augmented with an autonomous strategic process to enhance long-term organizational survival. The perspective presented in the paper adds some additional dynamism to these earlier frameworks and draws more explicit attention to the simultaneity of multiple strategy-making processes in organizations.

\subsection{\cite{Burgelman1991} Implications}
future research could examine whether consistently successful firms are characterized by top managements' spending efforts on building each organization's strategy-making processes; whether such firms simultaneously exercise induced and autonomous strategic processes; and whether successful reorientations are more likely to be preceded by internal experimentation and selection processes effected through the autonomous strategic process than are the unsuccessful ones. Future research could also examine the possibilities that there may be an optimal level of ambiguity in the concept of strategy (March 1978) and an optimal degree of coupling in the structural context (Weick 1976). This would require studying the working of strategy-making processes in different types of organizations, such as generalists versus specialists (Freeman and Hannan 1983) or defenders, prospectors, analyzers and reactors (Miles and Snow 1978), and under different types of environmental conditions (e.g., Freeman and Hannan 1983, Eisenhardt 1989). 

\subsection{\cite{Burgelman1994} Model}
\begin{figure}[h]
\begin{centering}
  \caption{Forces driving the strategic business-exit process, adopted from \cite{Burgelman1994}}
  \includegraphics[width=\textwidth]{Burgelman1994}
  \label{fig:Burgelman1994}
\end{centering}
\end{figure}

\subsection{\cite{Mintzberg1978} Model}
Figure ~\ref{fig:Mintzberg1978} in the appendix displays the original resource allocation process suggested by \cite{Mintzberg1978}

\subsection{\cite{Quinn1980} Model}
Figure ~\ref{fig:Quinn1980a} and Figure ~\ref{fig:Quinn1980b} in the appendix displays the original resource allocation process suggested by \cite{Quinn1980}

\subsection{\cite{Bower1970} Model}
Figure ~\ref{fig:Bower1970} in the appendix displays the original resource allocation process suggested by \cite{Bower1970}

\subsection{\cite{Burgelman1983b} Model}
Figure ~\ref{fig:Burgelman1983b} in the appendix displays the Bower-Burgelman (B-B) model proposed by \cite{Burgelman1983b}

\subsection{\cite{Noda1996} Notes}
An explicit recognition of inherent organizational complexities, often described as 'possible goal incongruence,' 'information asymmetry,' and 'organizational politics' (e.g., Barnard, 1938; Simon, 1945; Cyert and March, 1963; Crozier, 1964), as well as 'unpredictable' and 'uncontrol- lable' environments (e.g., Schumpeter, 1934; Nelson and Winter, 1982; Thompson, 1967; Pfeffer and Salancik, 1978; Miles, 1982), has led some strategic management scholars to describe how strategy is actually formed instead of prescribing what it should be. Findings from their empirical studies suggest that strategy is, more or less, emergent from lower levels of organizations (e.g., Mintzberg, 1978; Pascale, 1984; Mintzberg and Waters, 1985), whether through trial-and-error learning (Mintzberg and McHugh, 1985), incrementally with logical guidance from the top (Quinn, 1980), or such that small changes are often punctuated by a sudden big change in a relatively short period (Miller and Friesen, 1984; Tushman and Romanelli, 1985; Gersick, 1991). From this strategy process perspective, strategy is 'a pattern in a stream of decisions and actions' (Mintzberg and McHugh, 1985: 161) that are distributed across multiple levels of an organization.

Whereas some of the scholars associated with this line of research see the process as unguided or 'muddling through' (e.g., Lindbloom, 1959; Wrapp, 1967), others see part of top manage- ment's task as intervening in the emergent strat- egy process and attempting to maneuver the enterprise to a preferable course of direction. These scholars explore multilevel managerial activities that shape the strategy process, inter- acting with external and internal forces. Bower (1970) initiated this line of inquiry by conducting an intensive field-based study on strategic planning and capital investment in a large, diversified firm and presenting a parsimonious framework, grounded in the field data, for understanding the interplay of those managerial activities. His pro- cess model was validated by subsequent field studies in different organizational settings and on various strategic processes (see Bower and Doz, 1979, for the details of these studies). It was then further extended by Burgelman (1983a) in his clinical study on internal corporate venturing (ICV) in a large corporation.

The Bower-Burgelman (B-B) process model of strategy making in a large, complex firm depicts multiple, simultaneous, interlocking, and sequential managerial activities over three levels of organizational hierarchy (i.e., front-line or bot- tom, middle, and top managers) and concep- tualizes intraorganizational strategy-making pro- cesses as consisting of four subprocesses: two interlocking bottom-up core processes of 'defi- nition' and 'impetus' and two overlaying corpor- ate processes of 'structural context determination' and 'strategic context determination.' Definition is a cognitive process in which technological and market forces, initially ill defined, are communi- cated to the organization, and strategic initiatives are developed primarily by front-line managers who usually have specific knowledge on tech- nology and are closer to the market (Chakravarthy and Lorange, 1991; Jensen and Meckling, 1992). Impetus is a largely sociopolitical process by which these strategic initiatives are continually championed by front-line managers, and are adopted and brokered by middle managers who, in doing so, put their reputations for good judg- ment and organizational career at stake. The role of top managers is limited in that they do not necessarily have the appropriate knowledge or information to evaluate technical and economic aspects of the strategic initiatives, and tend to rely on the track records or credibility of propos- ing middle managers in making resource allo- cation decisions (Bower, 1970).

Strategic initiatives therefore 'emerge' pri- marily from managerial activities of front-line and middle managers, as implied by the Carnegie school bottom-up problem-solving perspective (Simon, 1945; Cyert and March, 1963; March and Simon, 1965) and suggested in many other descriptive strategy process studies. Nevertheless, top managers can exercise critical influences on these activities by setting up the structural context (i.e., various organizational and administrative mechanisms such as organizational architecture, information and measurement systems, and reward and punishing systems) to reflect the cor- porate objectives, and thereby manipulating the context in which the decisions and actions of lower-level managers are made (Bower, 1970), as suggested by the Harvard top-down administrative perspective (Chandler, 1962; Learned et al., 1965; Andrews, 1971). The development of those stra- tegic initiatives would lead to the refinement or change of the concept of corporate strategy, thereby determining 'strategic context' over time. Strategic context determination is conceived pri- marily as a political process through which middle managers delineate in concrete terms the content of new fields of business development for the corporation and attempt to convince top managers that the current concept of corporate strategy needs to be changed so as to accommo- date successful new business development (Burgelman, 1983a, 1983b).

The central feature of the B-B model is a resource allocation process in which bottom-up strategic initiatives compete for scarce corporate resources and top managers' attention to survive within the corporate contexts-structural and strategic contexts. Burgelman (1991), in his in-depth field study on Intel's corporate renewal, further developed the idea of intraorganizational competition among bottom-up initiatives and proposed an intraorganizational ecological perspective, fol- lowing the variation-selection-retention frame- work of cultural evolutionary theory (Campbell, 1969; Aldrich, 1979; Weick, 1979). Strategic initiatives are identified and examined in the definition process, within the corporate context (variation), are selected out in the impetus pro- cess by corporate context as 'internal selection environment' (selection), and lead to the reinforcement or modification of corporate context (retention). Burgelman (1994) argues that Intel's internal selection environment, particularly its 'maximizing margin-per-wafer-start' resource allocation rule, reflected selective pressures from the product market in ways that helped the firm exit from the increasingly competitive memory business and refocus on microprocessors.

\end{comment}

\begin{singlespace}
\renewcommand{\refname}{REFERENCES}
\bibliography{/Users/aiyenggar/code/bibliography/aiyenggar} 
\bibliographystyle{ai-amjlike}
\end{singlespace}

\newpage
\appendix
\begin{singlespace}
\section{APPENDIX A: Classical Process Models}
\begin{figure}[h]
\begin{centering}
  \caption{The Research Allocation Process, adopted from \cite{Bower1970}}
  \includegraphics[width=0.85\textwidth]{Bower1970}
  \label{fig:Bower1970}
\end{centering}
\end{figure}

\begin{figure}[h]
\begin{centering}
  \caption{Types of Strategies, adopted from \cite{Mintzberg1978}}
  \includegraphics[width=\textwidth]{Mintzberg1978}
  \label{fig:Mintzberg1978}
\end{centering}
\end{figure}

\begin{figure}[h]
\begin{centering}
  \caption{Strategies form in subsystems (involving different people, skills, goals, information, and timing imperatives), adopted from \cite{Quinn1980}}
  \includegraphics[width=\textwidth]{Quinn1980a}
  \label{fig:Quinn1980a}
\end{centering}
\end{figure}

\begin{figure}[h]
\begin{centering}
  \caption{Some typical process steps in logical incrementalism (highly simplified to help visualize a few basic relationships), adopted from \cite{Quinn1980}}
  \includegraphics[width=\textwidth]{Quinn1980b}
  \label{fig:Quinn1980b}
\end{centering}
\end{figure}

\begin{figure}[h]
\begin{centering}
  \caption{Key and peripheral activities in a process model of ICV, adopted from \cite{Burgelman1983b}}
  \includegraphics[width=\textwidth]{Burgelman1983b}
  \label{fig:Burgelman1983b}
\end{centering}
\end{figure}

\end{singlespace}

\end{document}
