%1521001##1##.tex
\documentclass[12pt,letterpaper]{article}
\usepackage{mathptmx}
\usepackage[margin=1in]{geometry}

\usepackage{setspace}
\singlespacing
  
\usepackage{amssymb,latexsym}
\usepackage[round,sort]{natbib}
\usepackage{fancyhdr}
\usepackage{lastpage}
\usepackage{graphicx}
\graphicspath{ {qe1/} }

% Bold Table and Figure captions
\usepackage{caption}
\captionsetup{figurename=FIGURE}
\captionsetup{tablename=TABLE}
\captionsetup[figure]{labelfont=bf}
\captionsetup[table]{labelfont=bf}
  
% Turns off all section numbering
\setcounter{secnumdepth}{0} 

  % Places all tables at end of document and creates AOM-style table-here placeholders
  \usepackage[nolists]{endfloat} % Places all figures and charts at end of manuscript and adds 'insert table x about here' lines.
  \renewcommand{\figureplace}{
    \begin{center}
    \begin{singlespace}
    ------------------------------------\\
    Insert \figurename \ \thepostfig\ about here.\\
    ------------------------------------
    \end{singlespace}
    \end{center}}
  \renewcommand{\tableplace}{%
    \begin{center}
    \begin{singlespace}
    ------------------------------------\\
    Insert \tablename \ \theposttbl\ about here.\\
    ------------------------------------
    \end{singlespace}
    \end{center}}

  \usepackage{titlesec}
   \titleformat{\title}
       {\filcenter\normalfont\bfseries\uppercase}{\thetitle}{1em}{}
  \titleformat{\section}
    {\filcenter\normalfont\bfseries\uppercase}{\thesection}{1em}{}
  \titleformat{\subsection}
    {\normalfont\bfseries}{\thesubsection}{1em}{}
  \titleformat{\subsubsection}[runin]
   {\normalfont\bfseries\slshape}{\thesubsubsection}{1em}{\hspace*{\parindent}}
       
\usepackage{tabu}
\usepackage{textcomp}
\usepackage{listings}
\usepackage{hyperref}
\usepackage{verbatim}
\usepackage{tabu}
\hypersetup{
    colorlinks=true,
    linkcolor=blue,
    filecolor=cyan,      
    urlcolor=cyan,
    citecolor=blue,
}

\usepackage{etoolbox}

\makeatletter

% Patch case where name and year are separated by aysep
\patchcmd{\NAT@citex}
  {\@citea\NAT@hyper@{%
     \NAT@nmfmt{\NAT@nm}%
     \hyper@natlinkbreak{\NAT@aysep\NAT@spacechar}{\@citeb\@extra@b@citeb}%
     \NAT@date}}
  {\@citea\NAT@nmfmt{\NAT@nm}%
   \NAT@aysep\NAT@spacechar\NAT@hyper@{\NAT@date}}{}{}

% Patch case where name and year are separated by opening bracket
\patchcmd{\NAT@citex}
  {\@citea\NAT@hyper@{%
     \NAT@nmfmt{\NAT@nm}%
     \hyper@natlinkbreak{\NAT@spacechar\NAT@@open\if*#1*\else#1\NAT@spacechar\fi}%
       {\@citeb\@extra@b@citeb}%
     \NAT@date}}
  {\@citea\NAT@nmfmt{\NAT@nm}%
   \NAT@spacechar\NAT@@open\if*#1*\else#1\NAT@spacechar\fi\NAT@hyper@{\NAT@date}}
  {}{}

\lstset{
basicstyle=\ttfamily,
columns=flexible,
breaklines=true
}
\newenvironment{hypothesis}{
  	\itshape
  	\leftskip=\parindent \rightskip=\parindent
  	\noindent\ignorespaces}

\fancypagestyle{plain}{
  \renewcommand{\headrulewidth}{0pt}
  \fancyhf{}
}	


\begin{document}
\title{Inducing Strategic Initiatives at a Startup Firm:\\Understanding the Role of the Co-founding Team}
\date{}
\maketitle

\begin{abstract} 
\normalsize 

\end{abstract}


{\textbf{Keywords:} \\\indent }

\newpage
\pagestyle{fancy}
\fancyhf{}
\lhead{Inducing Strategic Initiatives at a Startup Firm}
\rhead{\thepage}

\begin{center}
\textbf{Inducing Strategic Initiatives at a Startup Firm:\\Understanding the Role of the Co-founding Team}
\end{center}

In the form of a detailed literature review, describe the \cite{Lovas2000} model and compare it to prior models of \cite{Burgelman?} and \cite{Bower?}. Highlight the key aspects of the framework, how it is different and what salient aspects stand out that you then build on top of in this article. Suggest why you think \cite{Lovas2000} model is appropriate. Draw out a table Table ~\ref{payoffmatrix} that compares the various theories and their assumptions, traditions, and recommendations.

\begin{table}
\begin{centering}
\caption {Comparison}
\label{payoffmatrix}
{\tabulinesep=1.4mm
\begin{tabu}{|c|c|c|c|}
\hline
          & Bower Model & Burgelman Model & Lovas and Ghoshal Model \\
\hline   
    Tradition & Theta & Beta  & God \\
\hline    
    Assumptions & Alpha & Hexa  & None \\
\hline    
    Recommendations & Rationalist & Behavioral & Prgamatist \\
\hline 
\end{tabu}}

\end{centering}
\end{table} 

\section{Theory}
You also need to bring in literature discussing the phenomenon (startup firm strategy). Maybe look up strategy entrepreneurship journal.

\subsection{The \cite{Lovas2000} Model}
either redraw or paste the picture
\subsection{Variation, Selection, Retention}
Bring in the model here.

This should lead to the research question in the form of a gap in the current literature. Describe the question, and why it is important to study it (what additional insight will it provide, what policy/strategic implications will it have).

Develop hypotheses. Use theoretical arguments to lay out an interesting conundrum and then attempt to answer that in the form of propositions.
\subsection{On the topic of the general hypotheses}
 Figure ~\ref{fig:3a} lays out the average score charts for four agent-field combinations while enforcing the field to start in Right of Center (this is the same as saying $p_{0,F}^0 = 0.75$). 
\subsubsection{Leading into H1a}
We do so since the scale is symmetric across the Center (C), any initial mapping 

\begin{hypothesis}
{Hypothesis 1a: When the institutional field is open to influence, slow learning adversarial agents will raise overall performance higher than slow learning agents with a neutral orientation\\}
\end{hypothesis}

\subsubsection{Leading into H2a}
This trend is confirmed further in Figure ~\ref{fig:3a} where the learning rates of agents are increased even further to \textquotesingle Fast\textquotesingle .

\begin{hypothesis}
{Hypothesis 2a: For the same initial outcome preferences,  the overall performance score varies curvilinearly with difference in the rates of learning of the agent and the institutional field\\}
\end{hypothesis}

\section{Method}
Then describe an empirical setting in the form of a quasi-experimental setup where this will be examined.

\section{Limitations and Future Work}
Suggest how this study may help inform the literatures that it is drawing from, and the interesting research avenues it will open up. Discuss level of generalizability.

\begin{singlespace}
\bibliography{/Users/aiyenggar/code/bibliography/aiyenggar} 
\bibliographystyle{ai-amjlike}
\end{singlespace}

\end{document}
