%etig-termpaper-proposal.tex
\documentclass[12pt]{article}
\usepackage{amssymb,latexsym}
\usepackage[round,sort]{natbib}
\usepackage{multirow,array}
\usepackage{fancyhdr}
\usepackage{lastpage}
\usepackage{graphicx}
\usepackage[bottom]{footmisc}
\usepackage[T1]{fontenc}
\usepackage{mathptmx}
\usepackage{tabu}
\usepackage{textcomp}
\usepackage{stata}
\usepackage{listings}
\usepackage[letterpaper,margin=1.25in]{geometry}
\usepackage{hyperref}
\usepackage{setspace}
\usepackage{pdflscape}
\usepackage{longtable}
\hypersetup{
    colorlinks=true,
    linkcolor=blue,
    filecolor=magenta,      
    urlcolor=cyan,
    citecolor=magenta,
}

\lstset{
basicstyle=\ttfamily,
columns=flexible,
breaklines=true
}
\newenvironment{hypothesis}{
  	\itshape
  	\leftskip=\parindent \rightskip=\parindent
  	\noindent\ignorespaces}
	
\pagestyle{fancy}
\fancyhf{}
\fancyhead{}
\fancyfoot{}
\lhead{ETIG Term Paper Outline}
\rfoot{Page \thepage  \ of \pageref{LastPage}}
\rhead{Ashwin Iyenggar (1521001)}

\begin{document}
\title{Proposal Outline:\\The effect of inventor mobility on  invention complexity}
\author{Ashwin Iyenggar  (1521001) \\ ashwin.iyenggar15@iimb.ernet.in} 


\maketitle
\thispagestyle{empty}
\section{Background}
Several empirical studies have demonstrated the variation in the mobility of inventors across regions. \cite{Almeida1999} suggested that interfirm mobility of engineers influences the local transfer of knowledge.  More recently, \cite{Ge2016} have combined data from linkedin.com and the USPTO to interpret the higher levels of mobility in silicon valley as the outcome of targeted retention of human capital. A question that remains unanswered is if the variation in inventor mobility can also explain the variation in complexity of future inventions.

\section{Research Question}
In this paper, I intend to study the relationship between the movement of some inventors into or out of a region and the average complexity of inventions of employees working in the affected regions. 

\section{Theory}
The received wisdom earlier was that firms would have a greater incentive to keep  highly dependent technology developed in weaker IPR countries secret \citep{Cohen2000}. However  \cite{Zhao2006} has more recently used patent data to argue  that multinational enterprises may benefit from conducting R\&D in countries with weak IPR protection by  making up for the weaker IPR protection through better internal organization. The anecdotal increase in the mobility of employees at the weak IPR subsidiaries raises a potential paradox. If increased mobility of employees influences transfer of knowledge \citep{Almeida1999}, should we expect higher complex inventions from inventors in those teams into which other inventors have moved in? The answer to this question is not completely explained by theory, and is therefore proposed here as an empirical study.

\section{Data and Method}
I propose to integrate data from the USPTO (made available on patentsview.org) and the public data of inventors available on linkedin.com to answer this question. Specifically, I intend to capture at the level of the region-year, the number of incoming and outgoing inventors. I additionally compute the complexity of the invention at the level of the region-year by a composite construct involving the number of subclass combinations of the invention  and the number of subclass combinations of backward citations made. Controlling for the prior pool of patents, I intend to understand the effect of inventor movement into and out of regions on the productivity of those regions.

\section{Challenges}
A primary challenge in a such as this is in understanding the direction of causality. While I do not have an answer for this question, I hope to use the empirical context to explore the possible mechanisms that can help explain the phenomenon.

\bibliography{/Users/aiyenggar/OneDrive/code/bibliography/ae,/Users/aiyenggar/OneDrive/code/bibliography/fj,/Users/aiyenggar/OneDrive/code/bibliography/ko,/Users/aiyenggar/OneDrive/code/bibliography/pt,/Users/aiyenggar/OneDrive/code/bibliography/uz} 
\bibliographystyle{apalike}

\end{document}
