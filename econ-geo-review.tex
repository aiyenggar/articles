%econ-geo-review.tex 
\documentclass[12pt]{article}
\usepackage{times}
\usepackage{amsmath,amssymb,latexsym}
\usepackage[round,sort]{natbib}
\usepackage{multirow,array}
\usepackage{fancyhdr}
\usepackage{lastpage}
\usepackage{graphicx}
\usepackage[bottom]{footmisc}
\graphicspath{ {etig-term-paper-presentation-images/} }
\usepackage[T1]{fontenc}
\usepackage{mathptmx}
\usepackage{tabu}
\usepackage{textcomp}
\usepackage{stata}
\usepackage{listings}
\usepackage[a4paper,margin=1.0in]{geometry}
\usepackage{multirow}
\usepackage{caption}
\usepackage{setspace}
\usepackage{verbatim}
\usepackage{pdflscape}
\usepackage{longtable}
\usepackage{hyperref}
\setlength{\parindent}{4pt}
\setlength{\parskip}{1.2em}
\hypersetup{
    colorlinks=true,
    linkcolor=blue,
    filecolor=magenta,      
    urlcolor=cyan,
    citecolor=blue,
}
\lstset{
basicstyle=\ttfamily,
columns=flexible,
breaklines=true
}
\newenvironment{hypothesis}{
  	\itshape
  	\leftskip=\parindent \rightskip=\parindent
  	\noindent\ignorespaces}
	
\setlength\parindent{0pt}
\pagestyle{fancy}
\fancyhf{}

\lhead{Review of literature on the nature of knowledge spillovers}

\rfoot{Page \thepage  \ of \pageref{LastPage}}
\rhead{Iyenggar}
\newcommand\question[2]{\vspace{1em}\hrule\vspace{1em}\textbf{#1}{ #2}\vspace{1em}\hrule\vspace{1em}}

\begin{document}

\title{\LARGE Review of literature on the nature of knowledge spillovers}

\author{Ashwin Iyenggar  (1521001) \\ ashwin.iyenggar15@iimb.ernet.in} 
\large

\maketitle
\thispagestyle{empty}

\begin{abstract}
\large \noindent 
\end{abstract}
{Keywords:} Economic geography, Clusters, Knowledge Spillovers
\onehalfspacing



%\begin{landscape}
\begin{center}
\section{Notes on individual articles}
 \begin{longtable}{|p{0.20\textwidth}|p{0.60\textwidth}|p{0.20\textwidth}|}
 \caption{Notes on individual articles\label{long}}\\
 
 \hline\textbf{Article}&\textbf{Main Idea}&\textbf{Scholarly Tradition}\\\hline
 \endfirsthead
 
 \hline\textbf{Article}&\textbf{Main Idea}&\textbf{Scholarly Tradition}\\\hline
 \endhead
 
 \hline
 \endfoot
 
 \hline
 \endlastfoot

\cite{Almeida1999} & Regional variation in knowledge spillovers due to institutions and labor networks.&\\\hline
\cite*{Arora2017a} & Localization of patent citation may not be due to localization of knowledge flow. Local knowledge flows may exist but may not be captured by patent citations.&\\\hline

\cite{Audretsch1996} & Prior: Innovation and technological change depend on new economic knowledge. Literature has explored the role spillovers play in generating increasing returns and eventually economic growth. The argument here is that if the ability to receive knowledge spillovers is influenced by distance from knowledge source, then you should find spatial concentration in areas where knowledge spillovers are likely to play an important role. Purpose of \cite{Audretsch1996} is two fold. First, to understand to what extent does industrial activity cluster spatially?, and Second, to link the geographical concentration to the existence of knowledge externalities. \cite{Audretsch1996} control for spatial concentration of location of production, and suggest that while information may be transmitted better due to better communication infrastructure, the same may not be true of knowledge which may have a tacit element. &\\\hline

\cite{Jaffe1989} &Spatially mediated knowledge spillovers.&\\\hline

\cite{Glaeser2009} & Confirm the observation by \cite{Chinitz1961} that new entrants are drawn to regions with small suppliers &\\\hline

%\cite{Asmussen2011} & Knowledge flow from MNC subsidiary to other MNC affiliates requires lower external knowledge accumulation (a contamination) or a minimum threshold of internally sourced knowledge has to achieved (credibility)&\\\hline

\cite{Awate2015} & Diffusion of innovative capabilities goes from advanced country firms to emerging country firms. Since both AMNEs and EMNEs are internationalized, the EMNE headquarters develop innovation capabilities slower than AMNE subsidiaries&\\\hline

\cite{Bahlmann2014} & Local and distant search implies that geographical network diversity affects innovation in an inverted-U manner&\\\hline

\cite{Bell2005} & Disentangle the geographic effect from the effect of networks (managerial ties, and institutional ties)&\\\hline

\cite{Baptista1998} & Maybe diversification within clusters is not particularly useful&\\\hline

\cite{Cantwell2005} & Not just location but group level and subsidiary level mandates matter for R\&D.&\\\hline

%\cite{Chatterji2013} & We understand very little of what works and how it does when it comes to spatial concentration of entrepreneurship and innovation&\\\hline

\cite{Dunlap-Hinkler2010} & Generic innovation capabilities hinders breakthrough performance. Decentralization is helpful.&\\\hline

\cite{Eisingerich2010} & Cluster performance is determined by network effects (Network strength and openness), and is moderated by environmental uncertainty.&\\\hline

\cite{Eriksson2011} & On mobility, but not clear&\\\hline

\cite{Fitjar2015} & International personal and formal networks are correlated with innovation in firms. Local networks not as much. &\\\hline

\cite{Fitjar2016} & Goldilocks distance - innovation works well when geographic distance is neither too near nor too far.&\\\hline

\cite{Fu2012} & Path dependent evolution of regional innovation systems. First mover advantage and institutional first mover advantage.&\\\hline

\cite{Fu2013} & Learning by interacting - informal Guanxi network.&\\\hline

\cite{Giuliani2007} & Firm level social network characteristics as affecting diffusion as against geographical proximity and embeddednesss in local networks (prior). Economists view that public goods are subject to spillover effects. Economic geographists view that embeddedness in firms in localized networks.&\\\hline

\cite{Giuliani2005b} or \cite{Giuliani2005a} & Reversing the direction. Absorptive capacity of firms leads to agglomeration.  Knowledge is not diffused evenly in the air, but flows within a core group of firms with advanced absorptive capacities.&\\\hline

\cite{Grillitsch2017} & Negative spillovers - those with weak internal knowledge  grow faster in knowledge intensive regions&\\\hline

\cite{Henderson2005} & Are three digit patent classes too broad. Underlying forces run both ways. Knowledge spillovers provide incentives to collocate. Colocation (to begin with) may encourage cross pollination. Knowledge spillovers are highly elusive.&\\\hline

\cite{Huber2011} & To what extent does local spillovers help?&\\\hline

\cite{Alnuaimi2012a} or \cite{Alnuaimi2012b}& International R\&D helps inventors but not subsidiary level capabilities. Collaboration is more international and not local.&\\\hline

\cite{Jaffe1993}&&\\\hline

\cite{Krugman1991a} & Core - periphery model of economic differentiation.&\\\hline

\cite{Lissoni2001} & Codified and firm specific knowledge in SME clusters.&\\\hline

\cite{Lorenzen2013} & Personal relationships (prior: Organization based pipelines and MNE subsidiaries). Global linkages with decentralized network structures.&\\\hline

\cite{Mudambi2012} & Role of MNE clusters - a review of literature&\\\hline

\cite{Mudambi2004} & Intra MNC firm flows leads to bargaining power.&\\\hline

\cite{Murata2014} & Localization of flows&\\\hline
\end{longtable}
\end{center}
% \end{landscape}

\section{Limitations in using patents}
\cite{Griliches1990}
\cite{Scherer1984}
\bibliography{/Users/aiyenggar/code/bibliography/aiyenggar} 
\bibliographystyle{apalike}

\end{document}
